% -*- Mode: TeX -*-

%!!! Barmar: Tell me why multiple packags are needed. 
%    e.g., to prevent variable and function conflicts.

\beginsubSection{Introduction to Packages}

%% 10.0.0 7
%% 11.0.0 4
A \term{package} establishes a mapping from names to \term{symbols}. 
At any given time, one \term{package} is current.
The \newterm{current package} is the one that is \thevalueof{*package*}.
When using the \term{Lisp reader},
it is possible to refer to \term{symbols} in \term{packages} 
other than the current one through the use of \term{package prefixes} in the 
printed representation of the \term{symbol}.

%% 11.2.0 5
\Thenextfigure\ lists some \term{defined names} that are applicable
to \term{packages}.
%Shouldn't be needed.  This info is all explicit now. -kmp 29-Apr-91
% For the \term{operators} listed here, all optional arguments named 
% \param{package} default to the \term{current package}.
Where an \term{operator} 
takes an argument that is either a \term{symbol} or a \term{list} 
of \term{symbols},
an argument of \nil\ is treated as an empty \term{list} of \term{symbols}.
Any \param{package} argument may be either a \term{string}, a \term{symbol}, or
a \term{package}.  If a \term{symbol} is supplied, its name will be used
as the \term{package} name.
\issue{REQUIRE-PATHNAME-DEFAULTS:ELIMINATE}
% \funref{provide}, \funref{require}, and \varref{*modules*} will
% be removed from this table (and the language).
%PROVIDE and REQUIRE need to be removed, right? -kmp 29-Apr-91
%Nope! -kmp 13-Feb-92
\displaythree{Some Defined Names related to Packages}{
*modules*&import&provide\cr
*package*&in-package&rename-package\cr
defpackage&intern&require\cr
do-all-symbols&list-all-packages&shadow\cr
do-external-symbols&make-package&shadowing-import\cr
do-symbols&package-name&unexport\cr
export&package-nicknames&unintern\cr
find-all-symbols&package-shadowing-symbols&unuse-package\cr
find-package&package-use-list&use-package\cr
find-symbol&package-used-by-list&\cr
}
\endissue{REQUIRE-PATHNAME-DEFAULTS:ELIMINATE}

\beginsubsubsection{Package Names and Nicknames}

%% 11.0.0 19
Each \term{package} has a \term{name} (a \term{string}) and perhaps some \term{nicknames}
(also \term{strings}).
These are assigned when the \term{package} is created and can be changed later.  

%% 11.0.0 20
There is a single namespace for \term{packages}.  
\Thefunction{find-package} translates a package
\term{name} or \term{nickname} into the associated \term{package}.  
\Thefunction{package-name} returns the \term{name} of a \term{package}.  
\Thefunction{package-nicknames} returns 
a \term{list} of all \term{nicknames} for a \term{package}.
\funref{rename-package} removes a \term{package}'s
current \term{name} and \term{nicknames} and replaces them with new ones
specified by the caller.

\endsubsubsection%{Package Names and Nicknames}

\beginsubsubsection{Symbols in a Package}

\beginsubsubsubsection{Internal and External Symbols}

%% 11.0.0 5
The mappings in a \term{package} are divided into two classes, external and internal.
The \term{symbols} targeted by these different mappings 
are called \term{external symbols} and \term{internal symbols} of the
\term{package}. Within a \term{package}, a name refers to one
\term{symbol} or to none; if it does refer
to a \term{symbol}, then it is either external or internal in that
\term{package}, but not both.
%% 11.0.0 6
\newterm{External symbols} are part of the package's public interface to other
\term{packages}.  \term{Symbols} become \term{external symbols} of a given
\term{package} if they have been \term{exported} from that \term{package}.

%% 11.0.0 7                                 
A \term{symbol} has the same \term{name} no matter what \term{package} 
it is \term{present} in, but it might be an \term{external symbol} of some \term{packages}
and an \term{internal symbol} of others. 

\endsubsubsubsection%{Internal and External Symbols}

\beginsubsubsubsection{Package Inheritance}

%% 11.0.0 9
\term{Packages} can be built up in layers.  From one point of view,
a \term{package} is a single collection
of mappings from \term{strings} into \term{internal symbols} and 
\term{external symbols}.
However, some of these mappings might be established within the \term{package} 
itself, while other mappings are inherited from other \term{packages} 
via \funref{use-package}.
A \term{symbol} is said to be \newterm{present} in a \term{package} 
if the mapping is in the \term{package} itself and is
not inherited from somewhere else.

%% 11.4.0 
There is no way to inherit the \term{internal symbols} of another \term{package};
to refer to an \term{internal symbol} using the \term{Lisp reader}, 
    a \term{package} containing the \term{symbol} 
     must be made to be the \term{current package},
    a \term{package prefix} must be used,
 or the \term{symbol} must be \term{imported} into the \term{current package}.

\endsubsubsubsection%{Package Inheritance}

\beginsubsubsubsection{Accessibility of Symbols in a Package}

A \term{symbol} becomes \newterm{accessible} in a \term{package} 
    if that is its \term{home package} when it is created,
 or if it is \term{imported} into that \term{package},
 or by inheritance via \funref{use-package}.

If a \term{symbol} is \term{accessible} in a \term{package},
it can be referred to when using the \term{Lisp reader}
without a \term{package prefix} when that \term{package} is the \term{current package},
regardless of whether it is \term{present} or inherited.

%???Move the following to \secref\Syntax???
%A \term{symbol} will not necessarily always have the same
%printed representation because the way \term{symbols} are printed
%depends on whether or not they are \term{accessible} in the \term{current package}.

%% 11.0.0 35
\term{Symbols} from one \term{package} can be made \term{accessible} 
in another \term{package} in two ways.

%% 11.0.0 36
\beginlist 
\itemitem{--}
Any individual \term{symbol} can be added to a \term{package} by use
of \funref{import}.  After the call to \funref{import} the
\term{symbol} is \term{present} in the importing \term{package}.
The status of the \term{symbol} in the \term{package} 
it came from (if any) is unchanged, and the home package for                               
this \term{symbol} is unchanged.
Once \term{imported}, a \term{symbol} is \term{present} in the
importing \term{package}
and can be removed only by calling \funref{unintern}.

%% 11.4.0 4
A \term{symbol} is \term{shadowed}\meaning{3} by another \term{symbol} 
in some \term{package} if the first \term{symbol} would be \term{accessible}
by inheritance if not for the presence of the second \term{symbol}.
See \funref{shadowing-import}.

%% 11.4.0 39
%% 11.4.0 40
\itemitem{--}
The second mechanism for making \term{symbols} from one \term{package}
\term{accessible} in another is provided by \funref{use-package}.  
All of the \term{external symbols} of the used \term{package} are inherited
by the using \term{package}.
\Thefunction{unuse-package} undoes the effects of a previous \funref{use-package}.  
\endlist

\endsubsubsubsection%{Accessibility of Symbols in a Package}

\beginsubsubsubsection{Locating a Symbol in a Package}

%% 11.4.0 8
When a \term{symbol} is to be located in a given \term{package} 
the following occurs:
\beginlist 
\itemitem{--} The \term{external symbols} and \term{internal symbols} of the 
\term{package} are searched for the \term{symbol}.
\itemitem{--} The \term{external symbols} of the used \term{packages} are 
searched
in some unspecified order.  The
order does not matter; see the rules for handling name
conflicts listed below. 
\endlist

\endsubsubsubsection%{Locating a Symbol in a Package}

\beginsubsubsubsection{Prevention of Name Conflicts in Packages}

%% 11.0.0 46
Within one \term{package}, any particular name can refer to at most one
\term{symbol}.  A name conflict is said to occur when there would be more than
one candidate \term{symbol}.  Any time a name conflict is about to occur,
a \term{correctable} \term{error} is signaled.  

The following rules apply to name conflicts:
%% 11.0.0 47
\beginlist
%% 11.0.0 49
\itemitem{--}
Name conflicts are detected when they become possible, that is, when the
package structure is altered.  Name
conflicts are not checked during every name lookup.

\itemitem{--}
If the \term{same} \term{symbol} is \term{accessible} to a \term{package} 
through more than one path, there is no name conflict.
A \term{symbol} cannot conflict with itself. 
Name conflicts occur only between \term{distinct} \term{symbols} with
the same name (under \funref{string=}).

%% 11.0.0 48
\itemitem{--} Every \term{package} has a list of shadowing \term{symbols}.  
A shadowing \term{symbol} takes precedence over any other \term{symbol} of
the same name that would otherwise be \term{accessible} in the \term{package}.  
A name conflict involving a shadowing symbol is always resolved in favor of
the shadowing \term{symbol}, without signaling an error (except for one
exception involving \funref{import}).
See \funref{shadow} and \funref{shadowing-import}.

%% 11.0.0 50
\itemitem{--} 
The functions \funref{use-package}, \funref{import}, and 
\funref{export} check for name conflicts.  

%% 11.0.0 52
\itemitem{--} 
\funref{shadow} and \funref{shadowing-import} 
never signal a name-conflict error.

%% 11.0.0 53
\itemitem{--} 
% \funref{unuse-package}, \funref{unexport}, and \funref{unintern} 
% (when the \term{symbol} being uninterned is not a \term{shadowing symbol}) 
% do not need to do any name-conflict checking.
%% Rewording for JonL:
\funref{unuse-package} and \funref{unexport}
do not need to do any name-conflict checking.
\funref{unintern} does name-conflict checking only when a \term{symbol} 
being \term{uninterned} is a \term{shadowing symbol}.

%% 11.0.0 54
\itemitem{--} 
Giving a shadowing symbol to \funref{unintern} 
can uncover a name conflict that had
previously been resolved by the shadowing.  

%% 11.0.0 55
%\itemitem{--} 
%Aborting from a name-conflict error leaves the original \term{symbol} 
%\term{accessible}.

  \itemitem{--} 
  Package functions signal name-conflict errors \oftype{package-error} before making any
  change to the package structure.  When multiple changes are to be made,
  it is
  permissible for the implementation to process each change separately.
  For example, when \funref{export} is given a 
\term{list} of 
\term{symbols},
  aborting from a name
  conflict caused by the second \term{symbol} 
  in the \term{list} might still export the
  first \term{symbol} in the \term{list}.  
  However, a name-conflict error caused by \funref{export}
  of a single \term{symbol} will be signaled before
  that \term{symbol}'s \term{accessibility} in any \term{package} is changed.

%% 11.0.0 56
\itemitem{--} 
Continuing from a name-conflict error must offer the user a chance to
resolve the name conflict in favor of either of the candidates.  The
\term{package} 
structure should be altered to reflect the resolution of the
name conflict, via \funref{shadowing-import}, 
\funref{unintern},
%!!! Barmar: The next two bullets don't mention "unexport".
or \funref{unexport}.

%% 11.0.0 57
\itemitem{--} 
A name conflict in \funref{use-package} between a \term{symbol} 
%directly 
\term{present} in the using \term{package} and an \term{external symbol} of the used 
\term{package} is resolved in favor of the first \term{symbol} by making it a
shadowing \term{symbol}, or in favor of the second \term{symbol} by uninterning
the first \term{symbol} from the using \term{package}. 

%% 11.0.0 60
\itemitem{--} 
A name conflict in \funref{export} or \funref{unintern} 
due to a \term{package}'s inheriting two \term{distinct} \term{symbols} 
with the \term{same} \term{name} (under \funref{string=})
from two other \term{packages} can be resolved in
favor of either \term{symbol} by importing it into the using
\term{package} and making it a shadowing symbol, just as with
\funref{use-package}.
\endlist

\endsubsubsubsection%{Prevention of Name Conflicts in Packages}

\endsubsubsection%{Symbols in a Package}

\endsubSection%{Introduction to Packages}

\beginsubSection{Standardized Packages}

%% 11.6.0 1
This section describes the \term{packages} that are available
in every \term{conforming implementation}.  A summary of the
\term{names} and \term{nicknames} of those \term{standardized} \term{packages} 
is given in \thenextfigure.

\tablefigtwo{Standardized Package Names}{Name}{Nicknames}{
\packref{common-lisp}&\packref{cl}\cr
\packref{common-lisp-user}&\packref{cl-user}\cr
\packref{keyword}&\i{none}\cr
}

\issue{LISP-PACKAGE-NAME:COMMON-LISP}
%% 11.6.0 2
% Discussion of package LISP and USER removed. -kmp 15-Feb-92
\endissue{LISP-PACKAGE-NAME:COMMON-LISP}

\issue{PACKAGE-CLUTTER:REDUCE}
%% 11.6.0 5
% Discussion of the CLtL1 package named SYSTEM removed.
\endissue{PACKAGE-CLUTTER:REDUCE}

\beginsubsubsection{The COMMON-LISP Package} 

\issue{LISP-PACKAGE-NAME:COMMON-LISP}
 
\Thepackage{common-lisp} contains the primitives of the \clisp\ system as
defined by this specification.  Its \term{external} \term{symbols} include
all of the \term{defined names} (except for \term{defined names} in
\thepackage{keyword}) that are present in the \clisp\ system, 
such as \funref{car}, \funref{cdr},  \varref{*package*}, etc.
\Thepackage{common-lisp} has the \term{nickname} \packref{cl}.
 
\issue{PACKAGE-CLUTTER:REDUCE}
\Thepackage{common-lisp} has as \term{external} \term{symbols} those 
symbols enumerated in the figures in \secref\CLsymbols, and no others.
These \term{external} \term{symbols} are \term{present} in \thepackage{common-lisp}
but their \term{home package} need not be \thepackage{common-lisp}.

For example, the symbol \f{HELP} cannot be an \term{external symbol} of
\thepackage{common-lisp} because it is not mentioned in \secref\CLsymbols.
In contrast, the \term{symbol} \misc{variable}
must be an \term{external symbol} of \thepackage{common-lisp} 
even though it has no definition
because it is listed in that section
(to support its use as a valid second \term{argument} to \thefunction{documentation}). 

% Moved this sentence out of previous paragraph.  --sjl 7 Mar 92
\Thepackage{common-lisp} can have additional \term{internal symbols}.
\endissue{PACKAGE-CLUTTER:REDUCE}

\beginsubsubsubsection{Constraints on the COMMON-LISP Package for Conforming Implementations}

\issue{PACKAGE-CLUTTER:REDUCE}
In a \term{conforming implementation},
an \term{external} \term{symbol} of \thepackage{common-lisp} can have
   a \term{function}, \term{macro}, or \term{special operator} definition, 
%top level value  ???
   a \term{global variable} definition
   (or other status as a \term{dynamic variable} 
    due to a \declref{special} \term{proclamation}),
or a \term{type} definition
only if explicitly permitted in this standard.
%% That's the definition of a conforming implementation. -kmp
%that is, a \term{conforming program} may assume that this is true.
For example,
  \funref{fboundp} \term{yields} \term{false} 
  for any \term{external symbol} of \thepackage{common-lisp} 
  that is not the \term{name} of a \term{standardized} 
   \term{function}, \term{macro} or \term{special operator},
and
  \funref{boundp} returns \term{false} 
  for any \term{external symbol} of \thepackage{common-lisp} 
  that is not the \term{name} of a \term{standardized} \term{global variable}.
It also follows that
  \term{conforming programs} can use \term{external symbols} of \thepackage{common-lisp} 
  as the \term{names} of local \term{lexical variables} 
  with confidence that those \term{names} have not been \term{proclaimed} \declref{special} 
  by the \term{implementation}
  unless those \term{symbols} are
    \term{names} of \term{standardized} \term{global variables}.

%%KMP: Initially or for all times? -kmp 2-Jan-91
%%Sandra: The intent was to cover properties put there by the implementation, not the user.
%%KMP: I double-checked the issues, and that seems to be right.
% No \term{external symbols} of \thepackage{common-lisp} 
% can have \term{properties} with \term{property indicators} 
% that are either \term{external symbols} of any \term{standardized} \term{packages}
% or otherwise \term{accessible} in \thepackage{common-lisp-user}.
%% Rewritten. -kmp 15-Feb-92
A \term{conforming implementation} must not place any \term{property} on
an \term{external symbol} of \thepackage{common-lisp} using a \term{property indicator}
that is either an \term{external symbol} of any \term{standardized} \term{package}
or a \term{symbol} that is otherwise \term{accessible} in \thepackage{common-lisp-user}.
 
%Valid programs
%  can assume that the conformal Lisp implementation will not
%  have prohibited properties.  The proposal LISP-SYMBOL-REDEFINITION
%  addresses the converse; that is, what user programs are allowed
%  to do.
\endissue{PACKAGE-CLUTTER:REDUCE}
\endsubsubsubsection%{Constraints on the COMMON-LISP Package for Conforming Implementations}
 
\beginsubsubsubsection{Constraints on the COMMON-LISP Package for Conforming Programs}
\issue{LISP-SYMBOL-REDEFINITION:MAR89-X3J13}
Except where explicitly allowed, the consequences are undefined if any
of the following actions are performed on an \term{external symbol} 
of \thepackage{common-lisp}:

\beginlist
\itemitem{1.} 
\term{Binding} or altering its value (lexically or dynamically).
(Some exceptions are noted below.)
\itemitem{2.} 
Defining or \term{binding} it as a \term{function}.
(Some exceptions are noted below.)
\itemitem{3.} 
Defining or \term{binding} it as a \term{macro}
\issue{DEFINE-COMPILER-MACRO:X3J13-NOV89}
or \term{compiler macro}.
\endissue{DEFINE-COMPILER-MACRO:X3J13-NOV89}
(Some exceptions are noted below.)
\itemitem{4.} 
Defining it as a \term{type specifier} (via \macref{defstruct}, 
% added define-condition --sjl 7 Mar 92
\macref{defclass}, \macref{deftype}, \macref{define-condition}).
\itemitem{5.} 
Defining it as a structure (via \macref{defstruct}).
\itemitem{6.} 
Defining it as a \term{declaration} with a \declref{declaration} \term{proclamation}.
\itemitem{7.} 
Defining it as a \term{symbol macro}.
%% Sandra complained, too.
% \itemitem{n.} 
% Altering its name. %!!! Barmar notes that this can't be done to any symbols.
\itemitem{8.} 
Altering its \term{home package}.
\itemitem{9.} 
Tracing it  (via \macref{trace}).
\itemitem{10.} 
%!!! What's this "or lexical" biz? -kmp 13-May-91
Declaring or proclaiming it
(via \misc{declare},
\issue{PROCLAIM-ETC-IN-COMPILE-FILE:NEW-MACRO}
   \specref{declaim},
\endissue{PROCLAIM-ETC-IN-COMPILE-FILE:NEW-MACRO}
or \funref{proclaim}) 
% we voted down the lexical declaration proposal.  --sjl 7 Mar 92
%\declref{special} or lexical.
\declref{special}.
\itemitem{11.} 
Declaring or proclaiming its 
\declref{type} or \declref{ftype} (via \misc{declare},
\macref{declaim},  or \funref{proclaim}).
(Some exceptions are noted below.)
\itemitem{12.} 
Removing it from \thepackage{common-lisp}.
\endlist 

\beginsubsubsubsubsection{Some Exceptions to Constraints on the COMMON-LISP Package for Conforming Programs}

If an \term{external symbol} of \thepackage{common-lisp}
is not globally defined as a \term{standardized} \term{dynamic variable} 
					      or \term{constant variable},
it is allowed to lexically \term{bind} it 
          and to declare the \declref{type} of that \term{binding}, 
and
it is allowed to locally \term{establish} it as a \term{symbol macro} 
(\eg with \specref{symbol-macrolet}).

%KMP: Maybe clarify that binding CL special variables is ok,
%     but that their type decls are lexical.  Is that right?
%%I implemented the first half of that suggestion. -kmp 15-Feb-92
Unless explicitly specified otherwise,
if an \term{external symbol} of \thepackage{common-lisp} 
is globally defined as a \term{standardized} \term{dynamic variable},
it is permitted to \term{bind} or \term{assign} that \term{dynamic variable}
provided that the ``Value Type'' constraints on the \term{dynamic variable} 
are maintained, and that the new \term{value} of the \term{variable} 
is consistent with the stated purpose of the \term{variable}.

If an \term{external symbol} of \thepackage{common-lisp} is not defined
as a \term{standardized} \term{function}, \term{macro}, or \term{special operator},
it is allowed to lexically \term{bind} it as a \term{function} (\eg with \specref{flet}),
              to declare the \declref{ftype} of that \term{binding}, 
          and 
%KMP: Barmar wanted some explication here.  I think it was sandra who had 
%     asked for this tracing feature just in case the implementation supported it.
              (in \term{implementations} which provide the ability to do so)
	      to \macref{trace} that \term{binding}.

If an \term{external symbol} of \thepackage{common-lisp} is not defined
as a \term{standardized} \term{function}, \term{macro}, or \term{special operator},
it is allowed to lexically \term{bind} it as a \term{macro} (\eg with \specref{macrolet}).
\endissue{LISP-SYMBOL-REDEFINITION:MAR89-X3J13}

\endissue{LISP-PACKAGE-NAME:COMMON-LISP}

\endsubsubsubsubsection%{Some Exceptions to Constraints on the COMMON-LISP Package for Conforming Programs}

\endsubsubsubsection%{Constraints on the COMMON-LISP Package for Conforming Programs}

\endsubsubsection%{The COMMON-LISP Package} 

\beginsubsubsection{The COMMON-LISP-USER Package}

\Thepackage{common-lisp-user} is the \term{current package} when 
a \clisp\ system starts up.  This \term{package} \term{uses} \thepackage{common-lisp}.
\Thepackage{common-lisp-user} has the \term{nickname} \packref{cl-user}.
\issue{PACKAGE-CLUTTER:REDUCE}
\Thepackage{common-lisp-user} can have additional \term{symbols} \term{interned} within it;
it can \term{use} other \term{implementation-defined} \term{packages}.
\endissue{PACKAGE-CLUTTER:REDUCE}
 
\endsubsubsection%{The COMMON-LISP-USER Package}

\beginsubsubsection{The KEYWORD Package}

%% 5.1.2 6
%% 11.3.0 5
% % KMP: This isn't a very apt description! 
% % %% 11.6.0 4
% \Thepackage{keyword} contains all of the \newterm{keywords}\meaning{1}
% used by built-in or user-defined \term{functions}.  
%% Replaced:
\Thepackage{keyword} contains \term{symbols}, called \term{keywords}\meaning{1},
that are typically used as special markers in \term{programs} 
and their associated data \term{expressions}\meaning{1}.

\term{Symbol} \term{tokens} that start with a \term{package marker} 
are parsed by the \term{Lisp reader} as \term{symbols} 
in \thepackage{keyword}; \seesection\SymbolTokens.
This makes it notationally convenient to use \term{keywords}
when communicating between programs in different \term{packages}.  
For example, the mechanism for passing \term{keyword parameters} in a \term{call} uses 
\term{keywords}\meaning{1} to name the corresponding \term{arguments};
\seesection\OrdinaryLambdaLists.

\term{Symbols} in \thepackage{keyword} are, by definition, \oftype{keyword}.

\beginsubsubsubsection{Interning a Symbol in the KEYWORD Package}

\Thepackage{keyword} is treated differently than other \term{packages}
in that special actions are taken when a \term{symbol} is \term{interned} in it.
In particular, when a \term{symbol} is \term{interned} in \thepackage{keyword},
 it is automatically made to be an \term{external symbol} 
and is automatically made to be a \term{constant variable} with itself as a \term{value}.

%% Not really needed. This is documented adequately under SYMBOL-VALUE.
% \Thefunction{symbol-value} can be used to
% \term{access} the \term{value} of a \term{keyword}\meaning{1}.)

\endsubsubsubsection%{Interning a Symbol in the KEYWORD Package}

\beginsubsubsubsection{Notes about The KEYWORD Package}

% name conflicts are not an issue because these \term{symbols}
% are used only as labels, not to carry application-specific information.
%% This isn't really true and is arguably confusing.
%% I think the following is what it was getting at. -kmp 15-Feb-92

It is generally best to confine the use of \term{keywords} to situations in which
there are a finitely enumerable set of names to be selected between.  For example,
if there were two states of a light switch, they might be called \kwd{on} and \kwd{off}.

In situations where the set of names is not finitely enumerable
(\ie where name conflicts might arise)
it is frequently best to use \term{symbols} in some \term{package}
other than \packref{keyword} so that conflicts will be naturally avoided.
For example, it is generally not wise for a \term{program} to use a \term{keyword}\meaning{1} 
as a \term{property indicator}, since if there were ever another \term{program}
that did the same thing, each would clobber the other's data.

\endsubsubsubsection%{Notes about The KEYWORD Package}

\endsubsubsection%{The KEYWORD Package}

\beginsubsubsection{Implementation-Defined Packages} 

\issue{PACKAGE-CLUTTER:REDUCE}
Other, \term{implementation-defined} \term{packages} might be present
in the initial \clisp\ environment.
%If it is appropriate, the standard might contain
%  as an example that implementations might have a package named
%  "SYSTEM".
\endissue{PACKAGE-CLUTTER:REDUCE}

It is recommended, but not required, that the documentation for a
\term{conforming implementation} contain a full list of all \term{package} names
initially present in that \term{implementation} but not specified in this specification.
(See also the \term{function} \funref{list-all-packages}.)

\endsubsubsection%{Implementation-Defined Packages} 

\endsubSection%{Standardized Packages}

