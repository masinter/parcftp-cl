%-*- Mode: TeX -*-
%%%Organization of the Document

This is a reference document, not a tutorial document.  Where possible
and convenient, the order of presentation has been chosen so that the
more primitive topics precede those that build upon them;  however,
linear readability has not been a priority.

This document is divided into chapters by topic.
Any given chapter might contain conceptual material, dictionary entries, or both.

\term{Defined names} within the dictionary portion of a chapter are
grouped in a way that brings related topics into physical proximity.
Many such groupings were possible,
and no deep significance should be inferred from the particular grouping that was chosen.
To see \term{defined names} grouped alphabetically, consult the index.
For a complete list of \term{defined names}, \seesection\CLsymbols.

In order to compensate for the sometimes-unordered portions of this document, 
a glossary has been provided; \seechapter\Glossary.
The glossary provides connectivity by providing easy access to 
definitions of terms, and in some cases by providing examples or 
cross references to additional conceptual material.

For information about notational conventions used in this document,
\seesection\Definitions.

For information about conformance, \seesection\Conformance. 

For information about extensions and subsets, \seesection\LanguageExtensions\
and \secref\LanguageSubsets.

For information about how \term{programs} in the language are parsed by the
\term{Lisp reader}, \seechapter\Syntax.

For information about how \term{programs} in the language are \term{compiled}
and \term{executed}, \seechapter\EvaluationAndCompilation.

For information about data types, \seechapter\TypesAndClasses.
Not all \term{types} and \term{classes} are defined in this chapter;
many are defined in chapter corresponding to their topic--for example,
the numeric types are defined in \chapref\Numbers.
For a complete list of \term{standardized} \term{types}, 
\seefigure\StandardizedAtomicTypeSpecs.

For information about general purpose control and data flow,
\seechapter\DataAndControlFlow\ or \chapref\Iteration.

