% -*- Mode: TeX -*-
%% Rules about Test Functions

\beginsubsection{Satisfying a Two-Argument Test}
\DefineSection{SatisfyingTheTwoArgTest}

When an \term{object} $O$ is being considered iteratively 
against each \term{element} $E\sub i$
of a \term{sequence} $S$
by an \term{operator} $F$ listed in \thenextfigure,
it is sometimes useful to control the way in which the presence of $O$ 
is tested in $S$ is tested by $F$.
This control is offered on the basis of a \term{function} designated with 
either a \kwd{test} or \kwd{test-not} \term{argument}.

%!!! Sandra wonders if this table is complete.
\displaythree{Operators that have Two-Argument Tests to be Satisfied}{
adjoin&nset-exclusive-or&search\cr
assoc&nsublis&set-difference\cr
count&nsubst&set-exclusive-or\cr
delete&nsubstitute&sublis\cr
find&nunion&subsetp\cr
intersection&position&subst\cr
member&pushnew&substitute\cr
mismatch&rassoc&tree-equal\cr
nintersection&remove&union\cr
nset-difference&remove-duplicates&\cr
}

The object $O$ might not be compared directly to $E\sub i$.
If a \kwd{key} \term{argument} is provided,
it is a \term{designator} for a \term{function} of one \term{argument} 
to be called with each $E\sub i$ as an \term{argument}, 
and \term{yielding} an \term{object} $Z\sub i$ to be used for comparison.
(If there is no \kwd{key} \term{argument}, $Z\sub i$ is $E\sub i$.)

% Added per Barmar. -kmp 16-Feb-92
The \term{function} designated by \thekeyarg{key} is never called on $O$ itself.
However, if the function operates on multiple sequences
(\eg as happens in \funref{set-difference}), $O$
will be the result of calling the \kwd{key} function on an
\term{element} of the other sequence.  

A \kwd{test} \term{argument}, if supplied to $F$,
is a \term{designator} for a  \term{function}
of two \term{arguments}, $O$ and $Z\sub i$.
An $E\sub i$ is said (or, sometimes, an $O$ and an $E\sub i$ are said)
to \newterm{satisfy the test} 
if this \kwd{test} \term{function} returns a \term{boolean} representing 
\term{true}.

A \kwd{test-not} \term{argument}, if supplied to $F$, 
is \term{designator} for a \term{function} 
of two \term{arguments}, $O$ and $Z\sub i$.
An $E\sub i$ is said (or, sometimes, an $O$ and an $E\sub i$ are said)
to \newterm{satisfy the test} 
if this \kwd{test-not} \term{function}
returns a \term{boolean} representing \term{false}.

If neither a \kwd{test} nor a \kwd{test-not} \term{argument} is supplied, 
it is as if a \kwd{test} argument of \f{\#'eql} was supplied.

The consequences are unspecified if both a \kwd{test} and a \kwd{test-not} \term{argument}
are supplied in the same \term{call} to $F$.

\beginsubsubsection{Examples of Satisfying a Two-Argument Test}

\code
 (remove "FOO" '(foo bar "FOO" "BAR" "foo" "bar") :test #'equal)
\EV (foo bar "BAR" "foo" "bar")
 (remove "FOO" '(foo bar "FOO" "BAR" "foo" "bar") :test #'equalp)
\EV (foo bar "BAR" "bar")
 (remove "FOO" '(foo bar "FOO" "BAR" "foo" "bar") :test #'string-equal)
\EV (bar "BAR" "bar")
 (remove "FOO" '(foo bar "FOO" "BAR" "foo" "bar") :test #'string=)
\EV (BAR "BAR" "foo" "bar")

 (remove 1 '(1 1.0 #C(1.0 0.0) 2 2.0 #C(2.0 0.0)) :test-not #'eql)
\EV (1)
 (remove 1 '(1 1.0 #C(1.0 0.0) 2 2.0 #C(2.0 0.0)) :test-not #'=)
\EV (1 1.0 #C(1.0 0.0))
 (remove 1 '(1 1.0 #C(1.0 0.0) 2 2.0 #C(2.0 0.0)) :test (complement #'=))
\EV (1 1.0 #C(1.0 0.0))

 (count 1 '((one 1) (uno 1) (two 2) (dos 2)) :key #'cadr) \EV 2

 (count 2.0 '(1 2 3) :test #'eql :key #'float) \EV 1

 (count "FOO" (list (make-pathname :name "FOO" :type "X")  
                    (make-pathname :name "FOO" :type "Y"))
        :key #'pathname-name
        :test #'equal)
\EV 2
\endcode

\endsubsubsection%{Examples of Satisfying a Two-Argument Test}

\endsubsection%{Satisfying a Two-Argument Test}

\beginsubsection{Satisfying a One-Argument Test}
\DefineSection{SatisfyingTheOneArgTest}

When using one of the \term{functions} in \thenextfigure,
the elements $E$ of a \term{sequence} $S$ are filtered
not on the basis of the presence or absence of an object $O$ 
under a two \term{argument} \term{predicate},
as with the \term{functions} described in \secref\SatisfyingTheTwoArgTest,
but rather on the basis of a one \term{argument} \term{predicate}.

%!!! KMP wonders if this table is complete.
\displaythree{Operators that have One-Argument Tests to be Satisfied}{
assoc-if&member-if&rassoc-if\cr
assoc-if-not&member-if-not&rassoc-if-not\cr
count-if&nsubst-if&remove-if\cr
count-if-not&nsubst-if-not&remove-if-not\cr
delete-if&nsubstitute-if&subst-if\cr
delete-if-not&nsubstitute-if-not&subst-if-not\cr
find-if&position-if&substitute-if\cr
find-if-not&position-if-not&substitute-if-not\cr
}

The element $E\sub i$ might not be considered directly.
If a \kwd{key} \term{argument} is provided,
it is a \term{designator} for a \term{function} of one \term{argument} 
to be called with each $E\sub i$ as an \term{argument}, 
and \term{yielding} an \term{object} $Z\sub i$ to be used for comparison.
(If there is no \kwd{key} \term{argument}, $Z\sub i$ is $E\sub i$.)

\term{Functions} defined in this specification and having a name that
ends in ``\f{-if}'' accept a first \term{argument} that is a \term{designator} for a 
\term{function} of one \term{argument}, $Z\sub i$.
An $E\sub i$ is said to \newterm{satisfy the test} if this \kwd{test} \term{function}
returns a \term{boolean} representing \term{true}.

\term{Functions} defined in this specification and having a name that
ends in ``\f{-if-not}'' accept a first \term{argument} that is a \term{designator} for a 
\term{function} of one \term{argument}, $Z\sub i$.
An $E\sub i$ is said to \newterm{satisfy the test} if this \kwd{test} \term{function}
returns a \term{boolean} representing \term{false}.

\beginsubsubsection{Examples of Satisfying a One-Argument Test}

\code
 (count-if #'zerop '(1 #C(0.0 0.0) 0 0.0d0 0.0s0 3)) \EV 4

 (remove-if-not #'symbolp '(0 1 2 3 4 5 6 7 8 9 A B C D E F))
\EV (A B C D E F)
 (remove-if (complement #'symbolp) '(0 1 2 3 4 5 6 7 8 9 A B C D E F))
\EV (A B C D E F)

 (count-if #'zerop '("foo" "" "bar" "" "" "baz" "quux") :key #'length)
\EV 3
\endcode

\endsubsubsection%{Examples of Satisfying a One-Argument Test}

\endsubsection%{Satisfying a One-Argument Test}
