% -*- Mode: TeX -*-

\def\realtypespec#1{
\label Compound Type Specifier Kind::

Abbreviating.

\label Compound Type Specifier Syntax::

\Deftype{#1}{\ttbrac{lower-limit \brac{upper-limit}}}

\label Compound Type Specifier Arguments::

\param{lower-limit}, \param{upper-limit}---\term{interval designators} 
					   for \term{type} \typeref{#1}.
 \DefaultEach{\param{lower-limit} and \param{upper-limit}}{the \term{symbol} \misc{*}}

\label Compound Type Specifier Description::

This denotes the \term{#1s} on the interval described by
\param{lower-limit} and \param{upper-limit}.
}

% Numbers
%  Numeric Comparison
%  Rounding
%  Trig
%  Hyperbolic Trig
%  Arithmetic
%  Random Numbers
%  Numeric Types
%   Complex
%   Real
%    Rational
%     Integer
%      Integer Bits
%      Integer Bytes
%       Fixnum
%     Ratio
%    Float
%     Float Subtypes

%-------------------- Numeric Types --------------------
\begincom{number}\ftype{System Class}

\label Class Precedence List::
\typeref{number},
\typeref{t}

\label Description::
 
%%The following seems to fluffy to me.  Besides you can find the same info elsewhere
%%in more rigorous form. -kmp 13-Nov-90
% There are \term{real} and \term{complex} \term{numbers}.
% \term{Real} numbers are further divided into \term{floats},
% \term{rationals}, and \term{integers}.
%%Replaced by:
\Thetype{number} contains \term{objects} which represent
mathematical numbers.
\issue{REAL-NUMBER-TYPE:X3J13-MAR-89}
The \term{types} \typeref{real} and \typeref{complex} are \term{disjoint} 
\term{subtypes} of \typeref{number}.
\endissue{REAL-NUMBER-TYPE:X3J13-MAR-89}

%I added this: -kmp 15-Nov-90
\Thefunction{=} tests for numerical equality.
\Thefunction{eql}, when its arguments are both \term{numbers}, 
tests that they have both the same \term{type} and numerical value.
%% 12.0.0 4
Two \term{numbers} that are the \term{same} under \funref{eql} or \funref{=}
are not necessarily the \term{same} under \funref{eq}.

%% 2.1.4 1
%% 2.1.4 4

\label Notes::

\clisp\ differs from mathematics on some naming issues.  In mathematics,
the set of real numbers is traditionally described as a subset of the
complex numbers, but in \clisp, \thetype{real} and \thetype{complex} are
disjoint.  The \clisp\ type which includes all mathematical complex
numbers is called \typeref{number}.  The reasons for these differences
include historical precedent, compatibility with most other popular
computer languages, and various issues of time and space efficiency.

%RWK notes:
% Another way of describing this is that fundamental number types
% fixnum, bignum, xxx-float, ratio, complex, ... all name representations
% and are names for finite subsets of the corresponding mathematical concepts.
% Thus, complex is a type which can represent complex numbers.
% The extra capability of this type means that even if the imaginary part is 0.0,
% it is not a member of the more restrictive representation real.
%
%Ultimately I didn't buy this because it presumes that types are only about
%how a number is represented.  But 1.0 is not represented as a real.  In fact,
%no number is.  Rather, they are represented as subtypes of reals, and so real
%is available only for categorization.  Anyway, I decided to just let things stand
%as they are for brevity and let others intuit other explanations that make them happy.

\endcom%{number}\ftype{System Class}
\begincom{complex}\ftype{System Class}

\label Class Precedence List::
\typeref{complex},
\typeref{number},
\typeref{t}

\label Description::

%RWK: Again, describe it as a representation which is -capable- of representing
%     complex numbers.  1.0 has a different, more restricted representation.

\Thetype{complex} includes all mathematical complex numbers
other than those included in \thetype{rational}.
\term{Complexes} are 
%"represented" -> "expressed".  Some reviewer had problems with "represented". -kmp
expressed
in Cartesian form with a
real part and an imaginary part, each of which is a \term{real}.
The real part and imaginary part are either both
\term{rational} or both of the same \term{float} \term{type}.
The imaginary part can be a \term{float} zero, but can never
be a \term{rational} zero, for such a number is always represented
by \clisp\ as a \term{rational} rather than a \term{complex}.

\label Compound Type Specifier Kind::

Specializing.

\label Compound Type Specifier Syntax::

%% 4.5.0 11
\Deftype{complex}{\ttbrac{typespec | \misc{*}}}

\label Compound Type Specifier Arguments::

\param{typespec}---a \term{type specifier} that denotes a \subtypeof{real}.

\label Compound Type Specifier Description::

\editornote{KMP: If you ask me, this definition is a complete mess.  Looking at
issue ARRAY-TYPE-ELEMENT-TYPE-SEMANTICS:UNIFY-UPGRADING does not help me figure 
it out, either.  Anyone got any suggestions?}

Every element of this \term{type} is a \term{complex} whose
real part and imaginary part are each of type
\issue{ARRAY-TYPE-ELEMENT-TYPE-SEMANTICS:UNIFY-UPGRADING}
\f{(upgraded-complex-part-type \param{typespec})}.
%The following will be left out:
%\param{type}.
\endissue{ARRAY-TYPE-ELEMENT-TYPE-SEMANTICS:UNIFY-UPGRADING}
This \term{type} encompasses those \term{complexes}
that can result by giving numbers of \term{type} \param{typespec}
to \funref{complex}.
\issue{ARRAY-TYPE-ELEMENT-TYPE-SEMANTICS:UNIFY-UPGRADING}
%The following will be deleted:
%
%This might be different
%from what the \term{type} means for discrimination purposes.
%For example, Gaussian integers might be   
%described as the type {\tt (complex integer)}, even in implementations
%where giving two \term{integers} to \funref{complex} results
%in an \term{object} of type {\tt (complex rational)}.
%
%%%% 2.1.4 3
%%The type of a specific \term{complex} is indicated by a list
%%of the word \misc{complex} and the type of the components; for example,
%%a specialized representation for \typeref{complex} 
%%numbers with \term{short float}
%%parts would be of type {\tt (complex short-float)}.  \Thetype{complex}
%%encompasses all complex representations.
%
%End of deletion.

{\tt (complex \param{type-specifier})}
refers to all \term{complexes} that can result from giving
\term{numbers} of \param{type} \param{type-specifier} to \thefunction{complex}, 
plus all other \term{complexes} of the same specialized representation.      
%% Must fix the following according to Moon's 7-jul mail
%% will fix for next edition.
%Both the real and the imaginary parts of any such 
%\term{complex} must satisfy:
%
%{\tt  (typep \lbracket \param{realpart} $\vert$ \param{imagpart}
%\rbracket\ '\param{type-specifier})}
 
\endissue{ARRAY-TYPE-ELEMENT-TYPE-SEMANTICS:UNIFY-UPGRADING}

\label See Also::

{\secref\RuleOfCanonRepForComplexRationals},
{\secref\NumsFromTokens},
{\secref\PrintingComplexes}

\label Notes::

The input syntax for a \term{complex} with real part $r$ and 
imaginary part $i$ is \f{\#C($r$ $i$)}.  
For further details, \seesection\StandardMacroChars.

For every \term{float}, $n$, there is a \term{complex}
which represents the same mathematical number 
and which can be obtained by {\tt (COERCE $n$ 'COMPLEX)}.

\endcom%{complex}\ftype{System Class}
\begincom{real}\ftype{System Class}

\issue{REAL-NUMBER-TYPE:X3J13-MAR-89}

\label Class Precedence List::
\typeref{real},
\typeref{number},
\typeref{t}

\label Description::

\Thetype{real} includes all \term{numbers} that
represent mathematical real numbers, though there are mathematical real
numbers (\eg irrational numbers) that do not have an exact representation 
in \clisp.  Only \term{reals} can be ordered using the
\funref{<}, \funref{>}, \funref{<=}, and \funref{>=} functions.

\issue{REAL-NUMBER-TYPE:X3J13-MAR-89}
The \term{types} \typeref{rational} and \typeref{float} are \term{disjoint}
\subtypesof{real}.
\endissue{REAL-NUMBER-TYPE:X3J13-MAR-89}

\realtypespec{real}

\endissue{REAL-NUMBER-TYPE:X3J13-MAR-89}

\endcom%{real}\ftype{System Class}
\begincom{float}\ftype{System Class}

\label Class Precedence List::
\typeref{float},
\issue{REAL-NUMBER-TYPE:X3J13-MAR-89}
\typeref{real},
\endissue{REAL-NUMBER-TYPE:X3J13-MAR-89}
\typeref{number},
\typeref{t}

\label Description::

A \term{float}
is a mathematical rational (but \i{not} a \clisp\ \term{rational})
of the form
$s\cdot f\cdot b^{e-p}$,
where $s$ is $+1$ or $-1$, the \i{sign};
$b$ is an \term{integer} 
greater than~1, the \i{base} or \i{radix} of the representation;
$p$ is a positive \term{integer}, 
the \i{precision} (in base-$b$ digits) of the \term{float};
$f$ is a positive \term{integer} 
between $b^{p-1}$ and
$b^p-1$ (inclusive), the significand;
and $e$ is an \term{integer}, the exponent.
The value of $p$ and the range of~$e$
depends on the implementation and on the type of \term{float} 
within that implementation. In addition, there is a floating-point zero;
depending on the implementation, there can also be a ``minus zero''. If there
is no minus zero, then $0.0$ and~$-0.0$ are both interpreted as simply a
floating-point zero.
{\tt (= 0.0 -0.0)} is always true.  
If there is a minus zero, {\tt (eql -0.0 0.0)} is \term{false},
otherwise it is \term{true}.

%!!!
\reviewer{Barmar: What about IEEE NaNs and infinities?}

%!!!
\reviewer{RWK: In the following, what is the ``ordering''?  precision? range?
	  Can there be additional subtypes of float or does ``others'' in the
	  list of four?}

%% 2.15.0 12
The \term{types} \typeref{short-float}, \typeref{single-float}, \typeref{double-float}, 
and \typeref{long-float} are \subtypesof{float}.  Any two of them must be
either \term{disjoint} \term{types} or the \term{same} \term{type};
if the \term{same} \term{type}, then any other \term{types} between them in the
above ordering must also be the \term{same} \term{type}.  For example, 
if \thetype{single-float} and \thetype{long-float} are the \term{same} \term{type},
then \thetype{double-float} must be the \term{same} \term{type} also.

\realtypespec{float}

\label See Also::

{\figref\SyntaxForNumericTokens},
{\secref\NumsFromTokens},
{\secref\PrintingFloats}

\label Notes::

Note that all mathematical integers are representable not only as
\clisp\ \term{reals}, but also as \term{complex floats}.  For example,
possible representations of the mathematical number $1$ 
include the \term{integer} \f{1},
        the \term{float} \f{1.0},
     or the \term{complex} \f{\#C(1.0 0.0)}.  

\endcom%{float}\ftype{System Class}
\begincom{short-float, single-float, double-float, long-float}\ftype{Type}

\label Supertypes:: 

\typeref{short-float}: 
 \typeref{short-float},
 \typeref{float}, 
 \issue{REAL-NUMBER-TYPE:X3J13-MAR-89}
 \typeref{real},
 \endissue{REAL-NUMBER-TYPE:X3J13-MAR-89}
 \typeref{number},
 \typeref{t}

\typeref{single-float}: 
 \typeref{single-float},
 \typeref{float}, 
 \issue{REAL-NUMBER-TYPE:X3J13-MAR-89}
 \typeref{real},
 \endissue{REAL-NUMBER-TYPE:X3J13-MAR-89}
 \typeref{number},
 \typeref{t}

\typeref{double-float}: 
 \typeref{double-float},
 \typeref{float}, 
 \issue{REAL-NUMBER-TYPE:X3J13-MAR-89}
 \typeref{real},
 \endissue{REAL-NUMBER-TYPE:X3J13-MAR-89}
 \typeref{number},
 \typeref{t}

\typeref{long-float}: 
 \typeref{long-float},
 \typeref{float}, 
 \issue{REAL-NUMBER-TYPE:X3J13-MAR-89}
 \typeref{real},
 \endissue{REAL-NUMBER-TYPE:X3J13-MAR-89}
 \typeref{number},
 \typeref{t}

\label Description::

For the four defined \subtypesof{float}, it is true that
%% 2.1.3 6
intermediate between \thetype{short-float} and \thetype{long-float} are
\thetype{single-float} and \thetype{double-float}.
The precise definition of these categories is 
%Changed from CLtL "\term{implementation-dependent}" by KMP per suggestion of Barmar.
%Every implementation has to implement all the float-xxx inspection functions
%anyway, so it's not really possible for an implementation to be wishy-washy
%on this issue.
\term{implementation-defined}.
The precision (measured in ``bits'', computed as $p\log\sub{2}b$)
and the exponent size (also measured in ``bits,'' computed as
$\log\sub{2}(n+1)$, where $n$ is the maximum exponent value) is recommended
to be at least as great
as the values in \thenextfigure. 
Each of the defined \subtypesof{float} might or might not have a minus zero.

\showthree{Recommended Minimum Floating-Point Precision and Exponent Size}{
\hfil\b{Format}&
     \b{Minimum Precision}&
     \b{Minimum Exponent Size}\cr
\noalign{\vskip 2pt\hrule\vskip 2pt}
Short  & 13 bits & 5 bits\cr
Single & 24 bits & 8 bits\cr                             
Double & 50 bits & 8 bits\cr                         
Long   & 50 bits & 8 bits\cr
}

%% 2.1.3 10
%% 2.1.3 11
%% 2.1.3 18
There can be fewer than four internal 
representations for \term{floats}. 
If there are fewer distinct representations, the following rules apply:
\beginlist
\itemitem{--} If there is only one, it is 
%"of" removed -kmp 6-Jan-91
\thetype{single-float}.
In this representation, an \term{object} is simultaneously of \term{types} 
\typeref{single-float}, \typeref{double-float}, \typeref{short-float}, 
and \typeref{long-float}.
\itemitem{--} Two internal representations can be arranged in either of the
following ways:
\beginlist
\itemitem{\bull} Two \term{types} are provided: \typeref{single-float} and
\typeref{short-float}.  An \term{object} is simultaneously 
of \term{types} \typeref{single-float}, \typeref{double-float}, and \typeref{long-float}.
\itemitem{\bull} Two \term{types} are provided: \typeref{single-float} and
\typeref{double-float}. An \term{object} is simultaneously of \term{types}
\typeref{single-float} and \typeref{short-float}, or
\typeref{double-float} and \typeref{long-float}.
\endlist
\itemitem{--} Three internal representations can be arranged in either
of the following ways:
\beginlist
\itemitem{\bull} Three \term{types} are provided: \typeref{short-float}, 
\typeref{single-float}, and \typeref{double-float}.
An \term{object} can simultaneously be of \term{type} \typeref{double-float}
and \typeref{long-float}.
\itemitem{\bull} Three \term{types} are provided: 
\typeref{single-float}, \typeref{double-float},
and \typeref{long-float}. An \term{object} can simultaneously
be of \term{types} \typeref{single-float} and \typeref{short-float}.
\endlist
\endlist          

\label Compound Type Specifier Kind::

Abbreviating.

\label Compound Type Specifier Syntax::

%% 4.6.0 8                         
%% 4.6.0 9

\Deftype{short-float}{\ttbrac{short-lower-limit   \brac{short-upper-limit}}}
\Deftype{single-float}{\ttbrac{single-lower-limit \brac{single-upper-limit}}}
\Deftype{double-float}{\ttbrac{double-lower-limit \brac{double-upper-limit}}}
\Deftype{long-float}{\ttbrac{long-lower-limit     \brac{long-upper-limit}}}

\label Compound Type Specifier Arguments::

\param{short-lower-limit}, \param{short-upper-limit}---\term{interval designators} 
					   for \term{type} \typeref{short-float}.
 \DefaultEach{\param{lower-limit} and \param{upper-limit}}{the \term{symbol} \misc{*}}

\param{single-lower-limit}, \param{single-upper-limit}---\term{interval designators} 
					   for \term{type} \typeref{single-float}.
 \DefaultEach{\param{lower-limit} and \param{upper-limit}}{the \term{symbol} \misc{*}}

\param{double-lower-limit}, \param{double-upper-limit}---\term{interval designators} 
					   for \term{type} \typeref{double-float}.
 \DefaultEach{\param{lower-limit} and \param{upper-limit}}{the \term{symbol} \misc{*}}

\param{long-lower-limit}, \param{long-upper-limit}---\term{interval designators} 
					   for \term{type} \typeref{long-float}.
 \DefaultEach{\param{lower-limit} and \param{upper-limit}}{the \term{symbol} \misc{*}}

\label Compound Type Specifier Description::
                                                   
Each of these denotes the set of \term{floats} of the indicated \term{type} 
that are on the interval specified by the \term{interval designators}.

\endcom%{short-float, single-float, double-float, long-float}\ftype{Type}
\begincom{rational}\ftype{System Class}

\label Class Precedence List::
\typeref{rational},
\issue{REAL-NUMBER-TYPE:X3J13-MAR-89}
\typeref{real},
\endissue{REAL-NUMBER-TYPE:X3J13-MAR-89}
\typeref{number},
\typeref{t}

\label Description::

The canonical representation of a \term{rational} 
is as an \term{integer} if its value is integral,
and otherwise as a \term{ratio}.

%% 2.15.0 8            
The \term{types} \typeref{integer} and \typeref{ratio} 
are \term{disjoint} \subtypesof{rational}.

\realtypespec{rational}

\endcom%{rational}\ftype{System Class}
\begincom{ratio}\ftype{System Class}

\label Class Precedence List::
\typeref{ratio},
\typeref{rational},
\issue{REAL-NUMBER-TYPE:X3J13-MAR-89}
\typeref{real},
\endissue{REAL-NUMBER-TYPE:X3J13-MAR-89}
\typeref{number},
\typeref{t}

\label Description::

A \term{ratio} is a \term{number} 
representing the mathematical ratio of two non-zero integers,
the numerator and denominator,
whose greatest common divisor is one,
and of which the denominator is positive and greater than one.
            
\label See Also::

{\figref\SyntaxForNumericTokens},
{\secref\NumsFromTokens},
{\secref\PrintingRatios}

%% 2.1.1 1

\endcom%{ratio}\ftype{System Class}
\begincom{integer}\ftype{System Class}

\label Class Precedence List::
\typeref{integer},
\typeref{rational},
\issue{REAL-NUMBER-TYPE:X3J13-MAR-89}
\typeref{real},
\endissue{REAL-NUMBER-TYPE:X3J13-MAR-89}
\typeref{number},
\typeref{t}

\label Description::

An \term{integer} is a mathematical integer. There is no limit on the 
magnitude of an \term{integer}.

\issue{FIXNUM-NON-PORTABLE:TIGHTEN-DEFINITION}
%% 2.15.0 10                                
The \term{types} \typeref{fixnum} and \typeref{bignum} 
form an \term{exhaustive partition} of \term{type} \typeref{integer}.
%are \term{disjoint} \subtypesof{integer}.
\endissue{FIXNUM-NON-PORTABLE:TIGHTEN-DEFINITION}

%% 2.1.1 2

\realtypespec{integer}

\label See Also::

{\figref\SyntaxForNumericTokens},
{\secref\NumsFromTokens},
{\secref\PrintingIntegers}

\label Notes::

The \term{type} \f{(integer \i{lower} \i{upper})},
where \i{lower} and \i{upper} 
are \funref{most-negative-fixnum} and \funref{most-positive-fixnum}, respectively,
is also called \typeref{fixnum}.

The \term{type} \f{(integer 0 1)} is also called \typeref{bit}.
The \term{type} \f{(integer 0 *)} is also called \typeref{unsigned-byte}.

\endcom%{integer}\ftype{System Class}
\begincom{signed-byte}\ftype{Type}

%Added per Barrett's suggestion. -kmp 23-Oct-90

\label Supertypes:: 

\typeref{signed-byte},
\typeref{integer},
\typeref{rational},
\issue{REAL-NUMBER-TYPE:X3J13-MAR-89}
\typeref{real},
\endissue{REAL-NUMBER-TYPE:X3J13-MAR-89}
\typeref{number},
\typeref{t}

\label Description::

The atomic \term{type specifier} \typeref{signed-byte} denotes the same 
type as is denoted by the \term{type specifier} \typeref{integer};
however, the list forms of these two \term{type specifiers} have different semantics.

%!!! Barrett thinks this syntax spec needs work.

\label Compound Type Specifier Kind::

Abbreviating.

\label Compound Type Specifier Syntax::

\Deftype{signed-byte}{\ttbrac{s | \misc{*}}}

\label Compound Type Specifier Arguments::

\param{s}---a positive \term{integer}.

\label Compound Type Specifier Description::

%% 4.6.0 5                             
%\itemitem{\f{(signed-byte \param{s})}}

This denotes the set of \term{integers} that can be represented
in two's-complement form in a \term{byte} of \param{s} bits.  This is
equivalent to \f{(integer $-2^{s-1}$ $2^{s-1}-1$)}.  The type
\typeref{signed-byte} or the type \f{(signed-byte *)} is the same
as \thetype{integer}.


\endcom%{signed-byte}\ftype{Type}
\begincom{unsigned-byte}\ftype{Type}

%Added per Barrett's suggestion. -kmp 23-Oct-90

\label Supertypes:: 

\typeref{unsigned-byte},
\typeref{signed-byte},
\typeref{integer},
\typeref{rational},
\issue{REAL-NUMBER-TYPE:X3J13-MAR-89}
\typeref{real},
\endissue{REAL-NUMBER-TYPE:X3J13-MAR-89}
\typeref{number},
\typeref{t}

\label Description::

%!!! Barrett thinks this should be more consistent with presentation of signed-byte.
%!!! Barrett thinks we should be more careful about whitespace in the exponents here.
The atomic \term{type specifier} \typeref{unsigned-byte} denotes the same 
type as is denoted by the \term{type specifier} \f{(integer 0 *)}.  
%% Barmar notes that this is redundant with what's said below.
% The \term{type specifier} {\tt (unsigned-byte \param{s})} is equivalent to 
% {\tt (integer 0 $2^s-1$)}.

\label Compound Type Specifier Kind::

Abbreviating.

\label Compound Type Specifier Syntax::

%% 4.6.0 6
\Deftype{unsigned-byte}{\ttbrac{\param{s} | \misc{*}}}

\label Compound Type Specifier Arguments::

% The "positive" was suggested by Barmar, and seems consistent with signed-byte.
%  -kmp 20-Oct-91
\param{s}---a positive \term{integer}.

\label Compound Type Specifier Description::

This denotes the set of non-negative \term{integers} that can be
represented in a byte of size \param{s} (bits).  
This is equivalent 
to \f{(mod \param{m})} for $\hbox{\param{m}}=2^s$, or
to \f{(integer 0 \param{n})} for $\hbox{\param{n}}=2^s-1$.
\Thetype{unsigned-byte} or 
the type \f{(unsigned-byte *)} is the same as
the type \f{(integer 0 *)}, the set of non-negative \term{integers}.

\label Notes::

The \term{type} \f{(unsigned-byte 1)} is also called \typeref{bit}.

\endcom%{unsigned-byte}\ftype{Type}

%%% ========== MOD
\begincom{mod}\ftype{Type Specifier}

\label Compound Type Specifier Kind::

Abbreviating.

\label Compound Type Specifier Syntax::

%% 4.6.0 4
\Deftype{mod}{n}

\label Compound Type Specifier Arguments::

\param{n}---a positive \term{integer}.

\label Compound Type Specifier Description::
                                                         
This denotes the set of non-negative \term{integers} less than \param{n}.
This is equivalent to 
  \f{(integer 0 (\param{n}))}
or to
  \f{(integer 0 \param{m})},
where $\hbox{\param{m}}=\hbox{\param{n}}-1$.

\issue{TYPE-SPECIFIER-ABBREVIATION:X3J13-JUN90-GUESS}
The argument is required, and cannot be \misc{*}.

The symbol \typeref{mod} is not valid as a \term{type specifier}.
\endissue{TYPE-SPECIFIER-ABBREVIATION:X3J13-JUN90-GUESS}

\endcom%{mod}\ftype{Type Specifier}

%%% ========== BIT
\begincom{bit}\ftype{Type}

%Added per Barrett's suggestion. -kmp 23-Oct-90

\label Supertypes:: 

\typeref{bit},
\typeref{unsigned-byte},
\typeref{signed-byte},
\typeref{integer},
\typeref{rational},
\issue{REAL-NUMBER-TYPE:X3J13-MAR-89}
\typeref{real},
\endissue{REAL-NUMBER-TYPE:X3J13-MAR-89}
\typeref{number},
\typeref{t}

\label Description::

\Thetype{bit} is equivalent to the \term{type} \f{(integer 0 1)}
and \f{(unsigned-byte 1)}.

\endcom%{bit}\ftype{Type}

%%% ========== FIXNUM
\begincom{fixnum}\ftype{Type}

\label Supertypes:: 

\typeref{fixnum},
\typeref{integer},
\typeref{rational},
\issue{REAL-NUMBER-TYPE:X3J13-MAR-89}
\typeref{real},
\endissue{REAL-NUMBER-TYPE:X3J13-MAR-89}
\typeref{number},
\typeref{t}

\label Description::

A \term{fixnum} is an \term{integer} whose value is between 
\conref{most-negative-fixnum} and \conref{most-positive-fixnum} inclusive.
Exactly which \term{integers} are \term{fixnums} is 
% CLtL said "\term{implementation-dependent}" but Barmar suggested (and I agree)
% that implementations should say. -kmp 10-Oct-90
\term{implementation-defined}.
\issue{FIXNUM-NON-PORTABLE:TIGHTEN-DEFINITION}
The \term{type} \typeref{fixnum} is required to be a supertype of
\f{(signed-byte 16)}.
\endissue{FIXNUM-NON-PORTABLE:TIGHTEN-DEFINITION}

%Barrett thinks maybe some intentional information about fixnums
%(e.g., about efficiency) might belong here.  I don't that kind of thing is
%appropriate so I haven't added it. -kmp 23-Oct-90

\endcom%{fixnum}\ftype{Type}

%%% ========== BIGNUM
\begincom{bignum}\ftype{Type}

\label Supertypes:: 

\typeref{bignum},
\typeref{integer},
\typeref{rational},
\issue{REAL-NUMBER-TYPE:X3J13-MAR-89}
\typeref{real},
\endissue{REAL-NUMBER-TYPE:X3J13-MAR-89}
\typeref{number},
\typeref{t}

\label Description::

\issue{FIXNUM-NON-PORTABLE:TIGHTEN-DEFINITION}
\Thetype{bignum} is defined to be exactly \f{(and integer (not fixnum))}.
\endissue{FIXNUM-NON-PORTABLE:TIGHTEN-DEFINITION}

%% 2.2.4 1

\endcom%{bignum}\ftype{Type}

%-------------------- Numeric Comparison --------------------

%%% ========== =
%%% ========== /=
%%% ========== <
%%% ========== >
%%% ========== >=
%%% ========== <=
\begincom{=, /=, <, >, <=, >=}\ftype{Function}

\label Syntax::

\DefunWithValues  = {{\rest} \plus{numbers}} {generalized-boolean}
\DefunWithValues /= {{\rest} \plus{numbers}} {generalized-boolean}
\DefunWithValues  < {{\rest} \plus{numbers}} {generalized-boolean}
\DefunWithValues  > {{\rest} \plus{numbers}} {generalized-boolean}
\DefunWithValues <= {{\rest} \plus{numbers}} {generalized-boolean}
\DefunWithValues >= {{\rest} \plus{numbers}} {generalized-boolean}

%!!! Issue CONTAGION-ON-NUMERICAL-COMPARISONS still needs to be merged? -kmp 9-Aug-91

\label Arguments and Values::

\param{number}---for \funref{<}, \funref{>}, \funref{<=}, \funref{>=}: a \term{real};
		 for \funref{=}, \funref{/=}: a \term{number}.

\param{generalized-boolean}---a \term{generalized boolean}.

\label Description::

\funref{=}, \funref{/=}, \funref{<}, \funref{>}, \funref{<=}, and \funref{>=} 
perform arithmetic comparisons on their arguments as follows:

\beginlist
\itemitem{\funref{=}} 

The value of \funref{=} is \term{true} if all \param{numbers} are the same in value;
otherwise it is \term{false}.
Two \term{complexes} are considered equal by \funref{=}
if their real and imaginary parts are equal according to \funref{=}.
                                         
\itemitem{\funref{/=}} 

The value of \funref{/=} is \term{true} if no two \param{numbers} are the same in value;
otherwise it is \term{false}.

\itemitem{\funref{<}} 

The value of \funref{<} is \term{true} if the \param{numbers} are in monotonically increasing order;
otherwise it is \term{false}.

\itemitem{\funref{>}} 

The value of \funref{>} is \term{true} if the \param{numbers} are in monotonically decreasing order;
otherwise it is \term{false}.

\itemitem{\funref{<=}} 

The value of \funref{<=} is \term{true} if the \param{numbers} are in monotonically 
 nondecreasing order;
otherwise it is \term{false}.

\itemitem{\funref{>=}} 

The value of \funref{>=} is \term{true} if the \param{numbers} are in monotonically
 nonincreasing order;
otherwise it is \term{false}.
\endlist

\funref{=}, \funref{/=}, \funref{<}, \funref{>}, \funref{<=}, and \funref{>=} 
perform necessary type conversions. 

\label Examples::

The uses of these functions are illustrated in \thenextfigure.
%% 12.3.0 4

\showtwo{Uses of /=, =, <, >, <=, and >=}{
\f{(= 3 3)} is \term{true}.              & \f{(/= 3 3)} is \term{false}.            \cr
\f{(= 3 5)} is \term{false}.             & \f{(/= 3 5)} is \term{true}.             \cr
\f{(= 3 3 3 3)} is \term{true}.          & \f{(/= 3 3 3 3)} is \term{false}.        \cr
\f{(= 3 3 5 3)} is \term{false}.         & \f{(/= 3 3 5 3)} is \term{false}.        \cr
\f{(= 3 6 5 2)} is \term{false}.         & \f{(/= 3 6 5 2)} is \term{true}.         \cr
\f{(= 3 2 3)} is \term{false}.           & \f{(/= 3 2 3)} is \term{false}.          \cr
\f{(< 3 5)} is \term{true}.              & \f{(<= 3 5)} is \term{true}.             \cr
\f{(< 3 -5)} is \term{false}.            & \f{(<= 3 -5)} is \term{false}.           \cr
\f{(< 3 3)} is \term{false}.             & \f{(<= 3 3)} is \term{true}.             \cr
\f{(< 0 3 4 6 7)} is \term{true}.        & \f{(<= 0 3 4 6 7)} is \term{true}.       \cr
\f{(< 0 3 4 4 6)} is \term{false}.       & \f{(<= 0 3 4 4 6)} is \term{true}.       \cr
\f{(> 4 3)} is \term{true}.              & \f{(>= 4 3)} is \term{true}.             \cr
\f{(> 4 3 2 1 0)} is \term{true}.        & \f{(>= 4 3 2 1 0)} is \term{true}.       \cr
\f{(> 4 3 3 2 0)} is \term{false}.       & \f{(>= 4 3 3 2 0)} is \term{true}.       \cr
\f{(> 4 3 1 2 0)} is \term{false}.       & \f{(>= 4 3 1 2 0)} is \term{false}.      \cr
\f{(= 3)} is \term{true}.                & \f{(/= 3)} is \term{true}.               \cr
\f{(< 3)} is \term{true}.                & \f{(<= 3)} is \term{true}.               \cr
\f{(= 3.0 \#c(3.0 0.0))} is \term{true}. & \f{(/= 3.0 \#c(3.0 1.0))} is \term{true}.\cr
\f{(= 3 3.0)} is \term{true}.            & \f{(= 3.0s0 3.0d0)} is \term{true}.      \cr
\f{(= 0.0 -0.0)} is \term{true}.         & \f{(= 5/2 2.5)} is \term{true}.          \cr
\f{(> 0.0 -0.0)} is \term{false}.        & \f{(= 0 -0.0)} is \term{true}.           \cr
\f{(<= 0 x 9)}          is \term{true} if \f{x} is between \f{0} and \f{9}, 
			 inclusive\span\cr
\f{(< 0.0 x 1.0)}       is \term{true} if \f{x} is between \f{0.0} and \f{1.0}, 
			 exclusive\span\cr
\f{(< -1 j (length v))} is \term{true} if \f{j} is a \term{valid array index} for
			 a \term{vector} \f{v}\span\cr
}

\label Affected By:\None.                 

\label Exceptional Situations::

Might signal \typeref{type-error} if some \term{argument} is not a \term{real}.
Might signal \typeref{arithmetic-error} if otherwise unable to fulfill its contract.

\label See Also:\None.

\label Notes::

\funref{=} differs from \funref{eql} in that
\f{(= 0.0 -0.0)} is always true,
because \funref{=} compares the mathematical values of its operands,
whereas \funref{eql} compares the representational values, so to speak.

\endcom

%%% ========== MAX
%%% ========== MIN
\begincom{max, min}\ftype{Function}

\label Syntax::

\DefunWithValues max {{\rest} \plus{reals}} {max-real}
\DefunWithValues min {{\rest} \plus{reals}} {min-real}

\label Arguments and Values::

\param{real}---a \term{real}.

\param{max-real}, \param{min-real}---a \term{real}.

\label Description::

%% 12.3.0 8
\funref{max} returns the \param{real} that is greatest (closest to positive infinity).
\funref{min} returns the \param{real} that is least (closest to negative infinity).

For \funref{max},
the implementation has the choice of returning the largest
argument as is or applying the rules of floating-point \term{contagion},
taking all the arguments into consideration for \term{contagion} purposes.
Also, if one or more of the arguments are \funref{=}, then any one
of them may be chosen as the value to return.
For example, if the \param{reals} are a mixture of \term{rationals} and \term{floats},
and the largest argument is a \term{rational},
then the implementation is free to
produce either that \term{rational} 
or its \term{float} approximation;
if the largest argument is a 
\term{float} of a smaller format
than the largest format of any \term{float} argument,
then the implementation is free to
return the argument in its given format or expanded to the larger format.
Similar remarks apply to \funref{min} 
(replacing ``largest argument'' by ``smallest argument'').

\label Examples::

%% 12.3.0 9
\code
 (max 3) \EV 3 
 (min 3) \EV 3
 (max 6 12) \EV 12 
 (min 6 12) \EV 6
 (max -6 -12) \EV -6 
 (min -6 -12) \EV -12
 (max 1 3 2 -7) \EV 3 
 (min 1 3 2 -7) \EV -7
 (max -2 3 0 7) \EV 7 
 (min -2 3 0 7) \EV -2
 (max 5.0 2) \EV 5.0 
 (min 5.0 2)
\EV 2
\OV 2.0
 (max 3.0 7 1)
\EV 7
\OV 7.0 
 (min 3.0 7 1)
\EV 1
\OV 1.0
 (max 1.0s0 7.0d0) \EV 7.0d0
 (min 1.0s0 7.0d0)
\EV 1.0s0
\OV 1.0d0
 (max 3 1 1.0s0 1.0d0)
\EV 3
\OV 3.0d0
 (min 3 1 1.0s0 1.0d0)
\EV 1
\OV 1.0s0 
\OV 1.0d0
\endcode

\label Side Effects:\None.

\label Affected By:\None.

\label Exceptional Situations::

\Shouldcheckanytype{number}{a \term{real}}

\label See Also:\None.

\label Notes:\None.

\endcom                 

%%% ========== MINUSP
%%% ========== PLUSP
\begincom{minusp, plusp}\ftype{Function}

\label Syntax::

\DefunWithValues minusp {real} {generalized-boolean}

\DefunWithValues plusp {real} {generalized-boolean}

\label Arguments and Values::

\param{real}---a \term{real}.

\param{generalized-boolean}---a \term{generalized boolean}.

\label Description::

%% 12.2.0 4
%% 12.2.0 5
\NamedPredicate{minusp}{real}{less than zero}
                  
\NamedPredicate{plusp}{real}{greater than zero}

Regardless of whether an \term{implementation} provides distinct
representations for positive and negative \term{float} zeros,
\f{(minusp -0.0)} always returns \term{false}.

\label Examples::
\code
 (minusp -1) \EV \term{true}
 (plusp 0) \EV \term{false}
 (plusp least-positive-single-float) \EV \term{true}
\endcode

\label Side Effects:\None.

\label Affected By:\None.

\label Exceptional Situations::

\Shouldchecktype{real}{a \term{real}}

\label See Also:\None.

\label Notes:\None.

\endcom

%%% ========== ZEROP
\begincom{zerop}\ftype{Function}

\label Syntax::

\DefunWithValues zerop {number} {generalized-boolean}

\label Pronunciation::

\pronounced{\Stress{z\harde}(\stress{})r\hardo{}(\stress{})p\harde}

\label Arguments and Values::

\param{number}---a \term{number}.

\param{generalized-boolean}---a \term{generalized boolean}.

\label Description::

%% 12.2.0 3
\Predicate{number}{zero (\term{integer}, \term{float}, or \term{complex})}

Regardless of whether an \term{implementation} provides distinct representations
for positive and negative floating-point zeros, \f{(zerop -0.0)} 
always returns \term{true}.

\label Examples::

\code
 (zerop 0) \EV \term{true}
 (zerop 1) \EV \term{false}
 (zerop -0.0) \EV \term{true}
 (zerop 0/100) \EV \term{true}
 (zerop #c(0 0.0)) \EV \term{true}
\endcode

\label Side Effects:\None.

\label Affected By:\None.

\label Exceptional Situations::

\Shouldchecktype{number}{a \term{number}}

\label See Also:\None.

\label Notes::

\code
 (zerop \param{number}) \EQ (= \param{number} 0)
\endcode

\endcom

%-------------------- Rounding --------------------

%%% ========== FLOOR
%%% ========== CEILING
%%% ========== TRUNCATE
%%% ========== ROUND
%%% ========== FFLOOR
%%% ========== FCEILING
%%% ========== FTRUNCATE
%%% ========== FROUND
\begincom{floor, ffloor, ceiling, fceiling,
	  truncate, ftruncate, round, fround}\ftype{Function}

\label Syntax::

\DefunMultiWithValues {number {\opt} divisor} {quotient, remainder}
  {\entry{floor}
   \entry{ffloor}
   \entry{ceiling}
   \entry{fceiling}
   \entry{truncate}
   \entry{ftruncate}
   \entry{round}
   \entry{fround}}

\label Arguments and Values::

\param{number}---a \term{real}.

\param{divisor}---a non-zero \term{real}.
 \Default{the \term{integer} \f{1}}

\param{quotient}---for \funref{floor}, \funref{ceiling}, 
		       \funref{truncate}, and \funref{round}: an \term{integer};
		   for \funref{ffloor}, \funref{fceiling}, 
		       \funref{ftruncate}, and \funref{fround}: a \term{float}.

\param{remainder}---a \term{real}.

\label Description::

%% 12.6.0 16
These functions divide \param{number} by \param{divisor}, 
returning a \param{quotient} and \param{remainder}, such that 

\quad\param{quotient}\centerdot \param{divisor}+\param{remainder}=\param{number}

The \param{quotient} always represents a mathematical integer.
% For \funref{floor}, \funref{ceiling}, \funref{truncate}, and \funref{floor},
% the \param{quotient} is represented as an \term{integer}.
% For \funref{ffloor}, \funref{fceiling}, \funref{ftruncate}, and \funref{ffloor},
% the \param{quotient} is represented as a \term{float}.
When more than one mathematical integer might be possible
 (\ie when the remainder is not zero),
the kind of rounding or truncation depends on the \term{operator}:

\beginlist

\itemitem{\funref{floor}, \funref{ffloor}}

%% 12.6.0 11
\funref{floor} and \funref{ffloor} produce a \param{quotient} 
that has been truncated toward negative infinity;
that is, the \param{quotient} represents the largest mathematical integer
that is not larger than the mathematical quotient.

\itemitem{\funref{ceiling}, \funref{fceiling}}

%% 12.6.0 12
\funref{ceiling} and \funref{fceiling} produce a \param{quotient}
that has been truncated toward positive infinity;
that is, the \param{quotient} represents the smallest mathematical integer
that is not smaller than the mathematical result.

\itemitem{\funref{truncate}, \funref{ftruncate}}

%% 12.6.0 13
\funref{truncate} and \funref{ftruncate} produce a \param{quotient}
that has been truncated towards zero;
that is, the \param{quotient} represents the mathematical integer
of the same sign as the mathematical quotient, and
that has the greatest integral magnitude not greater than that of the mathematical quotient.

\itemitem{\funref{round}, \funref{fround}}

%% 12.6.0 14
\funref{round} and \funref{fround} produce a \param{quotient}
that has been rounded to the nearest mathematical integer;
if the mathematical quotient is exactly halfway between two integers,
(that is, it has the form \i{integer}+$1\over2$),
then the \param{quotient} has been rounded to the even (divisible by two) integer.

\endlist

%!!! Cross reference
All of these functions perform type conversion operations on \param{numbers}.

The \param{remainder}
is an \term{integer}  if both   \f{x} and \f{y} are \term{integers},
is a  \term{rational} if both   \f{x} and \f{y} are \term{rationals}, and
is a  \term{float}    if either \f{x} or  \f{y} is  a \term{float}.

\funref{ffloor}, \funref{fceiling}, \funref{ftruncate}, and \funref{fround} 
handle arguments of different \term{types} in the following way:
If  \param{number}  is     a \term{float},
and \param{divisor} is not a \term{float} of longer format,
then the first result is a \term{float} of the same \term{type} as \param{number}.
Otherwise, the first result is of the \term{type} determined by \term{contagion} rules;
\seesection\NumericContagionRules.

\label Examples::

\code
 (floor 3/2) \EV 1, 1/2
 (ceiling 3 2) \EV 2, -1
 (ffloor 3 2) \EV 1.0, 1
 (ffloor -4.7) \EV -5.0, 0.3
 (ffloor 3.5d0) \EV 3.0d0, 0.5d0
 (fceiling 3/2) \EV 2.0, -1/2
 (truncate 1) \EV 1, 0
 (truncate .5) \EV 0, 0.5
 (round .5) \EV 0, 0.5
 (ftruncate -7 2) \EV -3.0, -1
 (fround -7 2) \EV -4.0, 1
 (dolist (n '(2.6 2.5 2.4 0.7 0.3 -0.3 -0.7 -2.4 -2.5 -2.6))
   (format t "~&~4,1@F ~2,' D ~2,' D ~2,' D ~2,' D"
           n (floor n) (ceiling n) (truncate n) (round n)))
\OUT +2.6  2  3  2  3
\OUT +2.5  2  3  2  2
\OUT +2.4  2  3  2  2
\OUT +0.7  0  1  0  1
\OUT +0.3  0  1  0  0
\OUT -0.3 -1  0  0  0
\OUT -0.7 -1  0  0 -1
\OUT -2.4 -3 -2 -2 -2
\OUT -2.5 -3 -2 -2 -2
\OUT -2.6 -3 -2 -2 -3
\EV NIL
\endcode
%% ^-- 12.6.0 15

\label Side Effects:\None.             

\label Affected By:\None.

\label Exceptional Situations:\None.

\label See Also:\None.

\label Notes::

%% 12.6.0 18
When only \param{number} is given, the two results are exact;
the mathematical sum of the two results is always equal to the
mathematical value of \param{number}.

%% 12.6.0 17
\f{(\i{function} \param{number} \param{divisor})}
and \f{(\i{function} (/ \param{number} \param{divisor}))}
(where \i{function} is any of one
of \funref{floor}, \funref{ceiling}, \funref{ffloor}, 
\funref{fceiling}, \funref{truncate}, 
\funref{round}, \funref{ftruncate}, and \funref{fround})
return the same first value, but
they return different remainders as the second value. For example:

\code
 (floor 5 2) \EV 2, 1
 (floor (/ 5 2)) \EV 2, 1/2
\endcode

If an effect is desired that is similar to \funref{round},
but that always rounds up or down (rather than toward the nearest even integer) 
if the mathematical quotient is exactly halfway between two integers, 
the programmer should consider a construction such as
    \f{(floor (+ x 1/2))}
 or \f{(ceiling (- x 1/2))}.

\endcom


%-------------------- Trig --------------------

%%% ========== SIN
%%% ========== COS
%%% ========== TAN
\begincom{sin, cos, tan}\ftype{Function}

\label Syntax::

\DefunWithValues sin {radians} {number}
\DefunWithValues cos {radians} {number}
\DefunWithValues tan {radians} {number}

\label Arguments and Values::

\param{radians}---a \term{number} given in radians.

\param{number}---a \term{number}.

\label Description::

%% 12.5.2 14
\funref{sin}, \funref{cos}, and \funref{tan} 
return the sine, cosine, and tangent, respectively, of \param{radians}.

\label Examples::

\code
 (sin 0) \EV 0.0
 (cos 0.7853982) \EV 0.707107
 (tan #c(0 1)) \EV #C(0.0 0.761594)
\endcode

\label Side Effects:\None.

\label Affected By:\None.

\label Exceptional Situations::

\Shouldchecktype{radians}{a \term{number}}
Might signal \typeref{arithmetic-error}.

\label See Also::


\funref{asin},
\funref{acos},
\funref{atan},
{\secref\FloatSubstitutability}

\label Notes:\None.

\endcom

%%% ========== ACOS
%%% ========== ASIN
%%% ========== ATAN
\begincom{asin, acos, atan}\ftype{Function}

\label Syntax::

\DefunWithValues asin {number} {radians}
\DefunWithValues acos {number} {radians}
\DefunWithValues atan {number1 {\opt} number2} {radians}

\label Arguments and Values::

\param{number}---a \term{number}.

\param{number1}---a \term{number} if \param{number2} is not supplied,
	       or a \term{real}   if \param{number2} is     supplied.

\param{number2}---a \term{real}.

\param{radians}---a \term{number} (of radians).

\label Description::

%% 12.5.2 17
\funref{asin}, \funref{acos}, and \funref{atan}
compute the arc sine, arc cosine, and arc tangent respectively.

%% 12.5.2 18
The arc sine, arc cosine, and arc tangent (with only \param{number1} supplied)
functions can be defined mathematically for
\param{number} or \param{number1} specified as \i{x} as in \thenextfigure.

\issue{COMPLEX-ATANH-BOGUS-FORMULA:TWEAK-MORE}
\tablefigtwo{Mathematical definition of arc sine, arc cosine, and arc tangent}%
{Function}{Definition}{
Arc sine    & $ -i\ \ff{log} \bigl(ix+     \sqrt{1-x^2}     \bigr) $ \cr
% --sjl 4 Mar 92
%Arc cosine  & $ -i\ \ff{log} \bigl(x+i\    \sqrt{1-x^2}     \bigr) $ \cr
Arc cosine  & $ (\pi/2) - \ff{arcsin} x $ \cr
Arc tangent & $ -i\ \ff{log} \bigl((1+ix)\ \sqrt{1/(1+x^2)} \bigr) $ \cr
}
\endissue{COMPLEX-ATANH-BOGUS-FORMULA:TWEAK-MORE}

%% Barmar points out that these definitions are already given in the definitions
%  of log and sqrt, so there's no need to repeat them here. -kmp 19-Dec-90
% where
% \issue{IEEE-ATAN-BRANCH-CUT:SPLIT}
% \f{(log \i{x}) = (complex (log (abs \i{x})) (phase \i{x}))} and
% \f{(sqrt \i{x}) = (exp (/ (log \i{x}) 2))}.
% \endissue{IEEE-ATAN-BRANCH-CUT:SPLIT}

%% 12.5.2 19
These formulae are mathematically correct, assuming
completely accurate computation. They are not necessarily
the simplest ones for real-valued computations. 

%% 12.5.2 21
%% 12.5.2 23
If both \param{number1} and \param{number2} are supplied
for \funref{atan}, the result is the arc tangent 
of \hbox{\param{number1}/\param{number2}}.
The value of \funref{atan} is always between
$-\pi$ (exclusive) and~$\pi$ (inclusive)
\issue{COMPLEX-ATAN-BRANCH-CUT:TWEAK}                
 when minus zero is not supported.
The range of the two-argument arc tangent when minus zero is supported
includes $-\pi$.
\endissue{COMPLEX-ATAN-BRANCH-CUT:TWEAK}                

For a 
\issue{REAL-NUMBER-TYPE:X3J13-MAR-89}
\term{real}
\endissue{REAL-NUMBER-TYPE:X3J13-MAR-89}
\param{number1}, 
the result is 
\issue{REAL-NUMBER-TYPE:X3J13-MAR-89}
a \term{real}
\endissue{REAL-NUMBER-TYPE:X3J13-MAR-89}
and lies between
$-\pi/2$ and~$\pi/2$ (both exclusive).
\param{number1} can be a \term{complex} if \param{number2}
is not supplied. If both are supplied, \param{number2} can be zero provided
\param{number1} is not zero.  
\reviewer{Barmar: Should add ``However, if the implementation distinguishes 
		  positive and negative zero, both may be signed zeros, 
		  and limits are used to define the result.''}

%% 12.5.3 11
The following definition for arc sine determines the range and
branch cuts:

$$ \ff{arcsin} z = -i\ \ff{log} \Bigl(iz+\sqrt{1-z^2}\Bigr) $$

The branch cut for the arc sine function is in two pieces:
one along the negative real axis to the left of~$-1$
(inclusive), continuous with quadrant II, and one along the positive real
axis to the right of~$1$ (inclusive), continuous with quadrant IV.  The
range is that strip of the complex plane containing numbers whose real
part is between $-\pi/2$ and~$\pi/2$.  A number with real
part equal to $-\pi/2$ is in the range if and only if its imaginary
part is non-negative; a number with real part equal to $\pi/2$ is in
the range if and only if its imaginary part is non-positive.

%% 12.5.3 12
The following definition for arc cosine determines the range and
branch cuts:

% --sjl 4 Mar 92
\issue{COMPLEX-ATANH-BOGUS-FORMULA:TWEAK-MORE}
% $$ \ff{arccos} z = -i\ \ff{log} \Bigl(z+i\ \sqrt{1-z^2}\Bigr) $$
%
%or, which is equivalent,
%
% $$ \ff{arccos} z = {\pi\over2}- \ff{arcsin} z.$$

$$ \ff{arccos} z = {\pi\over2}- \ff{arcsin} z$$

or, which are equivalent,

$$ \ff{arccos} z = -i\ \ff{log} \Bigl(z+i\ \sqrt{1-z^2}\Bigr) $$

$$ \ff{arccos} z = {{2\ \ff{log} \bigl(\sqrt{(1+z)/2} + i\ \sqrt{(1-z)/2}\bigr)}\over{i}}$$
\endissue{COMPLEX-ATANH-BOGUS-FORMULA:TWEAK-MORE}

The branch cut for the arc cosine function is in two pieces:
one along the negative real axis to the left of~$-1$
(inclusive), continuous with quadrant II, and one along the positive real
axis to the right of~$1$ (inclusive), continuous with quadrant IV.  
This is the same branch cut as for arc sine.
The range is that strip of the complex plane containing numbers whose real
part is between 0 and~$\pi$.  A number with real
part equal to 0 is in the range if and only if its imaginary
part is non-negative; a number with real part equal to $\pi$ is in
the range if and only if its imaginary part is non-positive.

%% 12.5.3 13
The following definition for (one-argument) arc tangent determines the
range and branch cuts:

\issue{COMPLEX-ATAN-BRANCH-CUT:TWEAK}                
$$ \ff{arctan} z = {{\ff{log} (1+iz) - \ff{log} (1-iz)}\over{2i}} $$
\endissue{COMPLEX-ATAN-BRANCH-CUT:TWEAK}                

Beware of simplifying this formula; ``obvious'' simplifications are likely
to alter the branch cuts or the values on the branch cuts incorrectly.
The branch cut for the arc tangent function is in two pieces:
one along the positive imaginary axis above $i$
(exclusive), continuous with quadrant II, and one along the negative imaginary
axis below $-i$ (exclusive), continuous with quadrant IV.  
The points $i$ and~$-i$ are excluded from the domain.
The range is that strip of the complex plane containing numbers whose real
part is between $-\pi/2$ and~$\pi/2$.  A number with real
part equal to $-\pi/2$ is in the range if and only if its imaginary
part is strictly positive; a number with real part equal to $\pi/2$ is in
the range if and only if its imaginary part is strictly negative.  Thus the range of
arc tangent is identical to that of arc sine with the points
$-\pi/2$ and~$\pi/2$ excluded.

For \funref{atan},
the signs of \param{number1} (indicated as \i{x})
and \param{number2} (indicated as \i{y}) are used to derive quadrant
information. \Thenextfigure\ details various special cases.
\issue{IEEE-ATAN-BRANCH-CUT:SPLIT}
The asterisk (*) indicates that the entry in the figure applies to
implementations that support minus zero.

\def\Result{\hbox{result}}
\def\starY{\hbox to 1pc{*\hfil}}
\def\starN{\hbox to 1pc{\hfil}}
\tablefigfour{Quadrant information for arc tangent}%
{\starN\i{y} Condition}{\i{x} Condition}{Cartesian locus}{Range of result}{
\starN$ y =  0 $ & $ x >  0 $ & Positive x-axis & $ 0$                       \cr
\starY$ y = +0 $ & $ x >  0 $ & Positive x-axis & $+0$                       \cr
\starY$ y = -0 $ & $ x >  0 $ & Positive x-axis & $-0$                       \cr
\starN$ y >  0 $ & $ x >  0 $ & Quadrant I      & $0 < \Result < \pi/2 $     \cr
\starN$ y >  0 $ & $ x =  0 $ & Positive y-axis & $\pi/2$                    \cr
\starN$ y >  0 $ & $ x <  0 $ & Quadrant II     & $\pi/2 < \Result < \pi$    \cr
\starN$ y =  0 $ & $ x <  0 $ & Negative x-axis & $ \pi$                     \cr
\starY$ y = +0 $ & $ x <  0 $ & Negative x-axis & $+\pi$                     \cr
\starY$ y = -0 $ & $ x <  0 $ & Negative x-axis & $-\pi$                     \cr
\starN$ y <  0 $ & $ x <  0 $ & Quadrant III    & $-\pi < \Result < -\pi/2$  \cr
\starN$ y <  0 $ & $ x =  0 $ & Negative y-axis & $-\pi/2$                   \cr
\starN$ y <  0 $ & $ x >  0 $ & Quadrant IV     & $-\pi/2 < \Result < 0 $    \cr
\starN$ y =  0 $ & $ x =  0 $ & Origin          & undefined consequences     \cr
\starY$ y = +0 $ & $ x = +0 $ & Origin          & $+0$                       \cr
\starY$ y = -0 $ & $ x = +0 $ & Origin          & $-0$                       \cr
\starY$ y = +0 $ & $ x = -0 $ & Origin          & $+\pi$                     \cr
\starY$ y = -0 $ & $ x = -0 $ & Origin          & $-\pi$                     \cr
}
\endissue{IEEE-ATAN-BRANCH-CUT:SPLIT}

\label Examples::

\issue{COMPLEX-ATAN-BRANCH-CUT:TWEAK}
\code
 (asin 0) \EV 0.0 
 (acos #c(0 1))  \EV #C(1.5707963267948966 -0.8813735870195432)
 (/ (atan 1 (sqrt 3)) 6)  \EV 0.087266 
 (atan #c(0 2)) \EV #C(-1.5707964 0.54930615)
\endcode
\endissue{COMPLEX-ATAN-BRANCH-CUT:TWEAK}                

\label Affected By:\None.

\label Exceptional Situations::

\funref{acos} and \funref{asin} \shouldchecktype{number}{a \term{number}}
\funref{atan} should signal \typeref{type-error} if
  one argument is supplied and that argument is not a \term{number},
  or if two arguments are supplied and both of those arguments are not \term{reals}.

\funref{acos}, \funref{asin}, and \funref{atan} might signal \typeref{arithmetic-error}.

\label See Also::

\funref{log},
\funref{sqrt},
{\secref\FloatSubstitutability}

\label Notes::

The result of either \funref{asin} or \funref{acos} can be a \term{complex}
even if \param{number} is not a \term{complex}; this occurs when the 
absolute value of \param{number} is greater than one.

\endcom

%%% ========== PI
\begincom{pi}\ftype{Constant Variable}

\label Value:: 

an \term{implementation-dependent} \term{long float}.

\label Description::

%% 12.5.2 28
The best \term{long float} approximation to the mathematical constant $\pi$.

\label Examples::

\code
 ;; In each of the following computations, the precision depends 
 ;; on the implementation.  Also, if `long float' is treated by 
 ;; the implementation as equivalent to some other float format 
 ;; (e.g., `double float') the exponent marker might be the marker
 ;; for that equivalent (e.g., `D' instead of `L').
 pi \EV 3.141592653589793L0
 (cos pi) \EV -1.0L0

 (defun sin-of-degrees (degrees)
   (let ((x (if (floatp degrees) degrees (float degrees pi))))
     (sin (* x (/ (float pi x) 180)))))
\endcode

\label See Also:\None.

\label Notes::

An approximation to $\pi$ in some other precision can be obtained
by writing \f{(float pi x)}, where \f{x} is a \term{float} of the
desired precision, or by writing \f{(coerce pi \i{type})}, 
where \i{type} is the desired type, such as \typeref{short-float}.

\endcom

%-------------------- Hyperbolic Trig --------------------

%%% ========== SINH
%%% ========== COSH
%%% ========== TANH
%%% ========== ASINH
%%% ========== ACOSH
%%% ========== ATANH
\begincom{sinh, cosh, tanh, asinh, acosh, atanh}\ftype{Function}

\label Syntax::

\DefunWithValues sinh  {number} {result}
\DefunWithValues cosh  {number} {result}
\DefunWithValues tanh  {number} {result}
\DefunWithValues asinh {number} {result}
\DefunWithValues acosh {number} {result}
\DefunWithValues atanh {number} {result}

\label Arguments and Values:: 

\param{number}---a \term{number}.

\param{result}---a \term{number}.

\label Description::

These functions compute the hyperbolic sine, cosine, tangent,
arc sine, arc cosine, and arc tangent functions, 
which are mathematically defined for an argument \i{x}
as given in \thenextfigure.

\issue{COMPLEX-ATANH-BOGUS-FORMULA:TWEAK-MORE}
\tablefigtwo{Mathematical definitions for hyperbolic functions}%
{Function}{Definition}{
Hyperbolic sine        & $ (e^x-e^{-x})/2                       $ \cr
Hyperbolic cosine      & $ (e^x+e^{-x})/2                       $ \cr
Hyperbolic tangent     & $ (e^x-e^{-x})/(e^x+e^{-x})            $ \cr
Hyperbolic arc sine    & $ \ff{log} (x+\sqrt{1+x^2})            $ \cr
% --sjl 4 Mar 92
%Hyperbolic arc cosine  & $ \ff{log} (x+(x+1)\sqrt{(x-1)/(x+1)}) $ \cr
Hyperbolic arc cosine  & $ 2\ \ff{log} (\sqrt{(x+1)/2} + \sqrt{(x-1)/2}) $ \cr
% --sjl 4 Mar 92
%Hyperbolic arc tangent & $ \ff{log} ((1+x)\sqrt{1-1/x^2})       $ \cr
Hyperbolic arc tangent & $ (\ff{log} (1+x) - \ff{log}(1-x))/2   $ \cr
}
\endissue{COMPLEX-ATANH-BOGUS-FORMULA:TWEAK-MORE}


%% Not needed. These are defined in the LOG and SQRT functions. -kmp 22-Jan-92
% where
% \issue{IEEE-ATAN-BRANCH-CUT:SPLIT}
% \f{(log \i{x}) = (complex (log (abs \i{x})) (phase \i{x}))} and
% \f{(sqrt \i{x}) = (exp (/ (log \i{x}) 2))}.
% \endissue{IEEE-ATAN-BRANCH-CUT:SPLIT}

%% 12.5.3 15
The following definition for the inverse hyperbolic cosine
determines the range and branch cuts:

% --sjl 4 mar 92
\issue{COMPLEX-ATANH-BOGUS-FORMULA:TWEAK-MORE}
%$$ \ff{arccosh} z = \ff{log} \Bigl(z+(z+1)\sqrt{{{(z-1)}/{(z+1)}}}\Bigr). $$
$$ \ff{arccosh} z = 2\ \ff{log} \Bigl(\sqrt{(z+1)/2} + \sqrt{(z-1)/2}\Bigr). $$
\endissue{COMPLEX-ATANH-BOGUS-FORMULA:TWEAK-MORE}

%% This is said in the definitions of LOG and SQRT. -kmp 19-Jan-92
% where
% \issue{IEEE-ATAN-BRANCH-CUT:SPLIT}
% \f{(log \i{x}) = (complex (log (abs \i{x})) (phase \i{x}))} and
% \f{(sqrt \i{x}) = (exp (/ (log \i{x}) 2))}.
% \endissue{IEEE-ATAN-BRANCH-CUT:SPLIT}

The branch cut for the inverse hyperbolic cosine function
lies along the real axis to the left of~$1$ (inclusive), extending
indefinitely along the negative real axis, continuous with quadrant II
and (between $0$ and~$1$) with quadrant I.
The range is that half-strip of the complex plane containing numbers whose
real part is non-negative and whose imaginary
part is between $-\pi$ (exclusive) and~$\pi$ (inclusive).
A number with real part zero is in the range 
if its imaginary part is between zero (inclusive) and~$\pi$ (inclusive).

%% 12.5.3 14
The following definition for the inverse hyperbolic sine determines
the range and branch cuts:

%{\tt arcsinh \i{z}=log (\i{z}+$\sqrt{1+\i{z}^2}$)\/}

$$ \ff{arcsinh} z = \ff{log} \Bigl(z+\sqrt{1+z^2}\Bigr). $$

%% This is said in the definitions of LOG and SQRT. -kmp 19-Jan-92
% where
% \issue{IEEE-ATAN-BRANCH-CUT:SPLIT}
% {\tt (log \i{x}) = (complex (log (abs \i{x})) (phase \i{x}))\/} and
% {\tt (sqrt \i{x}) = (exp (/ (log \i{x}) 2))\/}.
% \endissue{IEEE-ATAN-BRANCH-CUT:SPLIT}


The branch cut for the inverse hyperbolic sine function is in two pieces:
one along the positive imaginary axis above $i$
(inclusive), continuous with quadrant I, and one along the negative imaginary
axis below $-i$ (inclusive), continuous with quadrant III.
The range is that strip of the complex plane containing numbers whose imaginary
part is between $-\pi/2$ and~$\pi/2$.  A number with imaginary
part equal to $-\pi/2$ is in the range if and only if its real
part is non-positive; a number with imaginary part equal to $\pi/2$ is in
the range if and only if its imaginary part is non-negative.

%% 12.5.3 16
The following definition for the inverse hyperbolic tangent
determines the range and branch cuts:

% --sjl 4 mar 92
\issue{COMPLEX-ATANH-BOGUS-FORMULA:TWEAK-MORE}
%$$ \ff{arctanh} z = \ff{log} \Bigl((1+z) \sqrt{1-1/z^2}\Bigr). $$ 
$$ \ff{arctanh} z = {{\ff{log} (1+z) - \ff{log} (1-z)}\over{2}}. $$ 

Note that:

$$ i\ \ff{arctan} z = \ff{arctanh} iz. $$
\endissue{COMPLEX-ATANH-BOGUS-FORMULA:TWEAK-MORE}

%% This is said in the definitions of LOG and SQRT. -kmp 19-Jan-92
% where
% \issue{IEEE-ATAN-BRANCH-CUT:SPLIT}
% \f{(log \i{x}) = (complex (log (abs \i{x})) (phase \i{x}))} and
% \f{(sqrt \i{x}) = (exp (/ (log \i{x}) 2))}.
% \endissue{IEEE-ATAN-BRANCH-CUT:SPLIT}

% I believe this first sentence applied to the old, bogus formula.
% -- sjl 4 mar 92
%Beware of simplifying this formula; ``obvious'' simplifications are
%likely to alter the branch cuts or the values on the branch cuts
%incorrectly.  
The branch cut for the inverse hyperbolic tangent function
is in two pieces: one along the negative real axis to the left of
$-1$ (inclusive), continuous with quadrant III, and one along
the positive real axis to the right of~$1$ (inclusive), continuous with
quadrant I.  The points $-1$ and~$1$ are excluded from the
domain.
The range is that strip of the complex plane containing
numbers whose imaginary part is between $-\pi/2$ and
$\pi/2$.  A number with imaginary part equal to $-\pi/2$
is in the range if and only if its real part is strictly negative; a number with
imaginary part equal to $\pi/2$ is in the range if and only if its imaginary
part is strictly positive.  
Thus the range of the inverse hyperbolic tangent function is identical to
that of the inverse hyperbolic sine function with the points
$-\pi i/2$ and~$\pi i/2$ excluded.

\label Examples::

\code
 (sinh 0) \EV 0.0 
 (cosh (complex 0 -1)) \EV #C(0.540302 -0.0)
\endcode

\label Side Effects:\None.

\label Affected By:\None.

\label Exceptional Situations::

\Shouldchecktype{number}{a \term{number}}
Might signal \typeref{arithmetic-error}.

\label See Also::

\funref{log},
\funref{sqrt},
{\secref\FloatSubstitutability}

\label Notes::

The result of \funref{acosh} may be a \term{complex} even if \param{number} 
is not a \term{complex}; this occurs when \param{number} is less than one.
Also, the result of \funref{atanh} may be a \term{complex} even if \param{number} 
is not a \term{complex}; this occurs when the absolute value of \param{number} 
is greater than one.

%% 12.5.2 29
The branch cut formulae are mathematically correct, assuming
completely accurate computation.  
Implementors should consult a good text on
numerical analysis.  The formulae given above are not necessarily
the simplest ones for real-valued computations; they are chosen
to define the branch cuts in desirable ways for the complex case.

\endcom

%-------------------- Arithmetic --------------------

%%% ========== * (Function)
\begincom{*}\ftype{Function}

%% 12.4.0 8

\label Syntax::

\DefunWithValues * {{\rest} numbers} {product}

\label Arguments and Values::

\param{number}---a \term{number}.

\param{product}---a \term{number}.

\label Description::

Returns the product of \param{numbers},
performing any necessary type conversions in the process.
If no \param{numbers} are supplied, \f{1} is returned.

\label Examples::

\code
 (*) \EV 1
 (* 3 5) \EV 15
 (* 1.0 #c(22 33) 55/98) \EV #C(12.346938775510203 18.520408163265305)
\endcode

\label Affected By:\None.

\label Exceptional Situations::

Might signal \typeref{type-error} if some \term{argument} is not a \term{number}.
Might signal \typeref{arithmetic-error}.

\label See Also::

{\secref\NumericOperations},
{\secref\RationalComputations},
{\secref\FloatingPointComputations},
{\secref\ComplexComputations}

\label Notes:\None.

\endcom

%%% ========== + (Function)
\begincom{+}\ftype{Function}

\label Syntax::

\DefunWithValues + {{\rest} numbers} {sum}

\label Arguments and Values::

\param{number}---a \term{number}.

\param{sum}---a \term{number}.

\label Description::

%% 12.4.0 3
Returns the sum of \param{numbers},
performing any necessary type conversions in the process.
If no \param{numbers} are supplied, \f{0} is returned.

\label Examples::
\code
 (+) \EV 0
 (+ 1) \EV 1
 (+ 31/100 69/100) \EV 1
 (+ 1/5 0.8) \EV 1.0
\endcode

\label Affected By:\None.

\label Exceptional Situations::

Might signal \typeref{type-error} if some \term{argument} is not a \term{number}.
Might signal \typeref{arithmetic-error}.

\label See Also::

{\secref\NumericOperations},
{\secref\RationalComputations},
{\secref\FloatingPointComputations},
{\secref\ComplexComputations}

\label Notes:\None.

\endcom

%%% ========== - (Function)
\begincom{$-$}\ftype{Function}

\label Syntax::

\DefunWithValues {$-$} {number} {negation}

\DefunWithValues {$-$} {minuend {\rest} \plus{subtrahends}} {difference}

\label Arguments and Values::

\param{number}, \param{minuend}, \param{subtrahend}---a \term{number}.

\param{negation}, \param{difference}---a \term{number}.

\label Description::

\Thefunction{-} performs arithmetic subtraction and negation.

%% 12.4.0 5
If only one \param{number} is supplied,
the negation of that \param{number} is returned.

%% 12.4.0 6
If more than one \term{argument} is given, 
it subtracts all of the \param{subtrahends} from the \param{minuend}
and returns the result.

\Thefunction{-} performs necessary type conversions. 

\label Examples::

\code
 (- 55.55) \EV -55.55
 (- #c(3 -5)) \EV #C(-3 5)
 (- 0) \EV 0
 (eql (- 0.0) -0.0) \EV \term{true}
 (- #c(100 45) #c(0 45)) \EV 100
 (- 10 1 2 3 4) \EV 0
\endcode

\label Affected By:\None.

\label Exceptional Situations::

Might signal \typeref{type-error} if some \term{argument} is not a \term{number}.
Might signal \typeref{arithmetic-error}.

\label See Also::

{\secref\NumericOperations},
{\secref\RationalComputations},
{\secref\FloatingPointComputations},
{\secref\ComplexComputations}

\label Notes:\None.

\endcom

%%% ========== / (Function)
\begincom{/}\ftype{Function}

\label Syntax::

\DefunWithValues / {number} {reciprocal}
\DefunWithValues / {numerator {\rest} \plus{denominators}} {quotient}

\label Arguments and Values::

\param{number}, \param{denominator}---a non-zero \term{number}.

\param{numerator}, \param{quotient}, \param{reciprocal}---a \term{number}.

\label Description::

%% 12.4.0 10
\Thefunction{/} performs division or reciprocation.

%% 12.4.0 11
If no \param{denominators} are supplied,
\thefunction{/} returns the reciprocal of \param{number}.

If at least one \param{denominator} is supplied,
\thefunction{/} divides the \param{numerator} by all of the \param{denominators} 
and returns the resulting \param{quotient}.

%% 12.4.0 12
If each \term{argument} is either an \term{integer} or a \term{ratio},
and the result is not an \term{integer}, then it is a \term{ratio}.

\Thefunction{/} performs necessary type conversions.

%% 12.4.0 14

If any \param{argument} is a \term{float} then 
the rules of floating-point contagion apply;
\seesection\FloatingPointComputations.

\label Examples::

\code
 (/ 12 4) \EV 3
 (/ 13 4) \EV 13/4
 (/ -8) \EV -1/8
 (/ 3 4 5) \EV 3/20
 (/ 0.5) \EV 2.0
 (/ 20 5) \EV 4
 (/ 5 20) \EV 1/4
 (/ 60 -2 3 5.0) \EV -2.0
 (/ 2 #c(2 2)) \EV #C(1/2 -1/2)
\endcode

\label Affected By:\None.

\label Exceptional Situations::

The consequences are unspecified if any \term{argument} other than the first is zero.
If there is only one \term{argument}, the consequences are unspecified if it is zero.

Might signal \typeref{type-error} if some \term{argument} is not a \term{number}.
Might signal \typeref{division-by-zero} if division by zero is attempted.
Might signal \typeref{arithmetic-error}.

\label See Also::

\funref{floor}, \funref{ceiling}, \funref{truncate}, \funref{round}

\label Notes:\None.

\endcom

%%% ========== 1+
%%% ========== 1-
\begincom{1+, 1$-$}\ftype{Function}

\label Syntax::

\DefunWithValues 1+   {number} {successor}
\DefunWithValues 1$-$ {number} {predecessor}

\label Arguments and Values:: 

\param{number}---a \term{number}.

\param{successor}, \param{predecessor}---a \term{number}.

\label Description::

%%12.4.0 17
\funref{1+} returns a \term{number} that is one more than its argument \param{number}.
\funref{1-} returns a \term{number} that is one less than its argument \param{number}.

\label Examples::

\code
 (1+ 99) \EV 100 
 (1- 100) \EV 99 
 (1+ (complex 0.0)) \EV #C(1.0 0.0) 
 (1- 5/3) \EV 2/3 
\endcode

\label Affected By:\None.

\label Exceptional Situations::

Might signal \typeref{type-error} if its \term{argument} is not a \term{number}.
Might signal \typeref{arithmetic-error}.

\label See Also::

\macref{incf}, \macref{decf}

\label Notes::

\code
 (1+ \param{number}) \EQ (+ \param{number} 1)
 (1- \param{number}) \EQ (- \param{number} 1)
\endcode
Implementors are encouraged to make the performance of both the previous
expressions be the same.

\endcom

%%% ========== ABS
\begincom{abs}\ftype{Function}

\label Syntax::

\DefunWithValues abs {number} {absolute-value}

\label Arguments and Values::

\param{number}---a \term{number}.

\param{absolute-value}---a non-negative \term{real}.

\label Description::

%% 12.5.2 3
\funref{abs} returns the absolute value of \param{number}.
%% 12.5.2 4

If \param{number} is 
\issue{REAL-NUMBER-TYPE:X3J13-MAR-89}
a \term{real},
\endissue{REAL-NUMBER-TYPE:X3J13-MAR-89}
the result is  of the same \term{type} as \param{number}.

If \param{number} is a \term{complex}, 
the result is a positive 
\issue{REAL-NUMBER-TYPE:X3J13-MAR-89}
\term{real}
\endissue{REAL-NUMBER-TYPE:X3J13-MAR-89}
with
the same magnitude as \param{number}.  
The result can be a \term{float} 
\reviewer{Barmar: Single-float.}%!!! What to do here? -kmp 17-Dec-90
even if \param{number}'s components are \term{rationals} 
and an exact rational result
would have been possible.
%% 12.5.2 7
Thus the result of \f{(abs \#c(3 4))} can be either \f{5} or \f{5.0},
depending on the implementation.


\label Examples::

\issue{COMPLEX-RATIONAL-RESULT:EXTEND}
\code
 (abs 0) \EV 0
 (abs 12/13) \EV 12/13
 (abs -1.09) \EV 1.09
 (abs #c(5.0 -5.0)) \EV 7.071068
 (abs #c(5 5)) \EV 7.071068
 (abs #c(3/5 4/5)) \EV 1 or approximately 1.0
 (eql (abs -0.0) -0.0) \EV \term{true}
\endcode
\endissue{COMPLEX-RATIONAL-RESULT:EXTEND}

\label Affected By:\None.

\label Exceptional Situations:\None.

\label See Also::

{\secref\FloatSubstitutability}

\label Notes::

%% 12.5.2 5
If \param{number} is a \term{complex}, 
the result is equivalent to the following: 
%\hfil\break

\f{(sqrt (+ (expt (realpart \param{number}) 2)  (expt (imagpart \param{number}) 2)))}
%%Barmar says ``SQRT is a defined name, so there's no need to redefine it here.''
% -kmp 17-Dec-90
% , where 
% \issue{IEEE-ATAN-BRANCH-CUT:SPLIT}
% \f{(sqrt \i{x}) = (exp (/ (log \i{x}) 2))}
% \endissue{IEEE-ATAN-BRANCH-CUT:SPLIT}

%% 12.5.2 6
An implementation should not use this formula directly
for all \term{complexes}
but should handle very large or very small components specially
to avoid intermediate overflow or underflow.

\endcom

%%% ========== EVENP
%%% ========== ODDP
\begincom{evenp, oddp}\ftype{Function}

\label Syntax::

\DefunWithValues evenp {integer} {generalized-boolean}
\DefunWithValues oddp  {integer} {generalized-boolean}

\label Arguments and Values::

\param{integer}---an \term{integer}.

\param{generalized-boolean}---a \term{generalized boolean}.

\label Description::

%% 12.2.0 7
\NamedPredicate{evenp}{integer}{even (divisible by two)}

%% 12.2.0 6
\NamedPredicate{oddp}{integer}{odd (not divisible by two)}

\label Examples::

\code
 (evenp 0) \EV \term{true}
 (oddp 10000000000000000000000) \EV \term{false}
 (oddp -1) \EV \term{true}
\endcode

\label Side Effects:\None.

\label Affected By:\None.

\label Exceptional Situations::

\Shouldchecktype{integer}{an \term{integer}}
                       
\label See Also:\None.

\label Notes::

\code
 (evenp \param{integer}) \EQ (not (oddp \param{integer}))
 (oddp \param{integer})  \EQ (not (evenp \param{integer}))
\endcode

\endcom

%%% ========== EXP
%%% ========== EXPT
\begincom{exp, expt}\ftype{Function}

\label Syntax::

\DefunWithValues exp  {number} {result}
\DefunWithValues expt {base-number power-number} {result}


\label Arguments and Values::

\param{number}---a \term{number}.

\param{base-number}---a \term{number}.

\param{power-number}---a \term{number}.

\param{result}---a \term{number}.

\label Description::

\funref{exp} and \funref{expt} perform exponentiation.

%% 12.5.1 2
\funref{exp} returns \i{e} raised to the power \param{number},
where \i{e} is the base of the natural logarithms.
%% 12.5.3 9
\funref{exp} has no branch cut.

%% 12.5.0 5
%% 12.5.1 3
\funref{expt} returns \param{base-number} 
raised to the power \param{power-number}.
If the \param{base-number} is a \term{rational} 
and \param{power-number} is
an \term{integer},
the calculation is exact and the result will be \oftype{rational};
otherwise a floating-point approximation might result.
\issue{COMPLEX-RATIONAL-RESULT:EXTEND}
For \funref{expt} of a \term{complex rational} to an \term{integer} power,
the calculation must be exact and the result is
of type \f{(or rational (complex rational))}.
\endissue{COMPLEX-RATIONAL-RESULT:EXTEND}
 
%% 12.5.1 6
The result of \funref{expt} can be a \term{complex},
even when neither argument is a \term{complex},
if \param{base-number} is negative and \param{power-number}
is not an \term{integer}. 
The result is always the \term{principal} \term{complex} \term{value}.
For example, \f{(expt -8 1/3)} is not permitted to return \f{-2},
even though \f{-2} is one of the cube roots of \f{-8}. 
The \term{principal} cube root is a \term{complex}
approximately equal to \f{\#C(1.0 1.73205)}, not \f{-2}.

%% 12.5.3 10
\funref{expt} is defined
as \i{$b^x$ = $e^{x log b\/}$}.
%%Barmar thinks this should just cross-reference the LOG function. (I agree.) -kmp 22-Jan-92
%% This is defined in LOG.
% , where
% \issue{IEEE-ATAN-BRANCH-CUT:SPLIT}
% \f{(log \i{b}) = (complex (log (abs \i{b})) (phase \i{b}))}.
% \endissue{IEEE-ATAN-BRANCH-CUT:SPLIT}
This defines the \term{principal} \term{values} precisely.  The range of
\funref{expt} is the entire complex plane.  Regarded
as a function of \i{x}, with \i{b} fixed, there is no branch cut.
Regarded as a function of \i{b}, with \i{x} fixed, there is in general
a branch cut along the negative real axis, continuous with quadrant II.
The domain excludes the origin.
By definition, $0^0$=1.  If \i{b}=0 and the real part of \i{x} is strictly
positive, then 
\i{$b^x$}=0. For all other values of \i{x}, $0^\i{x}$ 
is an error.

%% 12.5.1 4
When \param{power-number} is an \term{integer} \f{0},
then the result is always the value one in the \term{type} 
of \param{base-number},
even if the \param{base-number} is zero (of any \term{type}).  That is:

\code
 (expt x 0) \EQ (coerce 1 (type-of x))
\endcode
If \param{power-number} is a zero of any other \term{type},
then the result is also the value one, in the \term{type} of the arguments
after the application of the contagion rules in \secref\NumericContagionRules,
with one exception:
the consequences are undefined if \param{base-number} is zero when \param{power-number}
is zero and not \oftype{integer}.

\label Examples::

\issue{COMPLEX-RATIONAL-RESULT:EXTEND}
\code
 (exp 0) \EV 1.0
 (exp 1) \EV 2.718282
 (exp (log 5)) \EV 5.0 
 (expt 2 8) \EV 256
 (expt 4 .5) \EV 2.0
 (expt #c(0 1) 2) \EV -1
 (expt #c(2 2) 3) \EV #C(-16 16)
 (expt #c(2 2) 4) \EV -64 
\endcode
\endissue{COMPLEX-RATIONAL-RESULT:EXTEND}

\label Affected By:\None.

\label Exceptional Situations:\None.

\label See Also::

\funref{log},
{\secref\FloatSubstitutability}

\label Notes::


%% 12.5.1 5
Implementations of \funref{expt} are permitted to use different algorithms
for the cases of a \param{power-number} \oftype{rational} 
	     and a \param{power-number} \oftype{float}.

\issue{EXPT-RATIO:P.211}
 
  Note that by the following logic, 
\f{(sqrt (expt \i{x} 3))} is not equivalent to 
\f{(expt \i{x} 3/2)}.

%!!! Figure out some way to notate these `precise approximations'
\code
 (setq x (exp (/ (* 2 pi #c(0 1)) 3)))         ;exp(2.pi.i/3)
 (expt x 3) \EV 1 ;except for round-off error
 (sqrt (expt x 3)) \EV 1 ;except for round-off error
 (expt x 3/2) \EV -1 ;except for round-off error
\endcode
\endissue{EXPT-RATIO:P.211}

\endcom

%%% ========== GCD
\begincom{gcd}\ftype{Function}

\label Syntax::

\DefunWithValues gcd {{\rest} integers} {greatest-common-denominator}

\label Arguments and Values::

\param{integer}---an \term{integer}.

\param{greatest-common-denominator}---a non-negative \term{integer}.

\label Description::

%% 12.4.0 22
Returns the greatest common divisor of \param{integers}.
If only one \param{integer} is supplied, its absolute value is returned.
If no \param{integers} are given, \funref{gcd} returns \f{0},
which is an identity for this operation.

\label Examples::

%% 12.4.0 23
\code
 (gcd) \EV 0
 (gcd 60 42) \EV 6
 (gcd 3333 -33 101) \EV 1
 (gcd 3333 -33 1002001) \EV 11
 (gcd 91 -49) \EV 7
 (gcd 63 -42 35) \EV 7
 (gcd 5) \EV 5
 (gcd -4) \EV 4
\endcode


\label Side Effects:\None.

\label Affected By:\None.

\label Exceptional Situations::

\Shouldcheckanytype{integer}{an \term{integer}}

\label See Also::

\funref{lcm}

\label Notes::
For three or more arguments,

\code
 (gcd b c ... z) \EQ (gcd (gcd a b) c ... z)
\endcode
\endcom

%%% ========== INCF
%%% ========== DECF
\begincom{incf, decf}\ftype{Macro}                         

\label Syntax::

\DefmacWithValues incf {place \brac{delta-form}} {new-value}
\DefmacWithValues decf {place \brac{delta-form}} {new-value}

\label Arguments and Values:: 

\param{place}---a \term{place}.

\param{delta-form}---a \term{form}; evaluated to produce a \param{delta}.
 \Default{\f{1}}

\param{delta}---a \term{number}.

\param{new-value}---a \term{number}.

\label Description::

\macref{incf} and \macref{decf} are used for incrementing and
decrementing the \term{value} of \param{place}, respectively.

The \param{delta} is 
    added to        (in the case of \macref{incf})
 or subtracted from (in the case of \macref{decf})
the number in \param{place} and the result is stored in \param{place}.

Any necessary type conversions are performed automatically.

\issue{PUSH-EVALUATION-ORDER:FIRST-ITEM}
For information about the \term{evaluation} of \term{subforms} of \param{places},
\seesection\GenRefSubFormEval.  
\endissue{PUSH-EVALUATION-ORDER:FIRST-ITEM}

\label Examples::
\code
 (setq n 0)
 (incf n) \EV 1      
 n \EV 1
 (decf n 3) \EV -2   
 n \EV -2
 (decf n -5) \EV 3      
 (decf n) \EV 2      
 (incf n 0.5) \EV 2.5
 (decf n) \EV 1.5
 n \EV 1.5
\endcode

\label Side Effects::

\param{Place} is modified.

\label Affected By:\None.

\label Exceptional Situations:\None.

\label See Also::

\funref{+}, \funref{-}, \funref{1+}, \funref{1-}, \macref{setf}

\label Notes:\None.

%% Removed in response to Kawabe's comment #, first public review.
%
% The effect of \f{(incf \param{place} \param{delta})} is roughly equivalent to
% 
% \code
%  (setf \param{place} (+ \param{place} \param{delta}))
% \endcode
% except that the latter would evaluate any \term{subforms} of \param{place}
% twice, whereas \macref{incf} takes care to evaluate them only once.
       
\endcom

%%% ========== LCM
\begincom{lcm}\ftype{Function}

\label Syntax::

\DefunWithValues lcm {{\rest} integers} {least-common-multiple}

\label Arguments and Values::

\param{integer}---an \term{integer}.

\param{least-common-multiple}---a non-negative \term{integer}.

\label Description::

%% 12.4.0 24
\funref{lcm} returns the least common multiple of the \param{integers}.

\issue{LCM-NO-ARGUMENTS:1}
%% 12.4.0 27
%\funref{lcm} requires at least one argument.
If no \param{integer} is supplied, the \term{integer} \f{1} is returned.
\endissue{LCM-NO-ARGUMENTS:1}

If only one \param{integer} is supplied, 
the absolute value of that \param{integer} is returned.

For two arguments that are not both zero,

\code
 (lcm a b) \EQ (/ (abs (* a b)) (gcd a b))
\endcode

If one or both arguments are zero,

\code
 (lcm a 0) \EQ (lcm 0 a) \EQ 0
\endcode

%% 12.4.0 25
For three or more arguments,

\code
 (lcm a b c ... z) \EQ (lcm (lcm a b) c ... z)
\endcode

\label Examples::
%% 12.4.0 26
\code
 (lcm 10) \EV 10
 (lcm 25 30) \EV 150
 (lcm -24 18 10) \EV 360
 (lcm 14 35) \EV 70
 (lcm 0 5) \EV 0
 (lcm 1 2 3 4 5 6) \EV 60
\endcode

\label Side Effects:\None.

\label Affected By:\None.

\label Exceptional Situations::

Should signal \typeref{type-error} if any argument is not an \term{integer}.

\label See Also::

\funref{gcd}

\label Notes:\None.

\endcom

%%% ========== LOG
\begincom{log}\ftype{Function}

\label Syntax::

\DefunWithValues log {number {\opt} base} {logarithm}

\label Arguments and Values::

\param{number}---a non-zero \term{number}.

\param{base}---a \term{number}.

\param{logarithm}---a \term{number}.

\label Description::

%% 12.5.1 7
\funref{log} returns the logarithm of \param{number} in base \param{base}.
If \param{base} is not supplied its value is \i{e}, 
the base of the natural logarithms.

%% 12.5.1 8
\funref{log} may return a \term{complex} when given a 
\issue{REAL-NUMBER-TYPE:X3J13-MAR-89}
\term{real}
\endissue{REAL-NUMBER-TYPE:X3J13-MAR-89}
negative \param{number}.

\code
 (log -1.0) \EQ (complex 0.0 (float pi 0.0))
\endcode

If \param{base} is zero,
%%Removed as unnecessary per Barmar since \param{number} is described above as non-zero.
%and \param{number} is non-zero
\funref{log} returns zero.

The result of \f{(log 8 2)} may be either \f{3} or \f{3.0}, depending on the
implementation. An implementation can use floating-point calculations
even if an exact integer result is possible.

%% 12.5.3 7
The branch cut for the logarithm function of one argument (natural
logarithm) lies along the negative real axis, continuous with quadrant II.
The domain excludes the origin.

%%Per Barmar, this is removed since the above note from IEEE-ATAN-BRANCH-CUT
%%already says the same thing.
% When \param{number} is a \term{complex},
% 
% %{\tt log \param{number}
% %  \EQ\ (log $\vert$\param{number}$\vert$)+\i{i} \i{phase}(\param{number})\/}.
% \code
%  (log \param{complex})
%  \EQ (+ (log (abs \param{complex})) (* (phase \param{complex}) #c(0 1)))
% \endcode

\issue{IEEE-ATAN-BRANCH-CUT:SPLIT}
The mathematical definition of a complex logarithm 
is as follows, whether or not minus zero is supported by the
implementation:

\code
(log \i{x}) \EQ (complex (log (abs \i{x})) (phase \i{x}))
\endcode
\endissue{IEEE-ATAN-BRANCH-CUT:SPLIT}

Therefore the range of the one-argument logarithm function is that strip
of the complex plane containing numbers with imaginary parts between
\issue{IEEE-ATAN-BRANCH-CUT:SPLIT}
$-\pi$ (exclusive) and~$\pi$ (inclusive) if minus zero is not supported,
or $-\pi$ (inclusive) and~$\pi$ (inclusive) if minus zero is supported.
\endissue{IEEE-ATAN-BRANCH-CUT:SPLIT}

%% 12.5.3 8
The two-argument logarithm function is defined as 

\code
 (log \param{base} \param{number})
 \EQ (/ (log \param{number}) (log \param{base}))
\endcode

This defines the \term{principal} \term{values} precisely.  
The range of the two-argument logarithm function is the entire complex plane.

\label Examples::

\code
 (log 100 10)
\EV 2.0
\EV 2
 (log 100.0 10) \EV 2.0
 (log #c(0 1) #c(0 -1))
\EV #C(-1.0 0.0)
\OV #C(-1 0)
 (log 8.0 2) \EV 3.0
\endcode

\issue{COMPLEX-RATIONAL-RESULT:EXTEND}
%!!! Figure out some way to notate these `precise approximations'
\code
 (log #c(-16 16) #c(2 2)) \EV 3 or approximately #c(3.0 0.0)
                               or approximately 3.0 (unlikely)
\endcode
\endissue{COMPLEX-RATIONAL-RESULT:EXTEND}

\label Affected By::

The implementation.

\label Exceptional Situations:\None.

\label See Also::

\funref{exp},
\funref{expt},
{\secref\FloatSubstitutability}

\label Notes:\None.

\endcom

%%% ========== MOD
%%% ========== REM
\begincom{mod, rem}\ftype{Function}

\label Syntax::

\DefunWithValues mod {number divisor} {modulus}
\DefunWithValues rem {number divisor} {remainder}

\label Arguments and Values::

\param{number}---a \term{real}.

\param{divisor}---a \term{real}.

\param{modulus}, \param{remainder}---a \term{real}.

\label Description::

%% 12.6.0 22
\funref{mod} and \funref{rem} are generalizations of the modulus
and remainder functions respectively.

\funref{mod} performs the operation \funref{floor} 
on \param{number} and \param{divisor} 
and returns the remainder of the \funref{floor} operation.

\funref{rem} performs the operation \funref{truncate} 
on \param{number} and \param{divisor}
and returns the remainder of the \funref{truncate} operation.

\funref{mod} and \funref{rem} are 
%!!! Is this something from mathematics?? -kmp 24-Apr-91
the modulus and remainder functions 
when \param{number} and \param{divisor} are \term{integers}.

\label Examples::
\code
 (rem -1 5) \EV -1
 (mod -1 5) \EV 4
 (mod 13 4) \EV 1
 (rem 13 4) \EV 1
 (mod -13 4) \EV 3
 (rem -13 4) \EV -1
 (mod 13 -4) \EV -3
 (rem 13 -4) \EV 1
 (mod -13 -4) \EV -1
 (rem -13 -4) \EV -1
 (mod 13.4 1) \EV 0.4
 (rem 13.4 1) \EV 0.4
 (mod -13.4 1) \EV 0.6
 (rem -13.4 1) \EV -0.4
\endcode

\label Side Effects:\None.

\label Affected By:\None.

\label Exceptional Situations:\None.

\label See Also::

\funref{floor}, \funref{truncate}

\label Notes::

The result of \funref{mod} is either zero or a 
\issue{REAL-NUMBER-TYPE:X3J13-MAR-89}
\term{real}
\endissue{REAL-NUMBER-TYPE:X3J13-MAR-89}
with the same sign as \param{divisor}.
\endcom

%%% ========== SIGNUM
\begincom{signum}\ftype{Function}

\label Syntax::

\DefunWithValues signum {number} {signed-prototype}

\label Arguments and Values::

\param{number}---a \term{number}.

\param{signed-prototype}---a \term{number}.

\label Description::

\funref{signum} determines a numerical value that indicates whether
\param{number} is negative, zero, or positive.

%% 12.5.2 12
For a \term{rational},
\funref{signum} returns one of \f{-1}, \f{0}, or \f{1}
according to whether \param{number} is negative, zero, or positive.
For a \term{float}, 
the result is a \term{float} of the same format
whose value is minus one, zero, or one.
For a \term{complex} number \f{z},
\f{(signum \i{z})} is a complex number of the same phase but with unit magnitude,
unless \f{z} is a complex zero, in which case the result is \f{z}.

%% 12.5.2 13
%%Rewritten to avoid a suggestion that there might be complex rationals. -kmp 2-Jan-91
% For non-\term{complex} \term{rationals}, \funref{signum} 
% is a rational function, but it may be irrational for \term{complex} arguments.
For \term{rational} \term{arguments}, \funref{signum} is a rational function,
but it may be irrational for \term{complex} \term{arguments}.

If \param{number} is a \term{float}, the result is a \term{float}.
If \param{number} is a \term{rational}, the result is a \term{rational}.
If \param{number} is a \term{complex float}, the result is a \term{complex float}.
If \param{number} is a \term{complex rational}, the result is a \term{complex}, 
but it is \term{implementation-dependent} whether that result is a
\term{complex rational} or a \term{complex float}.

\label Examples::

\code
 (signum 0) \EV 0
 (signum 99) \EV 1
 (signum 4/5) \EV 1
 (signum -99/100) \EV -1
 (signum 0.0) \EV 0.0
 (signum #c(0 33)) \EV #C(0.0 1.0)
 (signum #c(7.5 10.0)) \EV #C(0.6 0.8)
 (signum #c(0.0 -14.7)) \EV #C(0.0 -1.0)
 (eql (signum -0.0) -0.0) \EV \term{true}
\endcode

\label Side Effects:\None.

\label Affected By:\None.

\label Exceptional Situations:\None.

\label See Also::

{\secref\FloatSubstitutability}

\label Notes::
%% 12.5.2 11
\code
 (signum x) \EQ (if (zerop x) x (/ x (abs x)))
\endcode

\endcom

%%% ========== SQRT
%%% ========== ISQRT
\begincom{sqrt, isqrt}\ftype{Function}

\label Syntax::

\DefunWithValues sqrt  {number}  {root}
\DefunWithValues isqrt {natural} {natural-root}

\label Arguments and Values:: 

\param{number}, \param{root}---a \term{number}.

\param{natural}, \param{natural-root}---a non-negative \term{integer}.

\label Description::

%% 12.5.1 1
\funref{sqrt} and \funref{isqrt} compute square roots.

%% 12.5.1 9                                        
\funref{sqrt} returns the \term{principal} square root of \param{number}.
If the \param{number} is not a \term{complex} but is negative,
then the result is a \term{complex}.

%% 12.5.1 10
\funref{isqrt} returns the greatest \term{integer} 
less than or equal to the exact positive square root of \param{natural}.

If \param{number} is a positive \term{rational},
it is \term{implementation-dependent} 
whether \param{root} is a \term{rational} or a \term{float}.
If \param{number} is a negative \term{rational},
it is \term{implementation-dependent} 
whether \param{root} is a \term{complex rational} or a \term{complex float}.

\issue{IEEE-ATAN-BRANCH-CUT:SPLIT}
The mathematical definition of complex square root (whether or not
minus zero is supported) follows:

\f{(sqrt \i{x}) = (exp (/ (log \i{x}) 2))}

%% This is said in the definition of LOG. -kmp 19-Jan-92
% where
% \f{(log \i{x}) = (complex (log (abs \i{x})) (phase \i{x}))}.
\endissue{IEEE-ATAN-BRANCH-CUT:SPLIT}

The branch cut for square root lies along the negative real axis,
continuous with quadrant II.
The range consists of the right half-plane, including the non-negative
imaginary axis and excluding the negative imaginary axis.


\label Examples::

\code
 (sqrt 9.0) \EV 3.0
 (sqrt -9.0) \EV #C(0.0 3.0)
 (isqrt 9) \EV 3
 (sqrt 12) \EV 3.4641016
 (isqrt 12) \EV 3
 (isqrt 300) \EV 17
 (isqrt 325) \EV 18
 (sqrt 25)
\EV 5
\OV 5.0
 (isqrt 25) \EV 5
 (sqrt -1) \EV #C(0.0 1.0)
 (sqrt #c(0 2)) \EV #C(1.0 1.0)
\endcode

\label Side Effects:\None.

\label Affected By:\None.

\label Exceptional Situations::

\Thefunction{sqrt} should signal \typeref{type-error} if its argument 
is not a \term{number}.

\Thefunction{isqrt} should signal \typeref{type-error} if its argument 
is not a non-negative \term{integer}.

The functions \funref{sqrt} and \funref{isqrt} might signal \typeref{arithmetic-error}.

\label See Also::

\funref{exp},
\funref{log},
{\secref\FloatSubstitutability}

\label Notes::

\code
 (isqrt x) \EQ (values (floor (sqrt x))) 
\endcode
but it is potentially more efficient.

\endcom

%-------------------- Random Numbers --------------------

%% 2.11.0 1
\begincom{random-state}\ftype{System Class}

\label Class Precedence List::
\typeref{random-state},
\typeref{t}

\label Description::

A \term{random state} \term{object} contains state
information used by the pseudo-random number generator.
%% 12.9.0 1
%The pseudo-random numbers 
%in a random number series are \term{implementation-dependent},
%but the distribution of the numbers should
%machine-independent.
The nature of a \term{random state} \term{object} is \term{implementation-dependent}.
It can be printed out and successfully read back in by the same \term{implementation},
but might not function correctly as a \term{random state} in another \term{implementation}.

%%% 12.9.0 20
%
%A recommended way to implement \thetype{random-state}
%is effectively to use the machinery for \macref{defstruct}.

%% 12.9.0 18
\term{Implementations} are required to provide a read syntax for
\term{objects} \oftype{random-state}, but the specific nature of that syntax 
is \term{implementation-dependent}.

\label See Also::

\varref{*random-state*},
\funref{random},
{\secref\PrintingRandomStates}

\endcom%{random-state}\ftype{System Class}


%%% ========== MAKE-RANDOM-STATE
\begincom{make-random-state}\ftype{Function}

\label Syntax::

\DefunWithValues make-random-state {{\opt} state} {new-state}

\label Arguments and Values::

% This isn't quite susceptible to being called a designator, I think.
% The uses are too idiosyncratic to the fact that this is a creation operator.
% If another function took the same arguments, it would use them differently,
% so the ideas don't generalize well. -kmp 12-Sep-91
\param{state}---a \term{random state}, or \nil, or \t.
 \Default{\nil}

\param{new-state}---a \term{random state} \term{object}.

\label Description::

%% 12.9.0 12
Creates a \term{fresh} \term{object} \oftype{random-state} suitable
for use as \thevalueof{*random-state*}.

If \param{state} is a \term{random state} \term{object},
the \param{new-state} is a \term{copy}\meaning{5} of that \term{object}.
If \param{state} is \nil,
the \param{new-state} is a \term{copy}\meaning{5} of the \term{current random state}.
If \param{state} is \t,
the \param{new-state} is a \term{fresh} \term{random state} \term{object}
that has been randomly initialized by some means.

\label Examples::

\code
 (let* ((rs1 (make-random-state nil))
        (rs2 (make-random-state t))
        (rs3 (make-random-state rs2))
        (rs4 nil))
   (list (loop for i from 1 to 10 
               collect (random 100)
               when (= i 5)
                do (setq rs4 (make-random-state)))
         (loop for i from 1 to 10 collect (random 100 rs1))
         (loop for i from 1 to 10 collect (random 100 rs2))
         (loop for i from 1 to 10 collect (random 100 rs3))
         (loop for i from 1 to 10 collect (random 100 rs4))))
\EV ((29 25 72 57 55 68 24 35 54 65)
    (29 25 72 57 55 68 24 35 54 65)
    (93 85 53 99 58 62 2 23 23 59)
    (93 85 53 99 58 62 2 23 23 59)
    (68 24 35 54 65 54 55 50 59 49))
\endcode

\label Side Effects:\None.

\label Affected By:\None.

\label Exceptional Situations::

\Shouldchecktype{state}{a \term{random state}, or \nil, or \t}

\label See Also::

\funref{random}, \varref{*random-state*}

\label Notes::

%% 12.9.0 19
One important use of \funref{make-random-state} is to allow the same
series of pseudo-random \term{numbers} to be generated many times within a 
single program.

\endcom

%%% ========== RANDOM
\begincom{random}\ftype{Function}

\label Syntax::

\DefunWithValues random {limit {\opt} random-state} {random-number}

\label Arguments and Values::

\param{limit}---a positive \term{integer},
	     or a positive \term{float}.

%% 12.9.0 3
\param{random-state}---a \term{random state}.
 \Default{the \term{current random state}}

\param{random-number}---a non-negative \term{number}
			less than \param{limit} 
		    and of the same \term{type} as \param{limit}.

\label Description::

%% 12.9.0 2
%% 12.9.0 5
%% 12.9.0 6
%% 12.9.0 7
%% 12.9.0 8
%% 12.9.0 9
%% 12.9.0 10

Returns a pseudo-random number that is a non-negative \term{number}
less than \param{limit} and of the same \term{type} as \param{limit}.

The \param{random-state}, which is modified by this function,
encodes the internal state maintained by the random number generator.

An approximately uniform choice distribution is used.  If \param{limit}
is an \term{integer}, each of the possible results occurs with
(approximate) probability 1/\param{limit}.

\label Examples::

\code
 (<= 0 (random 1000) 1000) \EV \term{true}
 (let ((state1 (make-random-state))
       (state2 (make-random-state)))
   (= (random 1000 state1) (random 1000 state2))) \EV \term{true}
\endcode

\label Side Effects::

The \param{random-state} is modified.

\label Affected By:\None.

\label Exceptional Situations::

\Shouldchecktype{limit}{a positive \term{integer} or a positive \term{real}}

\label See Also::

\funref{make-random-state}, \varref{*random-state*}

\label Notes::

See \CLtL\ for information about generating random numbers.

\endcom

%%% ========== RANDOM-STATE-P
\begincom{random-state-p}\ftype{Function}

\label Syntax::

\DefunWithValues random-state-p {object} {generalized-boolean}

\label Arguments and Values::

\param{object}---an \term{object}.

\param{generalized-boolean}---a \term{generalized boolean}.

\label Description::

%% 12.9.0 21
\TypePredicate{object}{random-state}

\label Examples::

\code
 (random-state-p *random-state*) \EV \term{true}
 (random-state-p (make-random-state)) \EV \term{true}
 (random-state-p 'test-function) \EV \term{false}
\endcode

\label Side Effects:\None.

\label Affected By:\None.

\label Exceptional Situations:\None!

\label See Also::

\funref{make-random-state}, \varref{*random-state*}

\label Notes::

\code
 (random-state-p \param{object}) \EQ (typep \param{object} 'random-state)
\endcode

\endcom

%%% ========== *RANDOM-STATE*
\begincom{*random-state*}\ftype{Variable}

\label Value Type::

a \term{random state}.

\label Initial Value::

\term{implementation-dependent}.

\label Description::

%% 12.9.0 11
The \term{current random state}, which is used, for example,
by \thefunction{random} when a \term{random state} is not explicitly supplied.

\label Examples::      

\code
 (random-state-p *random-state*) \EV \term{true}
 (setq snap-shot (make-random-state))
 ;; The series from any given point is random,
 ;; but if you backtrack to that point, you get the same series.
 (list (loop for i from 1 to 10 collect (random))
       (let ((*random-state* snap-shot))
         (loop for i from 1 to 10 collect (random)))
       (loop for i from 1 to 10 collect (random))
       (let ((*random-state* snap-shot))
         (loop for i from 1 to 10 collect (random))))
\EV ((19 16 44 19 96 15 76 96 13 61)
    (19 16 44 19 96 15 76 96 13 61)
    (16 67 0 43 70 79 58 5 63 50)
    (16 67 0 43 70 79 58 5 63 50))
\endcode

\label Affected By::

The \term{implementation}.

\funref{random}.

\label See Also::

\funref{make-random-state},
\funref{random},
\typeref{random-state}

\label Notes::

\term{Binding} \varref{*random-state*} to a different 
\term{random state} \term{object} correctly saves and 
restores the old \term{random state} \term{object}.

\endcom

%-------------------- Numeric Types --------------------

%%% ========== NUMBERP
\begincom{numberp}\ftype{Function}

\label Syntax::

\DefunWithValues numberp {object} {generalized-boolean}

\label Arguments and Values::

\param{object}---an \term{object}.

\param{generalized-boolean}---a \term{generalized boolean}.

\label Description::

%% 6.2.2 11
\TypePredicate{object}{number}

\label Examples::

\code
 (numberp 12) \EV \term{true}
 (numberp (expt 2 130)) \EV \term{true}
 (numberp #c(5/3 7.2)) \EV \term{true}
 (numberp nil) \EV \term{false}
 (numberp (cons 1 2)) \EV \term{false}
\endcode

\label Side Effects:\None.

\label Affected By:\None.

\label Exceptional Situations:\None!

\label See Also:\None.

\label Notes::

\code
 (numberp \param{object}) \EQ (typep \param{object} 'number)
\endcode

\endcom

%-------------------- Complex --------------------

%%% ========== CIS
\begincom{cis}\ftype{Function}

\label Syntax::

\DefunWithValues cis {radians} {number}

\label Arguments and Values::

\param{radians}---a \term{real}.

\param{number}---a \term{complex}.

\label Description::

%% 12.5.2 15
\funref{cis} returns the value of~$\i{e}^{i\cdot\ \i{radians}}$,
which is a \term{complex} in which the
real part is equal to the cosine of \param{radians}, and the 
imaginary part is equal to the sine of \param{radians}.

\label Examples::
\code
 (cis 0) \EV #C(1.0 0.0)
\endcode

\label Side Effects:\None.

\label Affected By:\None.

\label Exceptional Situations:\None.

\label See Also::

{\secref\FloatSubstitutability}

\label Notes:\None.

%The result is a complex number whose
%phase is the equal to the argument (mod 2$\pi$)
%and whose magnitude is unity.

\endcom

%%% ========== COMPLEX
\begincom{complex}\ftype{Function}

\label Syntax::

\DefunWithValues complex {realpart {\opt} imagpart} {complex}

\label Arguments and Values::

\param{realpart}---a \term{real}.

\param{imagpart}---a \term{real}.

\param{complex}---a \term{rational} or a \term{complex}.

\label Description::

\funref{complex} returns a \term{number} 
    whose real      part is \param{realpart} 
and whose imaginary part is \param{imagpart}.

If \param{realpart} is a \term{rational}
and \param{imagpart} is the \term{rational} number zero, 
the result of \funref{complex} is \param{realpart}, a \term{rational}.
Otherwise, the result is a \term{complex}.

%% 12.6.0 38
If either \param{realpart} or \param{imagpart} is a \term{float},
the non-\term{float} is converted to a \term{float} 
before the \term{complex} is created.
If \param{imagpart} is not supplied, the imaginary part is a 
zero of the same \term{type} as \param{realpart}; \ie
{\tt (coerce 0 (type-of \param{realpart}))} is
effectively used.  

\issue{ARRAY-TYPE-ELEMENT-TYPE-SEMANTICS:UNIFY-UPGRADING}

Type upgrading implies a movement upwards in the type 
hierarchy lattice.  
%!!! Barmar asks: ``What type specifier are you talking about?''
In the case of \term{complexes}, the \param{type-specifier} 
\reviewer{Barmar: What type specifier?}
must be a subtype of 
{\tt (upgraded-complex-part-type \param{type-specifier})}.
If \param{type-specifier1} is a subtype of \param{type-specifier2}, then
{\tt (upgraded-complex-element-type '\param{type-specifier1})}
must also be a subtype of
{\tt (upgraded-complex-element-type '\param{type-specifier2})}.  
Two disjoint types can be upgraded into 
the same thing.
 
\endissue{ARRAY-TYPE-ELEMENT-TYPE-SEMANTICS:UNIFY-UPGRADING}

\label Examples::
\code
 (complex 0) \EV 0
 (complex 0.0) \EV #C(0.0 0.0)
 (complex 1 1/2) \EV #C(1 1/2)
 (complex 1 .99) \EV #C(1.0 0.99)
 (complex 3/2 0.0) \EV #C(1.5 0.0)
\endcode


\label Side Effects:\None.

\label Affected By:\None.

\label Exceptional Situations:\None.

\label See Also::

\funref{realpart}, \funref{imagpart}

\label Notes::

\code
 #c(a b) \EQ #.(complex a b)
\endcode
\endcom

%%% ========== COMPLEXP
\begincom{complexp}\ftype{Function}

\label Syntax::

\DefunWithValues complexp {object} {generalized-boolean}

\label Arguments and Values::

\param{object}---an \term{object}.

\param{generalized-boolean}---a \term{generalized boolean}.

\label Description::

%% 6.2.2 16
\TypePredicate{object}{complex}

\label Examples::
\code
 (complexp 1.2d2) \EV \term{false}
 (complexp #c(5/3 7.2)) \EV \term{true}

\endcode

\label Side Effects:\None.

\label Affected By:\None.

\label Exceptional Situations:\None!

\label See Also::

\funref{complex} (\term{function} and \term{type}), \funref{typep}

\label Notes::

\code
 (complexp \param{object}) \EQ (typep \param{object} 'complex)
\endcode

\endcom

%%% ========== CONJUGATE
\begincom{conjugate}\ftype{Function}

\label Syntax::

\DefunWithValues conjugate {number} {conjugate}

\label Arguments and Values::

\param{number}---a \term{number}.

\param{conjugate}---a \term{number}.

\label Description::

%% 12.4.0 21
Returns the complex conjugate of \param{number}.
The conjugate of a 
\issue{REAL-NUMBER-TYPE:X3J13-MAR-89}
\term{real}
\endissue{REAL-NUMBER-TYPE:X3J13-MAR-89}
number is itself.  

\label Examples::

\code
 (conjugate #c(0 -1)) \EV #C(0 1)
 (conjugate #c(1 1)) \EV #C(1 -1)
 (conjugate 1.5) \EV 1.5
 (conjugate #C(3/5 4/5)) \EV #C(3/5 -4/5)
 (conjugate #C(0.0D0 -1.0D0)) \EV #C(0.0D0 1.0D0)
 (conjugate 3.7) \EV 3.7
\endcode

\label Side Effects:\None.

\label Affected By:\None.

\label Exceptional Situations:\None.

\label See Also:\None.

\label Notes::

For a \term{complex} number \f{z},

\code
 (conjugate z) \EQ (complex (realpart z) (- (imagpart z)))
\endcode

\endcom

%%% ========== PHASE
\begincom{phase}\ftype{Function}

\label Syntax::

\DefunWithValues phase {number} {phase}

\label Arguments and Values::

\param{number}---a \term{number}.

\param{phase}---a \term{number}.

\label Description::

%% 12.5.2 8
%% 12.5.2 9
\funref{phase} 
returns the phase 
of \param{number} (the angle part of its polar representation)
in radians, in the range 
\issue{IEEE-ATAN-BRANCH-CUT:SPLIT}
$-\pi$ (exclusive) if minus zero is not supported, or 
$-\pi$ (inclusive) if minus zero is supported, 
\endissue{IEEE-ATAN-BRANCH-CUT:SPLIT}
to $\pi$ (inclusive).  The phase of a positive 
\issue{REAL-NUMBER-TYPE:X3J13-MAR-89}
\term{real}
\endissue{REAL-NUMBER-TYPE:X3J13-MAR-89}
number
is zero; that of a negative 
\issue{REAL-NUMBER-TYPE:X3J13-MAR-89}
\term{real}
\endissue{REAL-NUMBER-TYPE:X3J13-MAR-89}
number is $\pi$.
The phase of zero is defined to be zero.

%% 12.5.2 10
If \param{number} is a \term{complex float},  
the result is a \term{float} of the same \term{type} 
as the components of \param{number}.
If \param{number} is a \term{float}, the result is a
\term{float} of the same \term{type}.
If \param{number} is a \term{rational} or a \term{complex rational},
the result is a \term{single float}.  

%% 12.5.3 6
The branch cut for \funref{phase} lies along the negative real
axis, continuous with quadrant II.  The range consists of that portion of
the real axis between $-\pi$ (exclusive) and~$\pi$ (inclusive).

\issue{IEEE-ATAN-BRANCH-CUT:SPLIT}
The mathematical definition of \funref{phase} is as follows:

\f{(phase \i{x}) = (atan (imagpart \i{x}) (realpart \i{x}))}
\endissue{IEEE-ATAN-BRANCH-CUT:SPLIT}
 
\label Examples::

\code
 (phase 1) \EV 0.0s0
 (phase 0) \EV 0.0s0
 (phase (cis 30)) \EV -1.4159266
 (phase #c(0 1)) \EV 1.5707964
\endcode

\label Side Effects:\None.

\label Affected By:\None.

\label Exceptional Situations::

Should signal \typeref{type-error} if its argument is not a \term{number}.
Might signal \typeref{arithmetic-error}.

\label See Also::

{\secref\FloatSubstitutability}

\label Notes:\None.

\endcom

%%% ========== REALPART
%%% ========== IMAGPART
\begincom{realpart, imagpart}\ftype{Function}

\label Syntax::

\DefunWithValues realpart {number} {real}
\DefunWithValues imagpart {number} {real}

\label Arguments and Values:: 

\param{number}---a \term{number}.

\param{real}---a \term{real}.

\label Description::

\funref{realpart} and \funref{imagpart} return the real and
imaginary parts of \param{number} respectively.
%% 12.6.0 39
If \param{number} is 
\issue{REAL-NUMBER-TYPE:X3J13-MAR-89}
\term{real},
\endissue{REAL-NUMBER-TYPE:X3J13-MAR-89}
then \funref{realpart} returns \param{number} and \funref{imagpart}
returns \f{(* 0 \param{number})}, which has the effect that the
imaginary part of a \term{rational} is \f{0} and that of
a \term{float} is a floating-point zero of the same format.

\label Examples::

\code
 (realpart #c(23 41)) \EV 23
 (imagpart #c(23 41.0)) \EV 41.0
 (realpart #c(23 41.0)) \EV 23.0
 (imagpart 23.0) \EV 0.0
\endcode

\label Side Effects:\None.

\label Affected By:\None.

\label Exceptional Situations::

\Shouldchecktype{number}{a \term{number}}

\label See Also::

\funref{complex}

\label Notes:\None.

\endcom

%%% ========== UPGRADED-COMPLEX-PART-TYPE
\begincom{upgraded-complex-part-type}\ftype{Function}

\issue{SUBTYPEP-ENVIRONMENT:ADD-ARG}
\issue{ARRAY-TYPE-ELEMENT-TYPE-SEMANTICS:UNIFY-UPGRADING}

\label Syntax::

\DefunWithValues upgraded-complex-part-type
		 {typespec {\opt} environment}
       	         {upgraded-typespec}

\label Arguments and Values:: 

\param{typespec}---a \term{type specifier}.

\param{environment}---an \term{environment} \term{object}.
  \Default{\nil, denoting the \term{null lexical environment}
	   and the and current \term{global environment}}
%!!! Need to say what happens with the environment.

\param{upgraded-typespec}---a \term{type specifier}.

\label Description::

\funref{upgraded-complex-part-type} returns the part type of the
most specialized \term{complex} number representation that can
hold parts of \term{type} \term{typespec}.

%Added by KMP in response to a Barmar comment. -kmp 29-Jul-91
The \param{typespec} is a \term{subtype} of 
(and possibly \term{type equivalent} to)
the \param{upgraded-typespec}.

The purpose of \funref{upgraded-complex-part-type}
is to reveal how an implementation does its \term{upgrading}.

\label Examples:\None.

\label Side Effects:\None.

\label Affected By:\None.

\label Exceptional Situations:\None.

\label See Also::

\funref{complex} (\term{function} and \term{type})

\label Notes::

\endissue{ARRAY-TYPE-ELEMENT-TYPE-SEMANTICS:UNIFY-UPGRADING}
\endissue{SUBTYPEP-ENVIRONMENT:ADD-ARG}

\endcom

%-------------------- Real --------------------

%%% ========== REALP
\begincom{realp}\ftype{Function}

\issue{REAL-NUMBER-TYPE:X3J13-MAR-89}

\label Syntax::

\DefunWithValues realp {object} {generalized-boolean}

\label Arguments and Values::

\param{object}---an \term{object}.

\param{generalized-boolean}---a \term{generalized boolean}.

\label Description::

%% 6.2.2 11
\TypePredicate{object}{real}

\label Examples::
\code
 (realp 12) \EV \term{true}
 (realp #c(5/3 7.2)) \EV \term{false}
 (realp nil) \EV \term{false}
 (realp (cons 1 2)) \EV \term{false}
\endcode

\label Side Effects:\None.

\label Affected By:\None.

\label Exceptional Situations:\None!

\label See Also:\None.

\label Notes::

\code
 (realp \param{object}) \EQ (typep \param{object} 'real)
\endcode

\endissue{REAL-NUMBER-TYPE:X3J13-MAR-89}
\endcom

%-------------------- Rational --------------------

%%% ========== NUMERATOR
%%% ========== DENOMINATOR
\begincom{numerator, denominator}\ftype{Function}

\label Syntax::

\DefunWithValues numerator   {rational} {numerator}
\DefunWithValues denominator {rational} {denominator}

\label Arguments and Values::

\param{rational}---a \term{rational}.

\param{numerator}---an \term{integer}.

\param{denominator}---a positive \term{integer}.

\label Description::

%% 12.6.0 8
\funref{numerator} and \funref{denominator} reduce \param{rational}
to canonical form and compute the numerator or denominator of that number.

\funref{numerator} and \funref{denominator} return the numerator 
or denominator of the canonical form of \param{rational}.

If \param{rational} is an \term{integer}, 
\funref{numerator} returns \param{rational}
and \funref{denominator} returns 1.

\label Examples::
\code
 (numerator 1/2) \EV 1
 (denominator 12/36) \EV 3
 (numerator -1) \EV -1
 (denominator (/ -33)) \EV 33
 (numerator (/ 8 -6)) \EV -4
 (denominator (/ 8 -6)) \EV 3
\endcode

\label Side Effects:\None.

\label Affected By:\None.

\label Exceptional Situations:\None.

\label See Also::

\funref{/}

\label Notes::
\code
 (gcd (numerator x) (denominator x)) \EV 1
\endcode

\endcom

%%% ========== RATIONAL
%%% ========== RATIONALIZE
\begincom{rational, rationalize}\ftype{Function}

\label Syntax::

\DefunWithValues rational    {number} {rational}
\DefunWithValues rationalize {number} {rational}

\label Arguments and Values::

\param{number}---a \term{real}.

\param{rational}---a \term{rational}.

\label Description::

%% 12.6.0 5
\funref{rational} and \funref{rationalize} convert 
\issue{REAL-NUMBER-TYPE:X3J13-MAR-89}
\term{reals}
\endissue{REAL-NUMBER-TYPE:X3J13-MAR-89}
to \term{rationals}.

If \param{number} is already \term{rational}, it is returned.

If \param{number} is a \term{float}, 
\funref{rational} returns a \term{rational} 
that is mathematically equal in value to the \term{float}. 
\funref{rationalize} returns a \term{rational} that
approximates the \term{float} to the accuracy of 
the underlying floating-point representation. 

\funref{rational} assumes that the \term{float} is completely accurate.

%% 12.6.0 6
\funref{rationalize} assumes that the
\term{float} is accurate only to the precision of the
floating-point representation.

\label Examples::
\code
 (rational 0) \EV 0
 (rationalize -11/100) \EV -11/100
 (rational .1) \EV 13421773/134217728 ;implementation-dependent
 (rationalize .1) \EV 1/10
\endcode

\label Side Effects:\None.

\label Affected By::

The \term{implementation}.

\label Exceptional Situations::

\Shouldchecktype{number}{a \term{real}}
Might signal \typeref{arithmetic-error}.

\label See Also:\None.

\label Notes::
%% 12.6.0 7

It is always the case that

\code
 (float (rational x) x) \EQ x
\endcode
and

\code
 (float (rationalize x) x) \EQ x
\endcode
That is, rationalizing a \term{float} by either method
and then converting it back
to a \term{float} 
of the same format produces the original \param{number}.
\endcom

%%% ========== RATIONALP
\begincom{rationalp}\ftype{Function}

\label Syntax::

\DefunWithValues rationalp {object} {generalized-boolean}

\label Arguments and Values::

\param{object}---an \term{object}.

\param{generalized-boolean}---a \term{generalized boolean}.

\label Description::

%% 6.2.2 14
\TypePredicate{object}{rational}

\label Examples::

\code
 (rationalp 12) \EV \term{true}
 (rationalp 6/5) \EV \term{true}
 (rationalp 1.212) \EV \term{false}
\endcode

\label Side Effects:\None.

\label Affected By:\None.

\label Exceptional Situations:\None!

\label See Also::

\funref{rational}

\label Notes::
\code
 (rationalp \param{object}) \EQ (typep \param{object} 'rational)
\endcode

\endcom

%-------------------- Integer --------------------

%%% ========== ASH
\begincom{ash}\ftype{Function}

\label Syntax::

\DefunWithValues ash {integer count} {shifted-integer}

\label Arguments and Values::

\param{integer}---an \term{integer}.

\param{count}---an \term{integer}.

\param{shifted-integer}---an \term{integer}.

\label Description::

%% 12.7.0 18
\funref{ash} performs the arithmetic shift operation on the binary
representation of \param{integer}, which is treated as if it were binary.


\funref{ash} shifts \param{integer} arithmetically left by \param{count} bit
positions if \param{count} is positive,
or right \param{count} bit positions if \param{count} is negative.
The shifted value of the same sign 
as \param{integer} is returned.

Mathematically speaking, \funref{ash} performs the computation
\f{floor}(\param{integer}\centerdot $2^\param{count}$).
%% 12.7.0 20
Logically, \funref{ash} 
moves all of the bits in \param{integer} to the left,
adding zero-bits at the right, or moves them to the right,
discarding bits.  

\funref{ash} is defined to behave as if \param{integer} were 
represented in two's complement form, regardless of
how \term{integers} are represented internally.
\label Examples::
\code
 (ash 16 1) \EV 32
 (ash 16 0) \EV 16
 (ash 16 -1) \EV 8
 (ash -100000000000000000000000000000000 -100) \EV -79
\endcode


\label Affected By:\None.

\label Exceptional Situations::

\Shouldchecktype{integer}{an \term{integer}}
\Shouldchecktype{count}{an \term{integer}}
Might signal \typeref{arithmetic-error}.

\label See Also:\None.

\label Notes::
%% 12.7.0 19

\code
 (logbitp \param{j} (ash \param{n} \param{k}))
 \EQ (and (>= \param{j} \param{k}) (logbitp (- \param{j} \param{k}) \param{n}))
\endcode
\endcom

%%% ========== INTEGER-LENGTH
\begincom{integer-length}\ftype{Function}

\label Syntax::

\DefunWithValues integer-length {integer} {number-of-bits}

\label Arguments and Values::

\param{integer}---an \term{integer}.

\param{number-of-bits}---a non-negative \term{integer}.
                                                                               
\label Description::

%% 12.7.0 22
Returns the number of bits needed to represent \param{integer}
in binary two's-complement format.

\label Examples::

%% 12.7.0 23
\code
 (integer-length 0) \EV 0
 (integer-length 1) \EV 1
 (integer-length 3) \EV 2
 (integer-length 4) \EV 3
 (integer-length 7) \EV 3
 (integer-length -1) \EV 0
 (integer-length -4) \EV 2
 (integer-length -7) \EV 3
 (integer-length -8) \EV 3
 (integer-length (expt 2 9)) \EV 10
 (integer-length (1- (expt 2 9))) \EV 9
 (integer-length (- (expt 2 9))) \EV 9
 (integer-length (- (1+ (expt 2 9)))) \EV 10
\endcode

\label Side Effects:\None.

\label Affected By:\None.

\label Exceptional Situations::

\Shouldchecktype{integer}{an \term{integer}}

\label See Also:\None.

\label Notes::

This function could have been defined by:

\code
(defun integer-length (integer)
  (ceiling (log (if (minusp integer)
                    (- integer)
                    (1+ integer))
                2)))
\endcode

If \param{integer} is non-negative, then its value can be represented
in unsigned binary form in a field whose width in bits is
no smaller than {\tt (integer-length \param{integer})}.
Regardless of the sign of \param{integer}, its value can be
represented in signed binary two's-complement form in a field
whose width in bits is no smaller than {\tt (+ (integer-length \param{integer}) 1)}.

\endcom

%%% ========== INTEGERP
\begincom{integerp}\ftype{Function}

\label Syntax::

\DefunWithValues integerp {object} {generalized-boolean}

\label Arguments and Values::

\param{object}---an \term{object}.

\param{generalized-boolean}---a \term{generalized boolean}.

\label Description::

%% 6.2.2 12
\TypePredicate{object}{integer}

\label Examples::
\code
 (integerp 1) \EV \term{true}
 (integerp (expt 2 130)) \EV \term{true}
 (integerp 6/5) \EV \term{false}
 (integerp nil) \EV \term{false}

\endcode

\label Side Effects:\None.

\label Affected By:\None.

\label Exceptional Situations:\None!

\label See Also:\None.

\label Notes::

\code
 (integerp \param{object}) \EQ (typep \param{object} 'integer)
\endcode

\endcom

%%% ========== PARSE-INTEGER
\begincom{parse-integer}\ftype{Function}

\label Syntax::

\DefunWithValues {parse-integer} {string {\key} start end radix junk-allowed} {integer, pos}

\label Arguments and Values::

\param{string}---a \term{string}.

\issue{SUBSEQ-OUT-OF-BOUNDS}
\issue{RANGE-OF-START-AND-END-PARAMETERS:INTEGER-AND-INTEGER-NIL}
\param{start}, \param{end}---\term{bounding index designators} of \param{string}.
 \Defaults{\param{start} and \param{end}}{\f{0} and \nil}
\endissue{RANGE-OF-START-AND-END-PARAMETERS:INTEGER-AND-INTEGER-NIL}
\endissue{SUBSEQ-OUT-OF-BOUNDS}

\param{radix}---a \term{radix}.
 \Default{\f{10}}

\param{junk-allowed}---a \term{generalized boolean}.
 \Default{\term{false}}

\param{integer}---an \term{integer} or \term{false}.

\param{pos}---a \term{bounding index} of \param{string}.

\label Description::

%% 22.2.1 45
\funref{parse-integer} parses an \term{integer} in the specified \param{radix}
from the substring of \param{string} delimited by \param{start} and \param{end}.

%% 22.2.1 49
\funref{parse-integer} expects an optional sign (\f{+} or \f{-}) followed by
a a non-empty sequence of digits to be interpreted in the specified \param{radix}.
Optional leading and trailing \term{whitespace}\meaning{1} is ignored.

\funref{parse-integer} does not recognize the syntactic radix-specifier
prefixes \f{\#O}, \f{\#B}, \f{\#X}, and \f{\#\i{n}R}, 
nor does it recognize a trailing decimal point.

If \param{junk-allowed} is \term{false}, an error \oftype{parse-error} is 
signaled if substring does not consist entirely of the representation of a 
signed \term{integer}, possibly surrounded on either side by \term{whitespace}\meaning{1} 
\term{characters}.

%% 22.2.1 46
The first \term{value} returned is either
    the \term{integer} that was parsed,
 or else \nil\ if no syntactically correct \term{integer} 
    was seen but \param{junk-allowed} was \term{true}.

%% 22.2.1 48
The second \term{value} is either 
    the index into the \term{string} of the delimiter that terminated the parse,
 or the upper \term{bounding index} of the substring if the parse terminated at
    the end of the substring (as is always the case if \param{junk-allowed} 
    is \term{false}).

% KMP:
%  This is actually somewhat ambiguous.
%   A. (parse-integer "3 5" :junk-allowed nil :end 2) \EV 3,1 or 3,2 ??
%   B. (parse-integer "3 5" :junk-allowed t :end 2)   \EV 3,1 or 3,2 ??
%   C. (parse-integer "3 a" :junk-allowed t)          \EV 3,1 or 3,2 ??
% Mail sent to Quinquevirate asking if anyone had any thoughts on this.

\label Examples::
\code
 (parse-integer "123") \EV 123, 3
 (parse-integer "123" :start 1 :radix 5) \EV 13, 3
 (parse-integer "no-integer" :junk-allowed t) \EV NIL, 0
\endcode

\label Side Effects:\None.

\label Affected By:\None.

\label Exceptional Situations::

If \param{junk-allowed} is \term{false}, 
an error is signaled if substring does not consist entirely of
the representation of an \term{integer}, 
possibly surrounded on either side by
\term{whitespace}\meaning{1} characters.

%% Per X3J13. -kmp 05-Oct-93
\label See Also:\None.
                      
\label Notes:\None.

\endcom

%-------------------- Integer Bits --------------------

%%% ========== BOOLE
\begincom{boole}\ftype{Function}

\label Syntax::

\DefunWithValues boole {op integer-1 integer-2} {result-integer}

\label Arguments and Values:: 

\param{Op}---a \term{bit-wise logical operation specifier}.

\param{integer-1}---an \term{integer}.

\param{integer-2}---an \term{integer}.

\param{result-integer}---an \term{integer}.

\label Description::

%% 12.7.0 12
\funref{boole} performs bit-wise logical operations on
\param{integer-1} and \param{integer-2}, which are treated as if
they were binary and in two's complement representation.

The operation to be performed and the return value are determined by 
\param{op}.  

\funref{boole} returns the values 
specified for any \param{op} in \thenextfigure.

{
\def\zz{0\ \ \ 0}
\def\ww{1\ \ \ 1}
\def\zw{0\ \ \ 1}
\def\wz{1\ \ \ 0}
\def\sp{\ \ \ }

\tablefigtwo{Bit-Wise Logical Operations}{Op}{Result}{
 \varref{boole-1}     & \param{integer-1}                                          \cr
 \varref{boole-2}     & \param{integer-2}                                          \cr
 \varref{boole-andc1} & and complement of \param{integer-1} with \param{integer-2} \cr
 \varref{boole-andc2} & and \param{integer-1} with complement of \param{integer-2} \cr
 \varref{boole-and}   & and                                                        \cr
 \varref{boole-c1}    & complement of \param{integer-1}                            \cr
 \varref{boole-c2}    & complement of \param{integer-2}                            \cr
 \varref{boole-clr}   & always 0 (all zero bits)                                   \cr
 \varref{boole-eqv}   & equivalence (exclusive nor)                                \cr
 \varref{boole-ior}   & inclusive or                                               \cr
 \varref{boole-nand}  & not-and                                                    \cr
 \varref{boole-nor}   & not-or                                                     \cr
 \varref{boole-orc1}  & or complement of \param{integer-1} with \param{integer-2}  \cr
 \varref{boole-orc2}  & or \param{integer-1} with complement of \param{integer-2}  \cr
 \varref{boole-set}   & always -1 (all one bits)                                   \cr
 \varref{boole-xor}   & exclusive or                                               \cr
}
}

\label Examples::

\code
 (boole boole-ior 1 16) \EV 17
 (boole boole-and -2 5) \EV 4
 (boole boole-eqv 17 15) \EV -31

;;; These examples illustrate the result of applying BOOLE and each
;;; of the possible values of OP to each possible combination of bits.
 (progn
   (format t "~&Results of (BOOLE <op> #b0011 #b0101) ...~
           ~%---Op-------Decimal-----Binary----Bits---~%")
   (dolist (symbol '(boole-1     boole-2    boole-and  boole-andc1
                     boole-andc2 boole-c1   boole-c2   boole-clr
                     boole-eqv   boole-ior  boole-nand boole-nor
                     boole-orc1  boole-orc2 boole-set  boole-xor))
     (let ((result (boole (symbol-value symbol) #b0011 #b0101)))
       (format t "~& ~A~13T~3,' D~23T~:*~5,' B~31T ...~4,'0B~%" 
               symbol result (logand result #b1111)))))
\OUT Results of (BOOLE <op> #b0011 #b0101) ...
\OUT ---Op-------Decimal-----Binary----Bits---
\OUT  BOOLE-1       3          11    ...0011
\OUT  BOOLE-2       5         101    ...0101
\OUT  BOOLE-AND     1           1    ...0001
\OUT  BOOLE-ANDC1   4         100    ...0100
\OUT  BOOLE-ANDC2   2          10    ...0010
\OUT  BOOLE-C1     -4        -100    ...1100
\OUT  BOOLE-C2     -6        -110    ...1010
\OUT  BOOLE-CLR     0           0    ...0000
\OUT  BOOLE-EQV    -7        -111    ...1001
\OUT  BOOLE-IOR     7         111    ...0111
\OUT  BOOLE-NAND   -2         -10    ...1110
\OUT  BOOLE-NOR    -8       -1000    ...1000
\OUT  BOOLE-ORC1   -3         -11    ...1101
\OUT  BOOLE-ORC2   -5        -101    ...1011
\OUT  BOOLE-SET    -1          -1    ...1111
\OUT  BOOLE-XOR     6         110    ...0110
\EV NIL
\endcode

%\filbreak


\label Affected By:\None.

\label Exceptional Situations::

Should signal \typeref{type-error} if its first argument is not a 
\term{bit-wise logical operation specifier} or if any subsequent argument is not
an \term{integer}.

\label See Also::

\funref{logand}

\label Notes::  

%% 12.7.0 14
In general,

\code
 (boole boole-and x y) \EQ (logand x y)
\endcode

%% The following example contributed by Dan Hoey (hoey@AIC.NRL.Navy.Mil)
\term{Programmers} who would prefer to use numeric indices rather than 
\term{bit-wise logical operation specifiers} can get an equivalent effect
by a technique such as the following:

\code
;; The order of the values in this `table' are such that
;; (logand (boole (elt boole-n-vector n) #b0101 #b0011) #b1111) => n
 (defconstant boole-n-vector
    (vector boole-clr   boole-and  boole-andc1 boole-2
            boole-andc2 boole-1    boole-xor   boole-ior
            boole-nor   boole-eqv  boole-c1    boole-orc1
            boole-c2    boole-orc2 boole-nand  boole-set))
\EV BOOLE-N-VECTOR
 (proclaim '(inline boole-n))
\EV \term{implementation-dependent}
 (defun boole-n (n integer &rest more-integers)
   (apply #'boole (elt boole-n-vector n) integer more-integers))
\EV BOOLE-N
 (boole-n #b0111 5 3) \EV 7
 (boole-n #b0001 5 3) \EV 1
 (boole-n #b1101 5 3) \EV -3
 (loop for n from #b0000 to #b1111 collect (boole-n n 5 3))
\EV (0 1 2 3 4 5 6 7 -8 -7 -6 -5 -4 -3 -2 -1)
\endcode

\endcom

%%% ========== BOOLE-1
%%% ========== BOOLE-2
%%% ========== BOOLE-AND
%%% ========== BOOLE-ANDC1
%%% ========== BOOLE-ANDC2
%%% ========== BOOLE-C1
%%% ========== BOOLE-C2
%%% ========== BOOLE-CLR
%%% ========== BOOLE-EQV
%%% ========== BOOLE-IOR
%%% ========== BOOLE-NAND
%%% ========== BOOLE-NOR
%%% ========== BOOLE-ORC1
%%% ========== BOOLE-ORC2
%%% ========== BOOLE-SET
%%% ========== BOOLE-XOR
\begincom{boole-1,    boole-2,   boole-and,  boole-andc1, boole-andc2,
	  boole-c1,   boole-c2,  boole-clr,  boole-eqv,   boole-ior,
          boole-nand, boole-nor, boole-orc1, boole-orc2,  boole-set, 
	  boole-xor}\ftype{Constant Variable}

\label Constant Value::

The identity and nature of the \term{values} of each of these \term{variables} 
is \term{implementation-dependent},
except that it must be \term{distinct} from each of the \term{values} of the others,
and it must be a valid first \term{argument} to \thefunction{boole}.

\label Description::

Each of these \term{constants} has a \term{value} which is one of the 
sixteen possible \term{bit-wise logical operation specifiers}.

\label Examples::
\code
 (boole boole-ior 1 16) \EV 17
 (boole boole-and -2 5) \EV 4
 (boole boole-eqv 17 15) \EV -31
\endcode

\label See Also::

\funref{boole}

%% Per X3J13. -kmp 05-Oct-93
\label Notes:\None.  

\endcom

%%% ========== LOGNAND
%%% ========== LOGEQV
%%% ========== LOGNOT
%%% ========== LOGIOR
%%% ========== LOGXOR
%%% ========== LOGNAND
%%% ========== LOGNOR
%%% ========== LOGANDC1
%%% ========== LOGANDC2
%%% ========== LOGORC1
%%% ========== LOGORC2
\begincom{logand,  logandc1, logandc2, logeqv,  logior,
	  lognand, lognor,   lognot,   logorc1, logorc2,
	  logxor}\ftype{Function}

\label Syntax::

\DefunWithValues logand {{\rest} integers} {result-integer}
\DefunWithValues logandc1 {integer-1 integer-2} {result-integer}
\DefunWithValues logandc2 {integer-1 integer-2} {result-integer}
\DefunWithValues logeqv {{\rest} integers} {result-integer}
\DefunWithValues logior {{\rest} integers} {result-integer}
\DefunWithValues lognand  {integer-1 integer-2} {result-integer}
\DefunWithValues lognor   {integer-1 integer-2} {result-integer}
\DefunWithValues lognot {integer} {result-integer}
\DefunWithValues logorc1  {integer-1 integer-2} {result-integer}
\DefunWithValues logorc2  {integer-1 integer-2} {result-integer}
\DefunWithValues logxor {{\rest} integers} {result-integer}

\label Arguments and Values::

\param{integers}---\term{integers}.

\param{integer}---an \term{integer}.

\param{integer-1}---an \term{integer}.

\param{integer-2}---an \term{integer}.

\param{result-integer}---an \term{integer}.

\label Description::

The \term{functions}
       \funref{logandc1},
       \funref{logandc2},
       \funref{logand},
       \funref{logeqv},
       \funref{logior},
       \funref{lognand},
       \funref{lognor}, 
       \funref{lognot},
       \funref{logorc1}, 
       \funref{logorc2},
   and \funref{logxor}
perform bit-wise logical operations on their \term{arguments},
that are treated as if they were binary.

\Thenextfigure\ lists the meaning of each of the \term{functions}.
Where an `identity' is shown, it indicates the \term{value} \term{yielded}
by the \term{function} when no \term{arguments} are supplied.
%% 12.7.0 11
%% 12.7.0 8
%% 12.7.0 15
%% 12.7.0 5
%% 12.7.0 6
%% 12.7.0 7

\tablefigthree{Bit-wise Logical Operations on Integers}%
{Function}{Identity}{Operation performed}{
 \funref{logandc1} & ---    & and complement of \param{integer-1} with \param{integer-2} \cr
 \funref{logandc2} & ---    & and \param{integer-1} with complement of \param{integer-2} \cr
 \funref{logand}   & \f{-1} & and                                                        \cr
 \funref{logeqv}   & \f{-1} & equivalence (exclusive nor)                                \cr
 \funref{logior}   & \f{0}  & inclusive or                                               \cr
 \funref{lognand}  & ---    & complement of \param{integer-1} and \param{integer-2}      \cr
 \funref{lognor}   & ---    & complement of \param{integer-1} or \param{integer-2}       \cr
 \funref{lognot}   & ---    & complement                                                 \cr
 \funref{logorc1}  & ---    & or complement of \param{integer-1} with \param{integer-2}  \cr
 \funref{logorc2}  & ---    & or \param{integer-1} with complement of \param{integer-2}  \cr
 \funref{logxor}   & \f{0}  & exclusive or                                               \cr
}

Negative \param{integers} are treated as if they were in two's-complement notation.

\label Examples::

\code
 (logior 1 2 4 8) \EV 15
 (logxor 1 3 7 15) \EV 10
 (logeqv) \EV -1
 (logand 16 31) \EV 16
 (lognot 0) \EV -1
 (lognot 1) \EV -2
 (lognot -1) \EV 0
 (lognot (1+ (lognot 1000))) \EV 999

;;; In the following example, m is a mask.  For each bit in
;;; the mask that is a 1, the corresponding bits in x and y are
;;; exchanged.  For each bit in the mask that is a 0, the 
;;; corresponding bits of x and y are left unchanged.
 (flet ((show (m x y)
          (format t "~%m = #o~6,'0O~%x = #o~6,'0O~%y = #o~6,'0O~%"
                  m x y)))
   (let ((m #o007750)
         (x #o452576)
         (y #o317407))
     (show m x y)
     (let ((z (logand (logxor x y) m)))
       (setq x (logxor z x))
       (setq y (logxor z y))
       (show m x y))))
\OUT m = #o007750
\OUT x = #o452576
\OUT y = #o317407
\OUT 
\OUT m = #o007750
\OUT x = #o457426
\OUT y = #o312557
\EV NIL
\endcode

\label Side Effects:\None.

\label Affected By:\None.

\label Exceptional Situations::

Should signal \typeref{type-error} if any argument is not an \term{integer}.

\label See Also::

\funref{boole}

\label Notes::

%Per Barmar.
\f{(logbitp \param{k} -1)} returns \term{true} for all values of \param{k}.

%% 12.7.0 9
Because the following functions are not associative,
they take exactly two arguments rather than any number
of arguments.

\code
 (lognand \param{n1} \param{n2}) \EQ (lognot (logand \param{n1} \param{n2}))
 (lognor \param{n1} \param{n2}) \EQ (lognot (logior \param{n1} \param{n2}))
 (logandc1 \param{n1} \param{n2}) \EQ (logand (lognot \param{n1}) \param{n2})
 (logandc2 \param{n1} \param{n2}) \EQ (logand \param{n1} (lognot \param{n2}))
 (logiorc1 \param{n1} \param{n2}) \EQ (logior (lognot \param{n1}) \param{n2})
 (logiorc2 \param{n1} \param{n2}) \EQ (logior \param{n1} (lognot \param{n2}))
 (logbitp \param{j} (lognot \param{x})) \EQ (not (logbitp \param{j} \param{x}))
\endcode

\endcom

%%% ========== LOGBITP
\begincom{logbitp}\ftype{Function}

\label Syntax::

\DefunWithValues logbitp {index integer} {generalized-boolean}

\label Arguments and Values::

\issue{ARGUMENTS-UNDERSPECIFIED:SPECIFY}
\param{index}---a non-negative \term{integer}.
\endissue{ARGUMENTS-UNDERSPECIFIED:SPECIFY}

\param{integer}---an \term{integer}.

\param{generalized-boolean}---a \term{generalized boolean}.

\label Description::

\funref{logbitp} is used to test the value of a particular bit 
in \param{integer}, that is treated as if it were binary.
%% 12.7.0 17
The value of \funref{logbitp} is \term{true} if the bit in \param{integer} 
whose index is \param{index} (that is, its weight is $2^\i{index}$) 
is a one-bit; otherwise it is \term{false}.

Negative \param{integers} are treated as if they were in 
two's-complement notation.

\label Examples::
\code
 (logbitp 1 1) \EV \term{false}
 (logbitp 0 1) \EV \term{true}
 (logbitp 3 10) \EV \term{true}
 (logbitp 1000000 -1) \EV \term{true}
 (logbitp 2 6) \EV \term{true}
 (logbitp 0 6) \EV \term{false}
\endcode

\label Side Effects:\None.

\label Affected By:\None.

\label Exceptional Situations::

\Shouldchecktype{index}{a non-negative \term{integer}}
\Shouldchecktype{integer}{an \term{integer}}

\label See Also:\None.

\label Notes::

\code
 (logbitp \param{k} \param{n}) \EQ (ldb-test (byte 1 \param{k}) \param{n})
\endcode

\endcom

%%% ========== LOGCOUNT
\begincom{logcount}\ftype{Function}

\label Syntax::

\DefunWithValues logcount {integer} {number-of-on-bits}

\label Arguments and Values:: 

\param{integer}---an \term{integer}.

\param{number-of-on-bits}---a non-negative \term{integer}.

\label Description::

%% 12.7.0 21
Computes and returns the number of bits 
in the two's-complement binary representation of \param{integer}
that are `on' or `set'.
If \param{integer} is negative, the \f{0} bits are counted;
otherwise, the \f{1} bits are counted.

\label Examples::

\code
 (logcount 0) \EV 0
 (logcount -1) \EV 0
 (logcount 7) \EV 3
 (logcount  13) \EV 3 ;Two's-complement binary: ...0001101
 (logcount -13) \EV 2 ;Two's-complement binary: ...1110011
 (logcount  30) \EV 4 ;Two's-complement binary: ...0011110
 (logcount -30) \EV 4 ;Two's-complement binary: ...1100010
 (logcount (expt 2 100)) \EV 1
 (logcount (- (expt 2 100))) \EV 100
 (logcount (- (1+ (expt 2 100)))) \EV 1
\endcode

\label Side Effects:\None.

\label Affected By:\None.

\label Exceptional Situations::

Should signal \typeref{type-error} if its argument is not an \term{integer}.

\label See Also:\None.

\label Notes::

Even if the \term{implementation} does not represent \term{integers} internally
in two's complement binary, \funref{logcount} behaves as if it did.

The following identity always holds:

\code
    (logcount \param{x})
 \EQ (logcount (- (+ \param{x} 1)))
 \EQ (logcount (lognot \param{x}))
\endcode

\endcom

%%% ========== LOGTEST
\begincom{logtest}\ftype{Function}

\label Syntax::

\DefunWithValues logtest {integer-1 integer-2} {generalized-boolean}

\label Arguments and Values::

\param{integer-1}---an \term{integer}.

\param{integer-2}---an \term{integer}.

\param{generalized-boolean}---a \term{generalized boolean}.

\label Description::

%% 12.7.0 16
Returns \term{true} if any of the bits designated by the 1's 
in \param{integer-1} is 1 in \param{integer-2};
otherwise it is \term{false}.
\param{integer-1} and \param{integer-2} are treated as if they were binary.

Negative \param{integer-1} and \param{integer-2} are treated as if
they were represented in two's-complement binary.

\label Examples::

\code
 (logtest 1 7) \EV \term{true}
 (logtest 1 2) \EV \term{false}
 (logtest -2 -1) \EV \term{true}
 (logtest 0 -1) \EV \term{false}
\endcode

\label Side Effects:\None.

\label Affected By:\None.

\label Exceptional Situations::

\Shouldchecktype{integer-1}{an \term{integer}}
\Shouldchecktype{integer-2}{an \term{integer}}

\label See Also:\None.

\label Notes::

\code
 (logtest \param{x} \param{y}) \EQ (not (zerop (logand \param{x} \param{y})))
\endcode

\endcom

%-------------------- Integer Bytes --------------------

%%% ========== BYTE
%%% ========== BYTE-SIZE
%%% ========== BYTE-POSITION
\begincom{byte, byte-size, byte-position}\ftype{Function}

\label Syntax::

\DefunWithValues byte {size position} {bytespec}

\DefunWithValues byte-size     {bytespec} {size}
\DefunWithValues byte-position {bytespec} {position}

\label Arguments and Values::

%% 12.8.0 2

\param{size}, \param{position}---a non-negative \term{integer}.

\param{bytespec}---a \term{byte specifier}.

\label Description::

%% 12.8.0 3
%% 12.8.0 2                 
\funref{byte} returns a \term{byte specifier} that indicates
a \term{byte} of width \param{size} and whose bits have weights            
$2^{\param{position} + \param{size} - 1\/}$ through $2^\param{position}$,
and whose representation is
\term{implementation-dependent}.                 

\funref{byte-size} returns the number of bits specified by \param{bytespec}.

\funref{byte-position} returns the position specified by \param{bytespec}.

\label Examples::

\code
 (setq b (byte 100 200)) \EV #<BYTE-SPECIFIER size 100 position 200>
 (byte-size b) \EV 100
 (byte-position b) \EV 200
\endcode

\label Affected By:\None.

\label Exceptional Situations:\None.

\label See Also::

\funref{ldb}, \funref{dpb}

\label Notes::
%% 12.8.0 4

\code
 (byte-size (byte \param{j} \param{k})) \EQ \param{j}
 (byte-position (byte \param{j} \param{k})) \EQ \param{k}
\endcode

A \term{byte} of \term{size} of \f{0} is permissible; 
it refers to a \term{byte} of width zero.  For example,

\code
 (ldb (byte 0 3) #o7777) \EV 0
 (dpb #o7777 (byte 0 3) 0) \EV 0
\endcode

\endcom

%%% ========== DEPOSIT-FIELD
\begincom{deposit-field}\ftype{Function}

%% 12.8.0 12
\label Syntax::

\DefunWithValues deposit-field {newbyte bytespec integer} {result-integer}

\label Arguments and Values::

\param{newbyte}---an \term{integer}.

\param{bytespec}---a \term{byte specifier}.

\param{integer}---an \term{integer}.

\param{result-integer}---an \term{integer}.

\label Description::

%% 12.8.0 13
Replaces a field of bits within \param{integer}; specifically,
returns an \term{integer} that contains the bits of \param{newbyte}
within the \term{byte} specified by \param{bytespec},
and elsewhere contains the bits of \param{integer}.  

\label Examples::

\code
 (deposit-field 7 (byte 2 1) 0) \EV 6
 (deposit-field -1 (byte 4 0) 0) \EV 15
 (deposit-field 0 (byte 2 1) -3) \EV -7
\endcode

\label Side Effects:\None.

\label Affected By:\None.

\label Exceptional Situations:\None.

\label See Also::

\funref{byte},
\funref{dpb}

\label Notes::

\code
 (logbitp \param{j} (deposit-field \param{m} (byte \param{s} \param{p}) \param{n}))
 \EQ (if (and (>= \param{j} \param{p}) (< \param{j} (+ \param{p} \param{s})))
        (logbitp \param{j} \param{m})
        (logbitp \param{j} \param{n}))
\endcode

   \funref{deposit-field} is to \funref{mask-field} 
as \funref{dpb}           is to \funref{ldb}.

\endcom

%%% ========== DPB
\begincom{dpb}\ftype{Function}

\label Syntax::

\DefunWithValues dpb {newbyte bytespec integer} {result-integer}

\label Pronunciation::

    \pronounced{\stress{d\schwa }\Stress{pib}}
 or \pronounced{\stress{d\schwa }\Stress{p\schwa b}}
 or \pronounced{\Stress{d\harde}\Stress{p\harde}\Stress{b\harde}}

\label Arguments and Values::

\param{newbyte}---an \term{integer}.

\param{bytespec}---a \term{byte specifier}.

\param{integer}---an \term{integer}.

\param{result-integer}---an \term{integer}.

\label Description::

%% 12.8.0 11
\funref{dpb} (deposit byte) is used to 
replace a field of bits within \param{integer}.  
\funref{dpb} returns an \term{integer} that is 
the same as \param{integer} except in the bits specified by \param{bytespec}.

Let \f{s} be the size specified
by \param{bytespec}; then the low \f{s} bits of \param{newbyte} appear in
the result in the byte specified by \param{bytespec}.
\param{Newbyte} is interpreted as
being right-justified, as if it were the result of \funref{ldb}.

\label Examples::

\code
 (dpb 1 (byte 1 10) 0) \EV 1024
 (dpb -2 (byte 2 10) 0) \EV 2048
 (dpb 1 (byte 2 10) 2048) \EV 1024
\endcode

\label Side Effects:\None.

\label Affected By:\None.

\label Exceptional Situations:\None.

\label See Also::

\funref{byte}, \funref{deposit-field}, \funref{ldb}

\label Notes::

\code
 (logbitp \param{j} (dpb \param{m} (byte \param{s} \param{p}) \param{n}))
 \EQ (if (and (>= \param{j} \param{p}) (< \param{j} (+ \param{p} \param{s})))
        (logbitp (- \param{j} \param{p}) \param{m})
        (logbitp \param{j} \param{n}))
\endcode

In general,

\code
 (dpb \param{x} (byte 0 \param{y}) \param{z}) \EV \param{z}
\endcode

for all valid values of \param{x}, \param{y}, and \param{z}.

Historically, the name ``dpb'' comes from a DEC PDP-10 assembly language
instruction meaning ``deposit byte.''

\endcom

%%% ========== LDB
\begincom{ldb}\ftype{Accessor}

\label Syntax::

\DefunWithValues ldb {bytespec integer} {byte}

\Defsetf ldb {bytespec place} {new-byte}

\label Pronunciation::

    \pronounced{\Stress{lid}ib}
 or \pronounced{\Stress{lid}\schwa b}
 or \pronounced{\Stress{el}\Stress{d\harde}\Stress{b\harde}}

\label Arguments and Values::

\param{bytespec}---a \term{byte specifier}.

\param{integer}---an \term{integer}.

\param{byte}, \param{new-byte}---a non-negative \term{integer}.

\label Description::

%% 12.8.0 5
\funref{ldb} extracts and returns the \term{byte} of \param{integer}
specified by \param{bytespec}.

\funref{ldb} returns an \term{integer} in which the bits with weights 
$2^{(\i{s}-1)}$ through $2^{0}$ are the same as those in 
\param{integer} with weights $2^{(\i{p}+\i{s}-1)}$ 
through $2^\i{p}$, and all other bits zero; \i{s} is
\f{(byte-size \param{bytespec})} 
and \i{p} is \f{(byte-position \param{bytespec})}.

%% 12.8.0 7

\macref{setf} may be used with \funref{ldb} to modify
a byte within the \param{integer} that is stored
in a given \param{place}.
\issue{PUSH-EVALUATION-ORDER:FIRST-ITEM}
The order of evaluation, when an \funref{ldb} form is supplied
to \macref{setf}, is exactly left-to-right.
\idxtext{order of evaluation}\idxtext{evaluation order}%
\endissue{PUSH-EVALUATION-ORDER:FIRST-ITEM}
The effect is to perform a \funref{dpb} operation
and then store the result back into the \param{place}.

\label Examples::

\code
 (ldb (byte 2 1) 10) \EV 1
 (setq a (list 8)) \EV (8)
 (setf (ldb (byte 2 1) (car a)) 1) \EV 1
 a \EV (10)
\endcode

\label Side Effects:\None.

\label Affected By:\None.

\label Exceptional Situations:\None.

\label See Also::

\funref{byte},
\funref{byte-position},
\funref{byte-size},
\funref{dpb}

\label Notes::

\code
 (logbitp \param{j} (ldb (byte \param{s} \param{p}) \param{n}))
    \EQ (and (< \param{j} \param{s}) (logbitp (+ \param{j} \param{p}) \param{n}))
\endcode

In general,

\code
 (ldb (byte 0 \param{x}) \param{y}) \EV 0
\endcode

for all valid values of \param{x} and \param{y}.

Historically, the name ``ldb'' comes from a DEC PDP-10 assembly language
instruction meaning ``load byte.''

\endcom

%%% ========== LDB-TEST
\begincom{ldb-test}\ftype{Function}              

\label Syntax::

\DefunWithValues ldb-test {bytespec integer} {generalized-boolean}

\label Arguments and Values::

\param{bytespec}---a \term{byte specifier}.

\param{integer}---an \term{integer}.

\param{generalized-boolean}---a \term{generalized boolean}.

\label Description::

%% 12.8.0 8
Returns \term{true} if any of the bits of the byte in \param{integer}
specified by \param{bytespec} is non-zero; otherwise returns \term{false}.

\label Examples::

\code
 (ldb-test (byte 4 1) 16) \EV \term{true}
 (ldb-test (byte 3 1) 16) \EV \term{false}
 (ldb-test (byte 3 2) 16) \EV \term{true}
\endcode


\label Side Effects:\None.

\label Affected By:\None.

\label Exceptional Situations:\None.

\label See Also::

\funref{byte}, \funref{ldb}, \funref{zerop}

\label Notes::
\code
 (ldb-test bytespec n) \EQ
 (not (zerop (ldb bytespec n))) \EQ
 (logtest (ldb bytespec -1) n)
\endcode
%Item 3 of that equivalence contributed by Barmar. -kmp 16-Jul-91

\endcom

%%% ========== MASK-FIELD
\begincom{mask-field}\ftype{Accessor}

\label Syntax::

\DefunWithValues mask-field {bytespec integer} {masked-integer}
\Defsetf         mask-field {bytespec place}   {new-masked-integer}

\label Arguments and Values::

\param{bytespec}---a \term{byte specifier}.

\param{integer}---an \term{integer}.

\param{masked-integer}, \param{new-masked-integer}---a non-negative \term{integer}.

\label Description::

%% 12.8.0 9
\funref{mask-field} performs a ``mask'' operation on \param{integer}.
It returns an \term{integer} that has the same bits as \param{integer} in 
the \term{byte} specified by \param{bytespec}, but that has zero-bits everywhere else.

%% 12.8.0 10
\macref{setf} may be used with \funref{mask-field}
to modify a byte within the \term{integer} that is stored
in a given \param{place}.
The effect is to perform a \funref{deposit-field} operation
and then store the result back into the \param{place}.

\label Examples::

\code
 (mask-field (byte 1 5) -1) \EV 32
 (setq a 15) \EV 15
 (mask-field (byte 2 0) a) \EV 3
 a \EV 15
 (setf (mask-field (byte 2 0) a) 1) \EV 1
 a \EV 13
\endcode

\label Side Effects:\None.

\label Affected By:\None.

\label Exceptional Situations:\None.

\label See Also::

\funref{byte},
\funref{ldb}

\label Notes::

\code
 (ldb \param{bs} (mask-field \param{bs} \param{n})) \EQ (ldb \param{bs} \param{n})
 (logbitp \param{j} (mask-field (byte \param{s} \param{p}) \param{n}))
   \EQ (and (>= \param{j} \param{p}) (< \param{j} \param{s}) (logbitp \param{j} \param{n}))
 (mask-field \param{bs} \param{n}) \EQ (logand \param{n} (dpb -1 \param{bs} 0))
\endcode

\endcom

%-------------------- Fixnum --------------------

%%% ========== MOST-POSITIVE-FIXNUM
%%% ========== MOST-NEGATIVE-FIXNUM
\begincom{most-positive-fixnum, most-negative-fixnum}\ftype{Constant Variable}

\label Constant Value::

\term{implementation-dependent}.

\label Description::

%% 12.10.0 1
%% 12.10.0 2
\conref{most-positive-fixnum} is that \term{fixnum} closest in value 
to positive infinity provided by the implementation,
\issue{FIXNUM-NON-PORTABLE:TIGHTEN-DEFINITION}
and greater than or equal to both $2^{15}$ - 1 and 
\conref{array-dimension-limit}.
\endissue{FIXNUM-NON-PORTABLE:TIGHTEN-DEFINITION}

%% 12.10.0 4
\conref{most-negative-fixnum} is that \term{fixnum} closest in value
to negative infinity provided by the implementation,
\issue{FIXNUM-NON-PORTABLE:TIGHTEN-DEFINITION}
and less than or equal to $-2^{15}$.
\endissue{FIXNUM-NON-PORTABLE:TIGHTEN-DEFINITION}

\label Examples:\None.

\label See Also:\None.

\label Notes:\None.

\endcom

%-------------------- Ratio --------------------



%-------------------- Float --------------------

%%% ========== DECODE-FLOAT
%%% ========== SCALE-FLOAT
%%% ========== FLOAT-RADIX
%%% ========== FLOAT-SIGN
%%% ========== FLOAT-DIGITS
%%% ========== FLOAT-PRECISION
%%% ========== INTEGER-DECODE-FLOAT
\begincom{decode-float, scale-float, float-radix, float-sign, 
	  float-digits, float-precision, integer-decode-float}\ftype{Function}

\label Syntax::

\DefunWithValues decode-float {float} {significand, exponent, sign}

\DefunWithValues scale-float {float integer} {scaled-float}

\DefunWithValues float-radix {float} {float-radix}

\DefunWithValues float-sign {float-1 {\opt} float-2} {signed-float}

\DefunWithValues float-digits {float} {digits1}

\DefunWithValues float-precision {float} {digits2}

\DefunWithValues integer-decode-float {float} {significand, exponent, integer-sign}

\label Arguments and Values::

\param{digits1}---a non-negative \term{integer}.

\param{digits2}---a non-negative \term{integer}.

\param{exponent}---an \term{integer}.

\param{float}---a \term{float}.

\param{float-1}---a \term{float}.

\param{float-2}---a \term{float}.

\param{float-radix}---an \term{integer}.

\param{integer}---a non-negative \term{integer}.

\param{integer-sign}---the \term{integer} \f{-1},
		    or the \term{integer} \f{1}.

\param{scaled-float}---a \term{float}.

\param{sign}---A \term{float} of the same \term{type} as \param{float}
	       but numerically equal to \f{1.0} or \f{-1.0}.

\param{signed-float}---a \term{float}.
                          
\param{significand}---a \term{float}.

\label Description::

%% 12.6.0 27
%% 12.6.0 28
\funref{decode-float} computes three values that characterize
\param{float}.
The first value is of the same \term{type}
as \param{float} and 
represents the significand.
The second value represents the exponent
to which the radix (notated in this description by \i{b}) must
be raised to obtain the value that, when multiplied with the first
result, produces the absolute value of \param{float}.
If \param{float} is zero, any \term{integer} value may be returned, 
provided that the identity shown for \funref{scale-float} holds.
The third value
is of the same \term{type} as \param{float}
and is 1.0 if \param{float} is greater
than or equal to zero or -1.0 otherwise.

\funref{decode-float} 
divides \param{float} by an integral power of \i{b}
so as to bring its value between $1/\i{b}$ (inclusive) and~$1$ (exclusive),
and returns the quotient as the first value.
If \param{float} is zero, however, the result
equals the absolute value of \param{float} (that is, if there is a negative
zero, its significand is considered to be a positive zero).

%% 12.6.0 29
\funref{scale-float} returns
{\tt (* \param{float} (expt (float \i{b} \param{float}) 
\param{integer}))\/}, where \i{b} is the radix of the floating-point
representation. \param{float} is not necessarily between $1/\i{b}$ and~$1$.
 

%% 12.6.0 31              
\funref{float-radix} returns 
the radix of \param{float}.

%% 12.6.0 32
\funref{float-sign} returns a number \f{z} such
that \f{z} and \param{float-1} have the same sign and also such that
\f{z} and \param{float-2} have the same absolute value.
If \param{float-2} is not supplied, its value is \f{(float 1 \param{float-1})}.
%% Barmar says this is redundant. I agree. -kmp 28-Dec-90
% \f{(float-sign x)} produces a \f{1.0} or \f{-1.0}
% of appropriate format
% according to the sign of \f{x} if \param{float-2} is not supplied.
If an implementation
has distinct representations for negative zero and positive zero,
then \f{(float-sign -0.0)} \EV\ \f{-1.0}.

%% 12.6.0 33
\funref{float-digits} returns
the number of radix \i{b} digits
used in the representation of \param{float} (including any implicit
digits, such as a ``hidden bit'').

\funref{float-precision}
returns
the number of significant radix \i{b} digits present in \param{float};
if \param{float} is a \term{float}
zero, then the result is an \term{integer} zero.

For \term{normalized} \term{floats},
the results of \funref{float-digits} and \funref{float-precision} are the same,
but the precision is less than the number of representation digits 
for a \term{denormalized} or zero number.

%% 12.6.0 34
%% 12.6.0 35
%% 12.6.0 36
\funref{integer-decode-float} computes three values that characterize
\param{float} - 
the significand scaled so as to be an \term{integer},
and the same last two
values that are returned by \funref{decode-float}.
If \param{float} is zero, \funref{integer-decode-float} returns 
zero as the first value.
The second value bears the same relationship to the first value
as for \funref{decode-float}:

\code
 (multiple-value-bind (signif expon sign)
                      (integer-decode-float f)
   (scale-float (float signif f) expon)) \EQ (abs f)
\endcode

\label Examples::

\code
 ;; Note that since the purpose of this functionality is to expose
 ;; details of the implementation, all of these examples are necessarily
 ;; very implementation-dependent.  Results may vary widely.
 ;; Values shown here are chosen consistently from one particular implementation.
 (decode-float .5) \EV 0.5, 0, 1.0
 (decode-float 1.0) \EV 0.5, 1, 1.0
 (scale-float 1.0 1) \EV 2.0
 (scale-float 10.01 -2) \EV 2.5025
 (scale-float 23.0 0) \EV 23.0
 (float-radix 1.0) \EV 2
 (float-sign 5.0) \EV 1.0
 (float-sign -5.0) \EV -1.0
 (float-sign 0.0) \EV 1.0
 (float-sign 1.0 0.0) \EV 0.0
 (float-sign 1.0 -10.0) \EV 10.0
 (float-sign -1.0 10.0) \EV -10.0
 (float-digits 1.0) \EV 24
 (float-precision 1.0) \EV 24
 (float-precision least-positive-single-float) \EV 1
 (integer-decode-float 1.0) \EV 8388608, -23, 1
\endcode

\label Side Effects:\None.

\label Affected By::

The implementation's representation for \term{floats}.

\label Exceptional Situations::

The functions \funref{decode-float}, \funref{float-radix}, \funref{float-digits},
\funref{float-precision}, and \funref{integer-decode-float} should signal an error
if their only argument is not a \term{float}.

\Thefunction{scale-float} should signal an error if its first argument
is not a \term{float} or if its second argument is not an \term{integer}.

\Thefunction{float-sign} should signal an error if its first argument
is not a \term{float} or if its second argument is supplied but is 
not a \term{float}.

\label See Also:\None.

\label Notes::

The product of the first result of \funref{decode-float} or \funref{integer-decode-float},
of the radix raised to the power of the second result, and of the third result
is exactly equal to the value of \param{float}.

%% 12.6.0 30
\code
 (multiple-value-bind (signif expon sign)
                      (decode-float f)
   (scale-float signif expon))
\EQ (abs f)
\endcode
and

\code
 (multiple-value-bind (signif expon sign)
                      (decode-float f)
   (* (scale-float signif expon) sign))
\EQ f
\endcode
\endcom

%%% ========== FLOAT
\begincom{float}\ftype{Function}

\label Syntax::

\DefunWithValues float {number {\opt} prototype} {float}

\label Arguments and Values::

\param{number}---a \term{real}.

\param{prototype}---a \term{float}.

\param{float}---a \term{float}.

\label Description::

%% 12.6.0 4
\funref{float} converts a 
\issue{REAL-NUMBER-TYPE:X3J13-MAR-89}
\term{real}
\endissue{REAL-NUMBER-TYPE:X3J13-MAR-89}
number to a \term{float}.

If a \param{prototype} is supplied,
a \term{float} is returned that is mathematically equal to \param{number}
but has the same format as \param{prototype}.

If \param{prototype} is not supplied,
then if the \param{number} is already a \term{float}, it is returned;
otherwise, a \term{float} is returned that is mathematically equal to \param{number}
but is a \term{single float}.

\label Examples:: 

\code
 (float 0) \EV 0.0
 (float 1 .5) \EV 1.0
 (float 1.0) \EV 1.0
 (float 1/2) \EV 0.5
\EV 1.0d0
\OV 1.0
 (eql (float 1.0 1.0d0) 1.0d0) \EV \term{true}
\endcode

\label Side Effects:\None.

\label Affected By:\None.

\label Exceptional Situations:\None.

\label See Also::

\funref{coerce}

\label Notes:\None.

\endcom

%%% ========== FLOATP
\begincom{floatp}\ftype{Function}

\label Syntax::

\Defun floatp {object} {generalized-boolean}

\label Arguments and Values::

\param{object}---an \term{object}.

\param{generalized-boolean}---a \term{generalized boolean}.

\label Description::

%% 6.2.2 15
\TypePredicate{object}{float}

\label Examples::

\code
 (floatp 1.2d2) \EV \term{true}
 (floatp 1.212) \EV \term{true}
 (floatp 1.2s2) \EV \term{true}
 (floatp (expt 2 130)) \EV \term{false}
\endcode

\label Side Effects:\None.

\label Affected By:\None.

\label Exceptional Situations:\None!

\label See Also:\None.

\label Notes::

\code
 (floatp \param{object}) \EQ (typep \param{object} 'float)
\endcode

\endcom

%-------------------- Float Subtypes --------------------

%%% ========== LEAST-NEGATIVE-DOUBLE-FLOAT
%%% ========== LEAST-NEGATIVE-NORMALIZED-DOUBLE-FLOAT
%%% ========== LEAST-NEGATIVE-LONG-FLOAT
%%% ========== LEAST-NEGATIVE-NORMALIZED-LONG-FLOAT
%%% ========== LEAST-NEGATIVE-SHORT-FLOAT
%%% ========== LEAST-NEGATIVE-NORMALIZED-SHORT-FLOAT
%%% ========== LEAST-NEGATIVE-SINGLE-FLOAT
%%% ========== LEAST-NEGATIVE-NORMALIZED-SINGLE-FLOAT
%%% ========== LEAST-POSITIVE-DOUBLE-FLOAT
%%% ========== LEAST-POSITIVE-NORMALIZED-DOUBLE-FLOAT
%%% ========== LEAST-POSITIVE-LONG-FLOAT
%%% ========== LEAST-POSITIVE-NORMALIZED-LONG-FLOAT
%%% ========== LEAST-POSITIVE-SHORT-FLOAT
%%% ========== LEAST-POSITIVE-NORMALIZED-SHORT-FLOAT
%%% ========== LEAST-POSITIVE-SINGLE-FLOAT
%%% ========== LEAST-POSITIVE-NORMALIZED-SINGLE-FLOAT
%%% ========== MOST-NEGATIVE-DOUBLE-FLOAT
%%% ========== MOST-NEGATIVE-LONG-FLOAT
%%% ========== MOST-NEGATIVE-SHORT-FLOAT
%%% ========== MOST-NEGATIVE-SINGLE-FLOAT
%%% ========== MOST-POSITIVE-DOUBLE-FLOAT
%%% ========== MOST-POSITIVE-LONG-FLOAT
%%% ========== MOST-POSITIVE-SHORT-FLOAT
%%% ========== MOST-POSITIVE-SINGLE-FLOAT
\begincom{most-positive-short-float, least-positive-short-float,
	  least-positive-normalized-short-float,
          most-positive-double-float, least-positive-double-float,
	  least-positive-normalized-double-float,
          most-positive-long-float, least-positive-long-float, 
	  least-positive-normalized-long-float,
          most-positive-single-float, least-positive-single-float,
	  least-positive-normalized-single-float,
          most-negative-short-float, least-negative-short-float,
	  least-negative-normalized-short-float,
          most-negative-single-float, least-negative-single-float,
	  least-negative-normalized-single-float,
          most-negative-double-float, least-negative-double-float,
	  least-negative-normalized-double-float,
          most-negative-long-float, least-negative-long-float,
	  least-negative-normalized-long-float}\ftype{Constant Variable}

\issue{FLOAT-UNDERFLOW:ADD-VARIABLES}

\label Constant Value::

\term{implementation-dependent}.

\label Description::
%% 12.10.0 9

These \term{constant variables} provide a way for programs to examine
the \term{implementation-defined} limits for the various float formats.

Of these \term{variables},
 each which has ``\f{-normalized}'' in its \term{name}
  must have a \term{value} which is a \term{normalized} \term{float}, and
 each which does not have ``\f{-normalized}'' in its name
  may have a \term{value} which is either a \term{normalized} \term{float} 
  or a \term{denormalized} \term{float}, as appropriate.

Of these \term{variables},
 each which has ``\f{short-float}'' in its name
  must have a \term{value} which is a \term{short float},
 each which has ``\f{single-float}'' in its name
  must have a \term{value} which is a \term{single float},
 each which has ``\f{double-float}'' in its name
  must have a \term{value} which is a \term{double float}, and
 each which has ``\f{long-float}'' in its name
  must have a \term{value} which is a \term{long float}.

\beginlist

\itemitem{\bull} 
  \vtop{\hbox{\conref{most-positive-short-float},
              \conref{most-positive-single-float},}
        \hbox{\conref{most-positive-double-float},
	      \conref{most-positive-long-float}}}
\Vskip 18pt!

  Each of these \term{constant variables} has as its \term{value} 
  the positive \term{float} of the largest magnitude 
  (closest in value to, but not equal to, positive infinity)
  for the float format implied by its name.

\itemitem{\bull}
  \vtop{\hbox{\conref{least-positive-short-float},
	      \conref{least-positive-normalized-short-float},}
	\hbox{\conref{least-positive-single-float},
	      \conref{least-positive-normalized-single-float},}
	\hbox{\conref{least-positive-double-float},
	      \conref{least-positive-normalized-double-float},}
	\hbox{\conref{least-positive-long-float},
	      \conref{least-positive-normalized-long-float}}}
\Vskip 42pt!

  Each of these \term{constant variables} has as its \term{value}
  the smallest positive (nonzero) \term{float}
  for the float format implied by its name.

\itemitem{\bull}
  \vtop{\hbox{\conref{least-negative-short-float},
	      \conref{least-negative-normalized-short-float},}
        \hbox{\conref{least-negative-single-float},
	      \conref{least-negative-normalized-single-float},}
        \hbox{\conref{least-negative-double-float},
	      \conref{least-negative-normalized-double-float},}
        \hbox{\conref{least-negative-long-float},
	      \conref{least-negative-normalized-long-float}}}
\Vskip 42pt!

  Each of these \term{constant variables} has as its \term{value}
  the negative (nonzero) \term{float} of the smallest magnitude
  for the float format implied by its name.
  (If an implementation supports minus zero as a \term{different} 
   \term{object} from positive zero, this value must not be minus zero.)

\itemitem{\bull}
  \vtop{\hbox{\conref{most-negative-short-float},
	      \conref{most-negative-single-float},}
	\hbox{\conref{most-negative-double-float},
	      \conref{most-negative-long-float}}}
\Vskip 18pt!

  Each of these \term{constant variables} has as its \term{value}
  the negative \term{float} of the largest magnitude 
  (closest in value to, but not equal to, negative infinity)
  for the float format implied by its name.

\endlist

\label Examples:\None.

\label See Also:\None.

\label Notes::

\endissue{FLOAT-UNDERFLOW:ADD-VARIABLES}

\endcom

%%% ========== SHORT-FLOAT-EPSILON
%%% ========== SHORT-FLOAT-NEGATIVE-EPSILON
%%% ========== SINGLE-FLOAT-EPSILON
%%% ========== SINGLE-FLOAT-NEGATIVE-EPSILON
%%% ========== DOUBLE-FLOAT-EPSILON
%%% ========== DOUBLE-FLOAT-NEGATIVE-EPSILON
%%% ========== LONG-FLOAT-EPSILON
%%% ========== LONG-FLOAT-NEGATIVE-EPSILON
\begincom{short-float-epsilon, short-float-negative-epsilon,
          single-float-epsilon, single-float-negative-epsilon,
          double-float-epsilon, double-float-negative-epsilon,
          long-float-epsilon, long-float-negative-epsilon}\ftype{Constant Variable}

\label Constant Value::

%% 12.10.0 10
\term{implementation-dependent}.

\label Description::

The value of each of the constants \conref{short-float-epsilon}, 
\conref{single-float-epsilon},
\conref{double-float-epsilon}, and \conref{long-float-epsilon} is 
the smallest positive \term{float} $\epsilon$ of the given format, 
such that the following expression is \term{true} when evaluated:

{\tt (not (= (float 1 $\epsilon$) (+ (float 1 $\epsilon$) $\epsilon$)))\/}


The value of each of the constants \conref{short-float-negative-epsilon}, 
\conref{single-float-negative-epsilon},
\conref{double-float-negative-epsilon}, and  
\conref{long-float-negative-epsilon} is the smallest positive
\term{float} $\epsilon$ of the given format, such that the following
expression is \term{true} when evaluated:

{\tt (not (= (float 1 $\epsilon$) (- (float 1 $\epsilon$) $\epsilon$)))\/}


\label Examples:\None.

\label See Also:\None.

\label Notes:\None.

\endcom

%%% ========== ARITHMETIC-ERROR
\begincom{arithmetic-error}\ftype{Condition Type}

\label Class Precedence List::
\typeref{arithmetic-error},
\typeref{error},
\typeref{serious-condition},
\typeref{condition},
\typeref{t}

\label Description::

\Thetype{arithmetic-error} consists of error conditions
that occur during arithmetic operations.
The operation and operands are initialized with
\theinitkeyargs{operation} and \kwd{operands} to \funref{make-condition},
and are \term{accessed} by
the functions \funref{arithmetic-error-operation} and 
\funref{arithmetic-error-operands}.

\label See Also::

\funref{arithmetic-error-operation}, \funref{arithmetic-error-operands}

\endcom%{arithmetic-error}\ftype{Condition Type}

%%% ========== ARITHMETIC-ERROR-OPERANDS
\begincom{arithmetic-error-operands, arithmetic-error-operation}\ftype{Function}

\label Syntax::

\DefunWithValues arithmetic-error-operands {condition} {operands}
\DefunWithValues arithmetic-error-operation {condition} {operation}

\label Arguments and Values::

\param{condition}---a \term{condition} \oftype{arithmetic-error}.

\param{operands}---a \term{list}.

\param{operation}---a \term{function designator}.

\label Description::

\funref{arithmetic-error-operands} returns a \term{list} of the operands 
which were used in the offending call to the operation that signaled 
the \param{condition}.

\funref{arithmetic-error-operation} returns a \term{list} of 
the offending operation in the offending call that signaled the \param{condition}.

\label Examples:\None.

\label Side Effects:\None.

\label Affected By:\None!

\label Exceptional Situations:\None.

\label See Also::

\typeref{arithmetic-error},
{\chapref\Conditions}

\label Notes::

%%Shouldn't be needed any more. -kmp 1-Sep-91
%It is an error to use \macref{setf} with \funref{arithmetic-error-operands}.

\endcom

\begincom{division-by-zero}\ftype{Condition Type}

\label Class Precedence List::
\typeref{division-by-zero},
\typeref{arithmetic-error},
\typeref{error},
\typeref{serious-condition},
\typeref{condition},
\typeref{t}

\label Description::

\Thetype{division-by-zero} consists of error conditions that
occur because of division by zero.

\endcom%{division-by-zero}\ftype{Condition Type}

%!!! ACW: A statement of the intended difference between
%         FLOATING-POINT-INVALID-OPERATION and FLOATING-POINT-INEXACT would be helpful.
\begincom{floating-point-invalid-operation}\ftype{Condition Type}

\issue{FLOATING-POINT-CONDITION-NAMES:X3J13-NOV-89}

\label Class Precedence List::
\typeref{floating-point-invalid-operation},
\typeref{arithmetic-error},
\typeref{error},
\typeref{serious-condition},
\typeref{condition},
\typeref{t}

\label Description::

\Thetype{floating-point-invalid-operation} consists of
error conditions that occur because of certain 
%!!! Barmar: which?
floating point traps. 

It is \term{implementation-dependent} whether floating point traps
occur, and whether or how they may be enabled or disabled.  Therefore,
conforming code may establish handlers for this condition, but must not
depend on its being \term{signaled}.

\endissue{FLOATING-POINT-CONDITION-NAMES:X3J13-NOV-89}

\endcom%{floating-point-invalid-operation}\ftype{Condition Type}

\begincom{floating-point-inexact}\ftype{Condition Type}

\issue{FLOATING-POINT-CONDITION-NAMES:X3J13-NOV-89}

\label Class Precedence List::
\typeref{floating-point-inexact},
\typeref{arithmetic-error},
\typeref{error},
\typeref{serious-condition},
\typeref{condition},
\typeref{t}

\label Description::

\Thetype{floating-point-inexact} consists of 
error conditions that occur because of certain 
%!!! Barmar: Which?
floating point traps.

It is \term{implementation-dependent} whether floating point traps
occur, and whether or how they may be enabled or disabled.  Therefore,
conforming code may establish handlers for this condition, but must not
depend on its being \term{signaled}.

\endissue{FLOATING-POINT-CONDITION-NAMES:X3J13-NOV-89}

\endcom%{floating-point-inexact}\ftype{Condition Type}

\begincom{floating-point-overflow}\ftype{Condition Type}

\label Class Precedence List::
\typeref{floating-point-overflow},
\typeref{arithmetic-error},
\typeref{error},
\typeref{serious-condition},
\typeref{condition},
\typeref{t}

\label Description::

\Thetype{floating-point-overflow} consists of error
conditions that occur because of floating-point overflow.

\endcom%{floating-point-overflow}\ftype{Condition Type}

\begincom{floating-point-underflow}\ftype{Condition Type}

\label Class Precedence List::
\typeref{floating-point-underflow},
\typeref{arithmetic-error},
\typeref{error},
\typeref{serious-condition},
\typeref{condition},
\typeref{t}

\label Description::

\Thetype{floating-point-underflow} consists of 
error conditions that occur because of floating-point underflow.

\endcom%{floating-point-underflow}\ftype{Condition Type}

